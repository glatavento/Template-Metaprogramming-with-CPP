\section{了解SFINAE}
编写模板时,有时需要限制模板参数。例如,应该适用于任何数字类型,整型和浮点型,但不适用于其他类型。或者可以有一个类模板,只接受普通类型的参数。

还有一些可能有重载函数模板的情况,每个函数模板只适用于某些类型。例如,一种重载适用于整型,另一种重载适用于浮点类型。实现这一目标可以用不同的方法,我们将在本章和下一章中进行探讨。

然而,类型特征以某种方式存在于这些类型中。本章将讨论的第一个特性是SFINAE。另一种优于SFINAE的方法由概念表示,将在下一章中讨论。

SFINAE表示\textbf{替换失败不是错误}。当编译器遇到函数模板时,会替换实参以实例化模板。若此时发生错误,则只将其视为推导失败。函数会从重载集中移除,而不是引起错误。只有在重载集中没有匹配项时,才会产生错误。

没有具体的例子的情况下,很难真正理解SFINAE。这里,将通过几个例子来了解一下这个概念。

每个标准容器,比如std::vector、std::array和std::map,不仅有迭代器可以访问容器的元素,还可以修改容器(在迭代器指向的元素之后插入元素)。因此,这些容器具有成员函数,用于返回容器的第一个元素和最后一个元素的迭代器,这些函数就是begin和end。

还有其他函数,如cbegin和cend, rbegin和rend,以及crbegin和crend,不过这些函数超出了本章的主题范畴。C++11中,也有独立函数std:begin和std::end,可以完成同样的事情。这些函数不仅适用于标准容器,也适用于数组。这样做的一个好处是为数组启用基于范围的for循环。问题是如何实现这个非成员函数来同时使用容器和数组?当然,需要函数模板的两次重载。一个可能的实现如下所示:

\begin{cppcode}
template <typename T>
auto begin(T& c) { return c.begin(); } // [1]

template <typename T, size_t N>
T* begin(T(&arr)[N]) {return arr; } // [2]
\end{cppcode}

第一个重载调用成员函数begin并返回值,这种重载仅限于具有成员函数begin的类型;否则,编译器将报错。第二个重载只返回一个指向数组第一个元素的指针。这仅限于数组类型,任何其他操作都会产生编译器错误。可以这样使用这些重载:

\begin{cppcode}
std::array<int, 5> arr1{ 1,2,3,4,5 };
std::cout << *begin(arr1) << '\n'; // [3] prints 1

int arr2[]{ 5,4,3,2,1 };
std::cout << *begin(arr2) << '\n'; // [4] prints 5
\end{cppcode}

编译这段代码时,因为SFINAE编译过程不会出现错误,甚至不会出现警告。解析begin(arr1)调用时,将std::array<int, 5>替换为第一个重载([1])成功,但替换第二个重载([2])失败。编译器不会发出错误,而是忽略它,因此它用单个实例化构建一个重载集,所以可以成功地为调用找到匹配。类似地,解析begin(arr2)调用时,对第一个重载替换int[5]失败并忽略,但对第二个重载替换成功并添加到重载集,最终为调用找到一个良好的匹配。因此,两个调用都可以成功进行。若两个重载中有一个不存在,begin(arr1)或begin(arr2)将无法匹配函数模板,并且编译器会报错。

SFINAE只适用于函数的直接上下文。直接上下文基本上是模板声明(包括模板参数列表、函数返回类型和函数参数列表),不适用于函数体。看看下面的例子:

\begin{cppcode}
template <typename T>
void increment(T& val) { val++; }

int a = 42;
increment(a); // OK

std::string s{ "42" };
increment(s); // error
\end{cppcode}

在增量函数模板的直接上下文中,对类型T没有限制。函数体中,参数val用后修正操作符++加1。所以,将T替换为没有实现后修复操作符++的类型都是失败的。因为这个失败是一个错误,所以编译器不会忽略。

C++标准(许可证使用链接:\url{http://creativecommons.org/licenses/by-sa/3.0/})定义了SFINAE错误的错误列表(在§13.10.2,模板参数推导,C++20标准中)。以下行为为SFINAE错误:


\begin{itemize}
  \item 创建一个void数组、一个引用数组、一个函数数组、一个长度为负的数组、一个长度为0的数组和一个非整型长度的数组
  \item 在作用域解析操作符::左侧使用不是类或枚举的类型(例如:T::value_type中,T是一个数字类型)
  \item 创建指向引用的指针
  \item 创建对void的引用
  \item 创建指向T成员的指针,其中T不是类类型
  \item 类型不包含该成员时,使用类型的成员
  \item 使用类型的成员,其中类型是必需的,但该成员不是类型
  \item 使用需要模板,但成员不是模板类型的成员
  \item 需要非类型,但成员不是非类型的情况下使用类型的成员
  \item 创建具有void类型参数的函数类型
  \item 创建返回数组类型或其他函数类型的函数类型
  \item 模板参数表达式或函数声明中使用的表达式中进行了无效转换
  \item 为非类型模板形参提供无效类型
  \item 实例化包含不同长度的多个包的扩展
\end{itemize}

列表中的最后一个错误是在C++11中与可变参数模板一起引入的,其他的是在C++11之前定义的。我们不会继续逐一举例,说明所有这些错误,但需要再看几个例子。第一个问题涉及尝试创建一个长度为0的数组。假设想要有两个函数模板重载,一个处理偶数长度的数组,另一个处理奇数长度的数组。解决方法如下:

\begin{cppcode}
template <typename T, size_t N>
void handle(T(&arr)[N], char(*)[N % 2 == 0] = 0)
{
	std::cout << "handle even array\n";
}

template <typename T, size_t N>
void handle(T(&arr)[N], char(*)[N % 2 == 1] = 0)
{
	std::cout << "handle odd array\n";
}

int arr1[]{ 1,2,3,4,5 };
handle(arr1);

int arr2[]{ 1,2,3,4 };
handle(arr2);
\end{cppcode}

模板参数和第一个函数参数类似于我们看到的数组的begin,这些句柄重载有第二个匿名参数,其默认值为0。该参数的类型是一个指向char类型数组的指针,数组长度由表达式N\%2==0和N\%2==1指定。对于每一个可能的数组,这两个中的一个为真,另一个为假。第二个参数是char(*)[1]或char(*)[0],后者是SFINAE错误(试图创建一个长度为0的数组),因此才能够调用其他重载,而不会产生编译器错误。

本节的最后一个示例将展示SFINAE尝试使用一个不存在的类的成员:

\begin{cppcode}
template <typename T>
struct foo
{
	using foo_type = T;
};

template <typename T>
struct bar
{
	using bar_type = T;
};

struct int_foo : foo<int> {};
struct int_bar : bar<int> {}
\end{cppcode}

这里有两个类,foo的成员类型是foo_type,而bar的成员类型是bar_type。还有派生自这两个类。目标是编写两个函数模板,一个处理类的foo层次结构,另一个处理类的bar层次结构。一个可能的实现如下所示:

\begin{cppcode}
template <typename T>
decltype(typename T::foo_type(), void()) handle(T const& v)
{
	std::cout << "handle a foo\n";
}

template <typename T>
decltype(typename T::bar_type(), void()) handle(T const& v)
{
	std::cout << "handle a bar\n";
}
\end{cppcode}

两个重载都有一个模板参数和一个类型为T const\&的函数参数,也返回相同的类型,并且该类型为void。表达式decltype(typename T::foo_type(), void())可能需要思考一下才能更好地理解。我们在第4章中讨论了decltype,这是一个类型说明符,用于推断表达式的类型。我们使用逗号操作符,因此对第一个参数求值,但随后丢弃,因此decltype将从void()进行类型推导,并且推导出的类型为void。然而,参数typename T::foo_type()和typename T::bar_type()确实使用了内部类型,而且这只存在于foo或bar中。这就是使用SFINAE的地方,如下所示:

\begin{cppcode}
int_foo fi;
int_bar bi;
int x = 0;
handle(fi); // OK
handle(bi); // OK
handle(x); // error
\end{cppcode}

调用带有int_foo值的句柄将匹配第一个重载,而第二个因为替换失败而丢弃。类似地,调用带有int_bar值的句柄将匹配第二个重载,而第一个因为替换失败而丢弃。然而,使用int类型调用句柄将导致两个重载的替换失败,因此用于替换int类型的最终重载集将为空,从而调用没有匹配项,所以会产生编译错误。

SFINAE并不是实现条件编译的最佳方式。现代C++中,最好能与enable_if的类型特征一起使用。这就是我们接下来要讨论的问题。






















