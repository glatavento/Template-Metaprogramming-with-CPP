\section{使用enable\_if类型特性启用SFINAE}
C++标准库中,类型库是其子库之一。这个库定义了std::size\_t、std::nullptr\_t和std::byte等类型,运行时类型标识支持std::type\_info等类型,以及类型特征的集合。类型特征有两类:

\begin{itemize}
\item
类型特征能够在编译时查询类型的属性。

\item
能够在编译时执行类型转换的类型特征(添加或删除const限定符,或从类型中添加或删除指针或引用),这些类型特征也称为元功能。
\end{itemize}

来自第二类的一个类型特征是std::enable\_if,这用于启用SFINAE并从函数的重载集中删除候选。一个可能的实现如下所示:

\begin{cpp}
template<bool B, typename T = void>
struct enable_if {};

template<typename T>
struct enable_if<true, T> { using type = T; };
\end{cpp}

有一个主模板,有两个模板参数,一个布尔型非类型模板和一个默认参数为void的类型参数。主模板是一个空类,非类型模板参数值也有偏特化。这定义了一个称为type的成员类型,它是模板参数T的别名模板。

enable\_if元函数用于布尔表达式,这个布尔表达式求值为true时,定义了一个名为type的成员类型。若布尔表达式为false,则不定义此成员类型。来看看它是如何工作的。

还记得本章开头理解和定义类型特征一节中的例子吗?例子中,有一些类提供了一个写入方法,将内容写入输出流,以及为了同样的目的重载了操作符<{}<的类。在那一节中,我们定义了一个名为uses\_write的类型特征,并编写了一个serialize函数模板,该模板允许以统一的方式序列化这两种类型的对象(widget和gadget)。然而,实现相当复杂。使用enable\_if,可以以一种简单的方式实现该函数。一个可能的实现如下所示:

\begin{cpp}
template <typename T,
		  typename std::enable_if<
			uses_write_v<T>>::type* = nullptr>
void serialize(std::ostream& os, T const& value)
{
	value.write(os);
}

template <typename T,
		  typename std::enable_if<
			!uses_write_v<T>>::type*=nullptr>
void serialize(std::ostream& os, T const& value)
{
	os << value;
}
\end{cpp}

这个实现中有两个重载函数模板,有两个模板参数。第一个参数是类型模板参数,称为T。第二个参数是指针类型的匿名非类型模板参数,其默认值是nullptr。只有当uses\_write\_v变量的值为true时,才使用enable\_if定义成为type的成员。因此,对于具有成员函数write的类,第一次重载替换成功,但第二次重载替换失败,因为typename * = nullptr不是有效参数。对于操作符<{}<重载的类,情况则相反。

enable\_if元函数可以在以下几种情况下使用:

\begin{itemize}
\item
定义具有默认参数的模板参数

\item
定义具有默认参数的函数参数

\item
指定函数的返回类型
\end{itemize}

出于这个原因,所以在前面提供serialize重载实现只是一种可能。类似的使用enable\_if来定义一个带有默认参数的函数参数,如下所示:

\begin{cpp}
template <typename T>
void serialize(
	std::ostream& os, T const& value,
	typename std::enable_if<
				uses_write_v<T>>::type* = nullptr)
{
	value.write(os);
}

template <typename T>
void serialize(
	std::ostream& os, T const& value,
	typename std::enable_if<
				!uses_write_v<T>>::type* = nullptr)
{
	os << value;
}
\end{cpp}

这里,我们把参数从模板参数列表移到了函数参数列表。没有其他变化,用法相同,如下所示:

\begin{cpp}
widget w{ 1, "one" };
gadget g{ 2, "two" };

serialize(std::cout, w);
serialize(std::cout, g);
\end{cpp}

第三种选择是使用enable\_if来包装函数的返回类型,实现略有不同(默认参数对于返回类型没有意义):

\begin{cpp}
template <typename T>
typename std::enable_if<uses_write_v<T>>::type serialize(
	std::ostream& os, T const& value)
{
	value.write(os);
}

template <typename T>
typename std::enable_if<!uses_write_v<T>>::type serialize(
	std::ostream& os, T const& value)
{
	os << value;
}
\end{cpp}

这个实现中,若uses\_write\_v<T>为true,则定义返回类型。否则,替换失败,SFINAE。

尽管在所有这些示例中,enable\_if类型特征都用于在函数模板的重载解析期间启用SFINAE,但此类型特征也可用于限制类模板的实例化。下面的例子中,有一个叫做integral\_wrapper的类,只进行了整型类型实例化,还有一个叫做float\_wrapper的类,只被浮点型类型实例化:

\begin{cpp}
template <
	typename T,
	typename=typenamestd::enable_if_t<
						 std::is_integral_v<T>>>
struct integral_wrapper
{
	T value;
};

template <
	typename T,
	typename=typename std::enable_if_t<
						 std::is_floating_point_v<T>>>
struct floating_wrapper
{
	T value;
};
\end{cpp}

这两个类模板都有两个类型模板参数。第一个称为T,但第二个匿名,有一个默认参数。根据布尔表达式的值,这个参数的值是否在enable\_if类型特征的帮助下定义。

这个实现中,可以看到:

\begin{itemize}
\item
别名模板std::enable\_if\_t,这是访问std::enable\_if<B, T>::type成员类型的方法。其定义如下:

\begin{cpp}
template <bool B, typename T = void>
using enable_if_t = typename enable_if<B,T>::type;
\end{cpp}

\item
两个变量模板std::is\_integral\_v和std::is\_floating\_point\_v是访问数据成员的方法,std::is\_integral<T>::value和std::is\_floating\_point<T>::value。std::is\_integral和std::is\_float\_point类是标准类型特征,分别检查类型是整型还是浮点型。
\end{itemize}

前面展示的两个wrapper类模板可以按如下方式使用:

\begin{cpp}
integral_wrapper w1{ 42 }; // OK
integral_wrapper w2{ 42.0 }; // error
integral_wrapper w3{ "42" }; // error

floating_wrapper w4{ 42 }; // error
floating_wrapper w5{ 42.0 }; // OK
floating_wrapper w6{ "42" }; // error
\end{cpp}

其中只有两个实例化可以工作:w1,integral\_wrapper是用int类型实例化的;w5,float\_wrapper是用double类型实例化的。所有其他选项都会让编译器报错。

需要指出的是,此代码示例仅适用于C++20中提供的integral\_wrapper和float\_wrapper的定义。对于标准的以前版本,因为编译器无法推导模板参数,即使是w1和w5的定义也会产生编译器错误。为了使其工作,必须更改类模板以包括一个构造函数,如下所示:

\begin{cpp}
template <
	typename T,
	typename=typenamestd::enable_if_t<
						 std::is_integral_v<T>>>
struct integral_wrapper
{
	T value;
	
	integral_wrapper(T v) : value(v) {}
};

template <
	typename T,
	typename=typename std::enable_if_t<
						 std::is_floating_point_v<T>>>
struct floating_wrapper
{
	T value;
	
	floating_wrapper(T v) : value(v) {}
};
\end{cpp}

虽然enable\_if有助于通过更简单、更可读的代码实现SFINAE,但它仍然复杂。幸运的是,C++17中有一个更好的选择——constexpr if。接下来让我们来探索一下这个替代方案。


































