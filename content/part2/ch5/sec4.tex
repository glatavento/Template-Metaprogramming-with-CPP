\section{constexpr if}
C++17的constexpr if特性使得SFINAE更加简单,是if语句的编译时版本,有助于用更简单的版本替换复杂的模板代码。先看一下serialize函数的C++17实现,可以统一序列化widget和gadget:

\begin{cppcode}
template <typename T>
void serialize(std::ostream& os, T const& value)
{
	if constexpr (uses_write_v<T>)
		value.write(os);
	else
		os << value;
}
\end{cppcode}

其语法是if constexpr(condition),条件必须是编译时表达式,求值表达式不执行短路逻辑。若表达式具有形式a \&\& b或a || b,那么a和b都必须是定义良好的。

constexpr if能够在编译时根据表达式的值丢弃分支。例子中,当uses_write_v为true时,else分支将丢弃,而第一个分支的主体将保留,最终得到了widget和gadget类的特化:

\begin{cppcode}
template<>
void serialize<widget>(std::ostream & os,
                      widget const & value)
{
	if constexpr(true)
	{
		value.write(os);
	}
}

template<>
void serialize<gadget>(std::ostream & os,
                      gadget const & value)
{
	if constexpr(false)
	{
	}
	else
	{
		os << value;
	}
}
\end{cppcode}

当然,编译器可能会进一步简化这段代码,这些特化可能会像下面这样:

\begin{cppcode}
template<>
void serialize<widget>(std::ostream & os,
                       widget const & value)
{
	value.write(os);
}

template<>
void serialize<gadget>(std::ostream & os,
					   gadget const & value)
{
	os << value;
}
\end{cppcode}

最终结果与使用SFINAE和enable_if实现的结果相同,但这里编写的实际代码更简单,更容易理解。

constexpr if是一个很好的简化代码的工具,在第3章中看到过,当时实现了一个名为sum的函数:

\begin{cppcode}
template <typename T, typename... Args>
T sum(T a, Args... args)
{
	if constexpr (sizeof...(args) == 0)
		return a;
	else
		return a + sum(args...);
}
\end{cppcode}

本例中,constexpr if协助我们避免出现两个重载,一个用于一般情况,另一个用于结束递归。本书中已经介绍了另一个例子,这个例子中,constexpr if可以简化实现,是第4章中的阶乘函数模板。该函数的实现如下所示:

\begin{cppcode}
template <unsigned int n>
constexpr unsigned int factorial()
{
	return n * factorial<n - 1>();
}

template<>
constexpr unsigned int factorial<1>() { return 1; }

template<>
constexpr unsigned int factorial<0>() { return 1; }
\end{cppcode}

使用constexpr if,可以用一个模板替换所有这些,并让编译器负责提供正确的特化。C++17版本的函数实现如下所示:

\begin{cppcode}
template <unsigned int n>
constexpr unsigned int factorial()
{
	if constexpr (n > 1)
		return n * factorial<n - 1>();
	else
		return 1;
}
\end{cppcode}

constexpr if在很多情况下都很有用。本节给出的最后一个示例是一个名为are_equal的函数模板,其决定提供的两个参数是否相等。通常,会认为使用operator==就足以确定两个值是否相等。大多数情况下是正确的,除了浮点数。因为浮点数可以存储而没有精度损失(像1,1.25,1.5这样的数字,并且小数部分都可以表示为2的逆幂级数),所以在比较浮点数时需要特别注意。通常,这是通过确保两个浮点值之间的差值小于某个阈值来解决的,所以函数的实现可能如下所示:

\begin{cppcode}
template <typename T>
bool are_equal(T const& a, T const& b)
{
	if constexpr (std::is_floating_point_v<T>)
		return std::abs(a - b) < 0.001;
	else
		return a == b;
}
\end{cppcode}

当T类型是浮点类型时,将两个数字之差的绝对值与所选阈值进行比较。否则,退回到operator==。这使得不仅可以对算术类型使用此函数,还可以对其他重载了相等操作符的类型使用此函数。

\begin{cppcode}
are_equal(1, 1); // OK
are_equal(1.999998, 1.999997); // OK
are_equal(std::string{ "1" }, std::string{ "1" }); // OK
are_equal(widget{ 1, "one" }, widget{ 1, "two" }); // error
\end{cppcode}

可以使用参数类型为int、double和std::string调用are_equal函数模板,但尝试对widget类型的值执行相同的操作将使编译器报错,因为==操作符没有在此类型中进行重载。

本章中,我们已经了解了什么是类型特征,以及执行条件编译的不同方法。还看到了标准库中可用的一些类型特征。本章的第二部分,我们将探讨标准中关于类型特征的内容。






































