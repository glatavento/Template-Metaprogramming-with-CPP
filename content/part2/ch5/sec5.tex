\section{探索标准类型特征}

标准库提供了一系列类型特征,用于查询类型的属性以及对类型执行转换。这些类型特征可以在<type\_traits>头文件中作为类型支持库使用。类型特征有以下几种:

\begin{itemize}
  \item
        查询类型类别(主或复合类型)

  \item
        查询类型属性

  \item
        查询支持的操作

  \item
        查询类型关系

  \item
        修改cv说明符、引用、指针或符号

  \item
        各种转换
\end{itemize}

尽管研究每一种类型的特征超出了本书的范畴,但将探讨这些类别,看看它们包含什么。下面的小节中,将列出构成这些类别的类型特征(或大部分类型特征)。这些列表以及关于每种类型特征的详细信息可以在C++标准中找到(参见本章末尾的扩展阅读,以获得标准草案版本的链接)或cppreference.com网站\url{https://en.cppreference.com/w/cpp/header/type_traits}(许可链接:\url{http://creativecommons.org/licenses/by-sa/3.0/})。

\subsection{查询类型类别}

之前已经使用了几个类型特征,例如std::is\_integral,std::is\_floating\_point,以及std::is\_arithmetic。这些用于查询主类型和复合类型类别的一些标准类型特征。下表列出了所有这些类型的特征:

\begin{longtblr}
  { colspec = {|l|X|}, hlines, rowhead = 1, rows={m}, row{1} = {c, font=\bfseries}, measure = vbox }
  名称                            & 描述                       \\
  is\_void                      &
  类型是否为void类型。                                             \\
  is\_null\_pointer             &
  一个类型是否为std::nullptr\_t类型。                                \\
  is\_integral                  &
  {类型是否为整型,包括有符号、无符号和cv限定变量。整型类型为:
      \begin{itemize}[tableitem]
        \item bool, char, char8\_t(C++20), char16\_t, char32\_t, wchar\_t, short, int, long和long long
        \item 任何扩展整数类型
      \end{itemize}}
  \\
  is\_floating\_point           &
  类型是否为浮点类型,包括cv限定变量。T可能的类型有 float,double和long double.     \\
  is\_array                     &
  类型是否为数组类型。                                               \\
  is\_enum                      &
  类型是否为枚举类型。                                               \\
  is\_union                     &
  类型是否为联合类型。                                               \\
  is\_class                     &
  类型是否是类类型,而不是联合类型。                                        \\
  is\_function                  &
  类型是否为函数类型。除了Lambda,重载调用操作符的类,指向函数的指针,而std::function类型除外。 \\
  is\_pointer                   &
  类型是指向对象的指针、指向函数的指针还是cv限定变量。这不包括指向成员对象的指针或指向成员函数的指针。      \\
  is\_member\_pointer           &
  类型是指向非静态成员对象的指针,还是指向非静态成员函数的指针。                          \\
  is\_member\_object\_pointer   &
  类型是否为非静态成员对象指针。                                          \\
  is\_member\_function\_pointer &
  类型是否为非静态成员函数指针。                                          \\
  is\_lvalue\_reference         &
  类型是否为左值引用类型。                                             \\
  is\_rvalue\_reference         &
  类型是否为右值引用类型。                                             \\
  is\_reference                 &
  类型是否为引用类型,可以是左值引用类型,也可以是右值引用类型。                          \\
  is\_fundamental               &
  类型是否为基本类型。基本类型为算术类型, void类型和std::nullptr\_t类型。           \\
  is\_scalar                    &
  {类型是标量类型还是标量类型的cv-限定符版本。 标量类型包括:
      \begin{itemize}[tableitem]
        \item 算术类型
        \item 指针类型
        \item 指向成员类型的指针
        \item 枚举类型
        \item std::nullptr\_t
      \end{itemize}}
  \\
  is\_object                    &
  类型是否为cv-qualifier版本的对象类型。对象类型不是函数类型、引用类型或void类型。         \\
  is\_compound                  &
  {类型是复合类型还是复合类型的cv变体。复合类型不是基本类型,它们是
      \begin{itemize}[tableitem]
        \item 数组
        \item 函数
        \item 类
        \item 联合
        \item 对象指针和函数指针
        \item 成员对象指针和成员函数指针
        \item 引用
        \item 枚举
      \end{itemize}}
  \\
\end{longtblr}


这些类型特征在C++11中可用。从C++17开始,每个变量都有一个变量模板来简化对布尔成员value的访问。对于名称为is\_abc的类型特征,则存在名为is\_abc\_v的变量模板。对于所有具有名为value的布尔成员的类型特征都是如此,这些变量的定义很简单。下面的代码展示了is\_arithmentc\_v变量模板的定义:

\begin{cpp}
template< class T >
inline constexpr bool is_arithmetic_v =
	is_arithmetic<T>::value;
\end{cpp}

下面是使用这些类型特征的例子:

\begin{cpp}
template <typename T>
std::string as_string(T value)
{
	if constexpr (std::is_null_pointer_v<T>)
		return "null";
	else if constexpr (std::is_arithmetic_v<T>)
		return std::to_string(value);
	else
		static_assert(always_false<T>);
}

std::cout << as_string(nullptr) << '\n'; // prints null
std::cout << as_string(true) << '\n'; // prints 1
std::cout << as_string('a') << '\n'; // prints a
std::cout << as_string(42) << '\n'; // prints 42
std::cout << as_string(42.0) << '\n'; // prints 42.000000
std::cout << as_string("42") << '\n'; // error
\end{cpp}

函数模板as\_string返回一个包含pass值作为参数的字符串。它只适用于算术类型,并且适用于nullptr\_t,将为其返回值“null”。

聪明的读者一定注意到了static\_assert(always\_false<T>),并想知道这个always\_false<T>表达式到底是什么。其是一个bool类型的变量模板,计算结果为false。其定义简单如下:

\begin{cpp}
template<class T>
constexpr bool always_false = std::false_type::value;
\end{cpp}

static\_assert(false)会使程序格式不正确。原因是它的条件不依赖于模板参数,而是求值为false。若模板中不能为constexpr if语句的子语句生成有效的特化,则程序是格式错误的(不需要诊断)。为了避免这种情况,static\_assert的条件必须依赖于模板参数。对于static\_assert(always\_false<T>),编译器在模板实例化之前,不知道这将计算为true还是false。

我们探索的下一类类型特征,是查询类型的属性。

\subsection{查询类型属性}

能够查询类型属性的类型特征如下所示:

\begin{longtblr}
  { colspec = {|l|c|X|}, hlines, rowhead = 1, rows={m}, row{1} = {c, font=\bfseries}, measure = vbox }
  名称                      & C++ 版本 & 描述                                           \\
  is\_const               & C++11  & 是否为const限定(const或const volatile)。            \\
  is\_volatile            & C++11  & 是否为volatile限定(volatile或const volatile)。      \\
  is\_trivial             & C++11  &
  {是普通类型,还是cv限定变量。以下是简单类型:
      \begin{itemize}[tableitem]
        \item 标量类型或标量类型的数组
        \item 具有简单的默认构造函数或此类数组的简单的可复制类。
      \end{itemize}}
  \\
  is\_trivially\_copyable & C++11  &
  {是否可复制。 以下是可复制的类型:
      \begin{itemize}[tableitem]
        \item 标量类型或标量类型的数组
        \item 可复制的类或此类的数组
      \end{itemize}}
  \\
  is\_standard\_layout    & C++11  &
  {是标准布局类型,还是cv限定类型。以下是标准布局类型:
      \begin{itemize}[tableitem]
        \item 标量类型或标量类型的数组
        \item 标准布局类或此类的数组
      \end{itemize}}
  \\
  is\_empty               & C++11  & 是否为空类型。空类型是一种类类型(不是联合体),没有虚函数,没有虚基类,也没有非空基类。
  \\
  is\_polymorphic         & C++11  & 是否为多态类型。多态类型至少继承自有一个虚函数的类型(不是联合体)。
  \\
  is\_abstract            & C++11  & 是否为抽象类型。抽象类型至少继承自有一个虚纯函数的类型(不是联合体)。
  \\
  is\_final               & C++14  & 是否是使用final说明符声明的类型。                          \\
  is\_aggregate           & C++17  & 是否为聚合类型。
  \\
  is\_signed              & C++11  & 是浮点类型,还是有符号整型。                               \\
  is\_unsigned            & C++11  & 类型是无符号整型,还是bool类型。                           \\
  is\_bounded\_array      & C++20  & 是否为已知边界的数组类型(比如int{[}5{]})                   \\
  is\_unbounded\_array    & C++20  & 是否为未知边界的数组类型(比如int{[}{]}).                   \\
  is\_scoped\_enum        & C++23  & 是否为范围枚举类型。                                   \\
  {has\_unique\_object                                                            \\\_representation} &C++17 &是否可复制,并且该类型的两个具有相同值的对象也具有相同表现的形式。
  \\
\end{longtblr}

其中大多数可能很容易理解,但有两个乍一看似乎是相同的,is\_trivial和\_trivial\_copyable。对于标量类型或标量类型数组,两者都成立,也适用于可复制的类或此类数组,但is\_trivial仅适用于具有普通默认构造函数的可复制类。

根据C++20标准§11.4.4.1,若默认构造函数不是用户提供的,那么默认构造函数是普通的,类没有虚成员函数,没有虚基类,没有具有默认初始化式的非静态成员,其每个直接基类都有一个普通的默认构造函数,类的每个非静态成员也都有一个普通的默认构造函数。为了更好地理解这一点,来看看下面的例子:

\begin{cpp}
struct foo
{
	int a;
};

struct bar
{
	int a = 0;
};

struct tar
{
	int a = 0;
	tar() : a(0) {}
};

std::cout << std::is_trivial_v<foo> << '\n'; // true
std::cout << std::is_trivial_v<bar> << '\n'; // false
std::cout << std::is_trivial_v<tar> << '\n'; // false

std::cout << std::is_trivially_copyable_v<foo>
          << '\n'; // true
std::cout << std::is_trivially_copyable_v<bar>
          << '\n'; // true
std::cout << std::is_trivially_copyable_v<tar>
          << '\n'; // true
\end{cpp}

本例中,有三个类似的类。这三个变量foo、bar和tar都可复制,只有foo类是一个普通类,因为它有一个普通的默认构造函数。bar类有一个带有默认初始化式的非静态成员,tar类有一个用户定义的构造函数,这使得它们不普通。

除了复制能力之外,还可以在其他类型特征的帮助下查询支持的操作。

\subsection{查询支持的操作}

下面的类型特征可以查询类型支持的操作:

\begin{longtblr}
  { colspec = {|l|X|}, hlines, rowhead = 1, rows={m}, row{1} = {c, font=\bfseries} }
  名称 & 描述                     \\
  {is\_constructible          \\ is\_trivially\_constructible\\ is\_nothrow\_constructible}
     & 是否有可以接受特定参数的构造函数。      \\
  {is\_default\_constructible \\ is\_trivially\_default\_constructible\\ is\_nothrow\_default\_constructible}
     & 是否有默认构造函数。             \\
  {is\_copy\_constructible    \\ is\_tribially\_copy\_constructible\\ is\_nothrow\_copy\_constructible}
     & 是否具有复制构造函数             \\
  {is\_move\_constructible    \\ is\_trivially\_move\_constructible\\ is\_nothrow\_move\_constructible}
     & 是否具有移动构造函数。            \\
  {is\_assignable             \\ is\_trivially\_assignable\\ is\_nothrow\_assignable}
     & 是否具有特定参数的赋值操作符。        \\
  {is\_copy\_assignable       \\ is\_trivially\_copy\_assignable\\ is\_nothrow\_copy\_assignable}
     & 是否为复制赋值运算符。            \\
  {is\_move\_assignable       \\ is\_trivially\_move\_assigneable\\ is\_nothrow\_move\_assignable}
     & 是否有移动赋值操作符。            \\
  {is\_destructible           \\ is\_trivially\_destructible\\ is\_nothrow\_destructible}
     & 是否具有析构函数。              \\
  has\_virtual\_destructor
     & 是否具有虚析构函数。             \\
  {is\_swappable\_with        \\ is\_swappable\\ is\_nothrow\_swappable\_with\\ is\_nothrow\_swappable}
     & 是否可以交换相同类型的对象或不同类型的对象。 \\
\end{longtblr}

除了最后一个是在C++17引入,其他都在C++11引入。每种类型特征都有多个版本,包括用于检查普通操作或使用noexcept说明符声明为无异常抛出操作的版本。

现在来看看类型特征,以及如何查询类型之间的关系。

\subsection{查询类型的关系}

这里,可以找到几个类型特征,可以查询类型之间的关系。这些类型特征如下所示:

\begin{longtblr}
  { colspec = {|l|c|X|}, hlines, rowhead = 1, rows={m}, row{1} = {c, font=\bfseries} }
  名称                     & C++ 版本 & 描述                   \\
  is\_same               & C++11  & 两个类型是否相同,包括可能的cv限定符。 \\
  is\_base\_of           & C++11  & 一个类型是否派生自另一个类型。      \\
  {is\_convertible                                       \\ is\_nothrow\_convertible}
                         &
  {C++11                                                 \\ C++14}
                         &
  一种类型是否可以转换为另一种类型。                                      \\
  {is\_invocable                                         \\
  is\_invocable\_r                                       \\
  is\_nothrow\_invocable                                 \\
  is\_nothrow\_invocable\_r}
                         & C++17  & 是否可以调用一个类型与指定参数类型。   \\
  is\_layout\_compatible & C++20  &
  检查两种类型是否具有兼容的布局。若两个类是相同类型  (忽略cv限定符),或者其公共初始序列包含所有非静态数据成员和位字段,或者是具有相同底层类型的枚举,则为布局兼容。
  \\
  {is\_pointer                                           \\
  \_inconvertible\_base\_of}
                         & C++20  & 是否为另一类型的指针不可转换基类。    \\
\end{longtblr}

这里,使用最多的可能是std::is\_same。这种类型特征在判断两种类型是否相同时非常有用,不过这种类型的特征需要考虑const和volatile限定符,所以int和int const不是同一类型。

可以使用这个类型特征来扩展前面所示的as\_string函数的实现,若用true或false参数调用,则会输出1或0,而不是true/false。可以为bool类型添加一个显式检查,并返回一个包含这两个值之一的字符串:

\begin{cpp}
template <typename T>
std::string as_string(T value)
{
	if constexpr (std::is_null_pointer_v<T>)
		return "null";
	else if constexpr (std::is_same_v<T, bool>)
		return value ? "true" : "false";
	else if constexpr (std::is_arithmetic_v<T>)
		return std::to_string(value);
	else
		static_assert(always_false<T>);
}

std::cout << as_string(true) << '\n'; // prints true
std::cout << as_string(false) << '\n'; // prints false
\end{cpp}

目前看到的所有类型特征,都用于查询关于类型的某种信息。下一节中,将看到对类型执行进行修改的类型特征。

\subsection{修改cv限定符、引用、指针或符号}

类型上执行转换的类型特征也称为元函数。这些类型特征提供了一个称为type的成员类型(typedef),表示转换后的类型。这些类型特征包括:


\begin{longtblr}
  { colspec = {|l|X|}, hlines, rowhead = 1, rows={m}, row{1} = {c, font=\bfseries} }
  名称 & 描述                            \\
  {add\_cv                           \\ add\_const\\ add\_volatile}
     & 将const、volatile或两者都添加到类型中。    \\
  {remove\_cv                        \\ remove\_const\\ remove\_volatile}
     & 从类型中删除const、volatile或同时删标识符。  \\
  {add\_lvalue\_reference            \\ add\_rvalue\_reference}
     & 向类型添加左值或右值引用。                 \\
  remove\_reference
     & 从类型中删除引用(左值或右值)。              \\
  remove\_cvref
     &
  {从类型中删除const和volatile限定符以及左值或右值引用。 \\ 融合了remove\_cv和remove\_reference。}
  \\
  add\_pointer
     & 添加指向类型的指针。                    \\
  remove\_pointer
     & 从类型中移除指针。                     \\
  {make\_signed                      \\ make\_unsigned}
     &
  {创建有符号或无符号的整型(bool类型除外)或枚举类型。支持的整数类型有short, int, long, long long, char, wchar\_t, char8\_t, char16\_t和 char32\_t。}
  \\
  {remove\_extent                    \\ remove\_all\_extents}
     & 从数组类型中移除一个范围或所有范围。            \\
\end{longtblr}

除了remove\_cvref是在C++20中添加的,本表中列出的所有其他类型特征在C++11中可用。这些并不是标准库中的所有元函数,更多的将在下一节中看到。

\subsection{各种转换}

除了前面列出的元函数之外,还有其他执行类型转换的类型特征。其中常用的如下表所示:

\begin{longtblr}
  { colspec = {|l|c|X|}, hlines, rowhead = 1, rows={m}, row{1} = {c, font=\bfseries} }
  名称                & C++ 版本 & 描述 \\
  enable\_if        &
  C++11             &
  启用从重载解析中删除函数重载或模板特化。            \\
  conditional       &
  C++11             &
  通过基于编译时布尔条件进行选择, 将type成员类型定义为两种可能类型之一。
  \\
  decay             &
  C++11             &
  在类型上应用转换(数组类型为数组到指针, 引用类型为左值到右值,函数类型为指针),删除const和volatile限定符,并使用结果成员类型定义类型作为其自己的成员类型定义类型。
  \\
  common\_type      &
  C++11             &
  从一组类型中确定公共类型。                   \\
  common\_reference &
  C++20             &
  从一组类型中确定公共引用类型。                 \\
  underlaying\_type &
  C++11             &
  确定枚举类型的基础类型。                    \\
  void\_t           &
  C++17             &
  将类型序列映射到void类型的类型别名。            \\
  type\_identity    &
  C++20             &
  提供成员typedef类型作为类型参数T的别名。        \\
\end{longtblr}

列表中,已经讨论了enable\_if,还有一些需要举例说明的类型特征。先来看看std::decay,考虑一下as\_string函数的另一种实现方式:

\begin{cpp}
template <typename T>
std::string as_string(T&& value)
{
	if constexpr (std::is_null_pointer_v<T>)
		return "null";
	else if constexpr (std::is_same_v<T, bool>)
		return value ? "true" : "false";
	else if constexpr (std::is_arithmetic_v<T>)
		return std::to_string(value);
	else
		static_assert(always_false<T>);
}
\end{cpp}

其中的变化是将参数传递给函数的方式。不是按值传递,而是按右值引用传递,这是一个转发引用。可以通过传递右值(例如字面量)进行调用,但传递左值会触发编译器错误:

\begin{cpp}
std::cout << as_string(true) << '\n'; // OK
std::cout << as_string(42) << '\n'; // OK

bool f = true;
std::cout << as_string(f) << '\n'; // error

int n = 42;
std::cout << as_string(n) << '\n'; // error
\end{cpp}

最后两个调用将导致static\_assert失败,实际的类型模板参数是bool\&和int\&。因此std::is\_same<bool, bool\&>将用false初始化值member,std::is\_arithmetic<int\&>也会做同样的事情。为了求值这些类型,需要忽略引用以及const和volatile限定符。这里使用的类型特征是std::decay,其会执行几个转换。其执行的概念如下所示:

\begin{cpp}
template <typename T>
struct decay
{
private:
	using U = typename std::remove_reference_t<T>;
public:
	using type = typename std::conditional_t<
		std::is_array_v<U>,
		typename std::remove_extent_t<U>*,
		typename std::conditional_t<
			std::is_function<U>::value,
			typename std::add_pointer_t<U>,
			typename std::remove_cv_t<U>
		>
	>;
};
\end{cpp}

可以看到std::decay是在其他元函数的帮助下实现的,包括std::conditional,是基于编译时表达式在一种类型或另一种类型之间进行选择的关键。这种类型特征会多次使用,若需要根据多个条件进行选择,那么可以这样做。

在std::decay的帮助下,可以修改as\_string函数、剥离引用和cv-限定符:

\begin{cpp}
template <typename T>
std::string as_string(T&& value)
{
	using value_type = std::decay_t<T>;

	if constexpr (std::is_null_pointer_v<value_type>)
		return "null";
	else if constexpr (std::is_same_v<value_type, bool>)
		return value ? "true" : "false";
	else if constexpr (std::is_arithmetic_v<value_type>)
		return std::to_string(value);
	else
		static_assert(always_false<T>);
}
\end{cpp}

通过修改实现,可使消除前面对as\_string的编译错误。

std::decay的实现中,多次使用了std::conditional。这是一个相当好用的元函数,可以简化许多实现。第2章中,我们看到了一个例子,构建了一个名为list\_t的列表类型。有一个名为type的成员别名模板,若列表的大小为1,则该模板类型为T,若列表的大小大于1,则该模板类型为std::vector<T>。再来回顾一下:

\begin{cpp}
template <typename T, size_t S>
struct list
{
	using type = std::vector<T>;
};

template <typename T>
struct list<T, 1>
{
	using type = T;
};

template <typename T, size_t S>
using list_t = typename list<T, S>::type;
\end{cpp}

在std::conditional的帮助下,这个实现可以进行简化:

\begin{cpp}
template <typename T, size_t S>
using list_t =
	typename std::conditional<S ==
					1, T, std::vector<T>>::type;
\end{cpp}

没有必要依赖类模板特化来定义这样的列表类型,整个解决方案可以简化为定义一个别名模板。可以用一些static\_assert来验证它是否如预期的那样工作:

\begin{cpp}
static_assert(std::is_same_v<list_t<int, 1>, int>);
static_assert(std::is_same_v<list_t<int, 2>,
							std::vector<int>>);
\end{cpp}

举例说明每个标准类型特征的使用超出了本书的范畴。本章的下一节提供了更复杂的例子,需要使用几个标准的类型特征。