\section{实际使用类型特征的例子}

本章的前一节中,已经探讨了标准库提供的各种类型特征。为每一种类型特征找到例子的确很困难,也没有必要。但有必要展示一些可以使用多种类型特征,来解决问题的示例。

\subsection{实现一个复制算法}

第一个示例问题是std::copy标准算法(<algorithm>)的可能实现。我们接下来将看到的不是实际的实现,而是一个可帮助我们更多地了解类型特征使用的实现。该算法的声明如下:

\begin{cpp}
template <typename InputIt, typename OutputIt>
constexpr OutputIt copy(InputIt first, InputIt last,
						OutputIt d_first);
\end{cpp}

这个函数只有在C++20中是constexpr,但是可以在这里进行讨论。其所做的是将范围[first, last)中的所有元素复制到另一个以d\_first开头的范围内。还有一个重载接受执行策略,还有一个版本std::copy\_if,复制与谓词匹配的所有元素,但这些对于我们的示例并不重要。这个函数的简单实现如下所示:

\begin{cpp}
template <typename InputIt, typename OutputIt>
constexpr OutputIt copy(InputIt first, InputIt last,
						OutputIt d_first)
{
	while (first != last)
	{
		*d_first++ = *first++;
	}
	return d_first;
}
\end{cpp}

但在某些情况下,这种实现可以通过简单地复制内存来优化。为了达到这个目的,必须满足一些条件:

\begin{itemize}
  \item 两种迭代器类型InputIt和OutputIt都必须是指针。
  \item 两个模板参数InputIt和OutputIt必须指向相同的类型(忽略cv-限定符)。
  \item InputIt所指向的类型,具有普通的复制赋值操作符。
\end{itemize}

可以用以下标准类型特征来检查这些条件:

\begin{itemize}
  \item std::is\_same(和std::is\_same\_v变量)来检查两种类型是否相同。
  \item std::is\_pointer(和std::is\_pointer\_v变量)来检查一个类型是否是指针类型。
  \item std::is\_trivially\_copy\_assignable(和std::is\_trivially\_copy\_assignable\_v变量)检查类型是否具有普通的复制赋值操作符。
  \item std::remove\_cv(和std::remove\_cv\_t别名模板)从类型中删除cv-限定符。
\end{itemize}

来看看如何实现。首先,需要有一个带有泛型实现的主模板,然后有一个带有优化实现的指针类型特化。可以使用带有成员函数模板的类模板来实现,如下所示:

\begin{cpp}
namespace detail
{
	template <bool b>
	struct copy_fn
	{
		template<typename InputIt, typename OutputIt>
		constexpr static OutputIt copy(InputIt first,
										InputIt last,
										OutputIt d_first)
		{
			while (first != last)
			{
				*d_first++ = *first++;
			}
			return d_first;
		}
	};

	template <>
	struct copy_fn<true>
	{
		template<typename InputIt, typename OutputIt>
		constexpr static OutputIt* copy(
			InputIt* first, InputIt* last,
			OutputIt* d_first)
		{
			std::memmove(d_first, first,
						(last - first) * sizeof(InputIt));
			return d_first + (last - first);
		}
	};
}
\end{cpp}

为了在源和目标之间复制内存,这里使用std::memmove,即使对象重叠,也会复制数据。这些实现是在一个名为detail的命名空间中,由复制函数轮流使用进行实现的细节,而不是直接由用户使用。该通用复制算法的实现方式如下:

\begin{cpp}
template<typename InputIt, typename OutputIt>
constexpr OutputIt copy(InputIt first, InputIt last,
OutputIt d_first)
{
	using input_type = std::remove_cv_t<
		typename std::iterator_traits<InputIt>::value_type>;
	using output_type = std::remove_cv_t<
		typename std::iterator_traits<OutputIt>::value_type>;
		
	constexpr bool opt =
		std::is_same_v<input_type, output_type> &&
		std::is_pointer_v<InputIt> &&
		std::is_pointer_v<OutputIt> &&
		std::is_trivially_copy_assignable_v<input_type>;
		
	return detail::copy_fn<opt>::copy(first, last, d_first);
}
\end{cpp}

可以看到选择哪个特化的决定,是基于使用前面提到的类型特征确定的constexpr布尔值。下面的代码段显示了使用这个复制函数的例子:

\begin{cpp}
std::vector<int> v1{ 1, 2, 3, 4, 5 };
std::vector<int> v2(5);

// calls the generic implementation
copy(std::begin(v1), std::end(v1), std::begin(v2));

int a1[5] = { 1,2,3,4,5 };
int a2[5];

// calls the optimized implementation
copy(a1, a1 + 5, a2);
\end{cpp}

这不是在标准库实现中找到的泛型算法副本的真正定义,标准库实现是进一步优化的。但这是一个很好的示例,可以演示如何将类型特征用于实际问题。

简单起见,我在全局命名空间中定义了复制函数。这是一个不好的做法。一般来说,代码(尤其是库中的代码)是按命名空间分组的。在本书附带的GitHub源代码中,会发现这个函数定义在一个名为n520的命名空间中(这只是一个名称,与主题无关)。当调用我们定义的复制函数时,实际上需要使用完全限定名(包括命名空间的名称),如下所示:

\begin{cpp}
n520::copy(std::begin(v1), std::end(v1), std::begin(v2));
\end{cpp}

若没有这个条件,参数依赖查找(ADL)就会开始运作。因为传递的参数可以在std命名空间中找到,所以这将调用std::copy函数。可以在\url{https://en.cppreference.com/w/cpp/language/adl}上阅读更多关于ADL的信息。

接下来,看看下一个例子。

\subsection{构建一个同构可变参的函数模板}

对于第二个示例,我们希望构建一个可变参数函数模板,该模板只能接受相同类型的参数,或者可以隐式转换为公共类型的参数。从下面的框架定义开始:

\begin{cpp}
template<typename... Ts>
void process(Ts&&... ts) {}
\end{cpp}

这样做的问题是,下面所有的函数调用都可以工作(这个函数的函数体是空的,不会因为执行某些类型上不可用的操作而出现错误):

\begin{cpp}
process(1, 2, 3);
process(1, 2.0, '3');
process(1, 2.0, "3");
\end{cpp}

第一个例子中,传递了三个int值。第二个例子中,传递了一个int型、一个double型和一个char型的值;int和char都可以隐式转换为double类型,所以这是正确的。然而,第三个例子中,传递了一个int型、一个double型和一个char型const*的值,最后一个类型不能隐式转换为int型或double型。所以,最后一个调用应该会触发编译器错误,但实际上并没有。

为此,需要确保当函数参数的公共类型不可用时,编译器将生成一个错误。可以使用static\_assert或std::enable\_if和SFINAE,但确实需要弄清楚是否存在一种常见类型。这在std::common\_type类型特征的帮助下,可以进行判断。

std::common\_type是一个元函数,它在所有类型参数中定义公共类型,所有类型都可以隐式转换为该类型,所以std::common\_type<int, double, char>::type将为double类型的别名。使用这个类型特征,可以构建另一个类型特征,从而说明共同类型是否存在。一个可能的实现如下所示:

\begin{cpp}
template <typename, typename... Ts>
struct has_common_type : std::false_type {};
template <typename... Ts>
struct has_common_type<
		std::void_t<std::common_type_t<Ts...>>,
		Ts...>
	: std::true_type {};
	
template <typename... Ts>
constexpr bool has_common_type_v =
	sizeof...(Ts) < 2 ||
	has_common_type<void, Ts...>::value;
\end{cpp}

可以看到我们将实现基于其他几个类型特征。有std::false\_type和std::true\_type,分别是std::bool\_constant<false>和std::bool\_constant<true>的类型别名。std::bool\_constant类在C++17中可用,反之,它是bool类型的std::integral\_constant类特化的别名模板。最后一个类模板包装了一个指定类型的静态常量,其的概念实现如下所示(也提供了一些操作):

\begin{cpp}
template<class T, T v>
struct integral_constant
{
	static constexpr T value = v;
	using value_type = T;
};
\end{cpp}

这有助于简化需要定义布尔编译时值的类型特征的定义。

has\_common\_type类的实现中使用的第三种类型的特征是std::void\_t,此类型特征定义了可变数量的类型与void类型之间的映射。可以使用它来构建公共类型(若存在的话)和void类型之间的映射。可使我们能够利用SFINAE对has\_common\_type类模板进行特化。

最后,定义了名为has\_common\_type\_v的变量模板,以简化has\_common\_type特征的使用。

所有这些都可以用来修改流程函数模板的定义,以确保它只允许公共类型的参数。一个可能的实现如下所示:

\begin{cpp}
template<typename... Ts,
		 typename = std::enable_if_t<
						has_common_type_v<Ts...>>>
void process(Ts&&... ts)
{ }
\end{cpp}

因此,process(1, 2.0, "3")这样的调用将产生编译器错误,因为对于这组参数没有重载的process函数。

如前所述,有不同的方法来使用has\_common\_type特征来实现定义的目标。其中之一是使用std::enable\_if,也可以使用static\_assert,而使用概念应该是更好的方式,我们将在下一章中看到相关的内容。








