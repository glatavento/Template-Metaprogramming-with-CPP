\section{定义类型特征}
简而言之,类型特征是包含常量值的小型类模板,该常量值就是所询问的关于类型问题的答案。此类问题的例子是:该类型是浮点类型吗?构建提供此类类型信息的类型特征依赖于模板特化:定义一个主模板,以及一个或多个特化。

来看看如何构建类型特征,在编译时确定类型是否为浮点类型:

\begin{cpp}
template <typename T>
struct is_floating_point
{
	static const bool value = false;
};

template <>
struct is_floating_point<float>
{
	static const bool value = true;
};

template <>
struct is_floating_point<double>
{
	static const bool value = true;
};

template <>
struct is_floating_point<long double>
{
	static const bool value = true;
};
\end{cpp}

有两件事需要注意:

\begin{itemize}
\item
定义了一个主模板,以及几个完整的特化,每个类型对应一个浮点类型。

\item
主模板有一个用false值初始化的静态const布尔成员,全特化将该成员的值设置为true。
\end{itemize}

is\_float\_point<T>是一种类型特征,告诉我们一种类型是否是浮点类型。可以这样使用:

\begin{cpp}
int main()
{
	static_assert(is_floating_point<float>::value);
	static_assert(is_floating_point<double>::value);
	static_assert(is_floating_point<long double>::value);
	static_assert(!is_floating_point<int>::value);
	static_assert(!is_floating_point<bool>::value);
}
\end{cpp}

这证明我们构建了正确的类型特征,但这并不是一个实际的用例场景。为了使这个类型特征真正有用,需要在编译时使用它来处理信息。

假设要构建一个函数,需要对浮点值做一些事情。有多种浮点类型,例如:float、double和long double。为了避免编写多个实现,将其构建为模板函数。可以将其他类型作为模板参数传递,所以需要一种方法来避免这种情况。一个简单的解决方案是使用前面看到的static\_assert(),若用户提供的值不是浮点数,就会报错:

\begin{cpp}
template <typename T>
void process_real_number(T const value)
{
	static_assert(is_floating_point<T>::value);
	std::cout << "processing a real number: " << value
			  << '\n';
}
int main()
{
	process_real_number(42.0);
	process_real_number(42); // error:
	// static assertion failed
}
\end{cpp}

这是一个非常简单的示例,演示了使用类型特征进行条件编译。除了使用static\_assert(),还有其他方法。现在,来看看第二个例子。

假设有定义写入输出流的操作的类,这基本上是序列化的一种形式。但有些方法使用重载操作符<{}<进行支持,其他方法则使用名为write的成员函数:

\begin{cpp}
struct widget
{
	int id;
	std::string name;
	
	std::ostream& write(std::ostream& os) const
	{
		os << id << ',' << name << '\n';
		return os;
	}
};

struct gadget
{
	int id;
	std::string name;
	
	friend std::ostream& operator <<(std::ostream& os,
	                                 gadget const& o);
};

std::ostream& operator <<(std::ostream& os,
gadget const& o)
{
	os << o.id << ',' << o.name << '\n';
	return os;
}
\end{cpp}

本例中,widget类包含一个成员函数write。但对于gadget类,流操作符<{}<出于同样的目的进行重载。可以使用这些类编写以下代码:

\begin{cpp}
widget w{ 1, "one" };
w.write(std::cout);

gadget g{ 2, "two" };
std::cout << g;
\end{cpp}

然而,我们的目标是定义一个函数模板,并且能够以相同的方式对待它们。换句话说,不使用write或<{}<操作符:

\begin{cpp}
serialize(std::cout, w);
serialize(std::cout, g);
\end{cpp}

这就带来了一些问题。首先,这样的函数模板是什么样的?其次,如何知道类型是否提供了写入方法或<{}<操作符重载?第二个问题的答案是类型特征。可以构建一个类型特征,帮助我们在编译时回答后一个问题。所以需要这样的类型特征:

\begin{cpp}
template <typename T>
struct uses_write
{
	static constexpr bool value = false;
};

template <>
struct uses_write<widget>
{
	static constexpr bool value = true;
};
\end{cpp}

这与之前定义的类型特征非常相似,uses\_write说明一个类型是否定义了写成员函数。主模板将名为value的数据成员设置为false,但widget类的全特化将其设置为true。为了避免uses\_write<T>::value的冗长语法,还可以定义一个变量模板,将语法简化为uses\_write\_v<T>的形式。这个变量模板如下所示:

\begin{cpp}
template <typename T>
inline constexpr bool uses_write_v = uses_write<T>::value;
\end{cpp}

为了使例子更简单,假设不提供write成员函数的类型会重载输出流操作符。实践中,情况并非如此,为了简单起见,我们将基于此假设进行构建。

定义为序列化所有类提供统一API的函数模板serialize的下一步是定义更多的类模板,它们将遵循相同的方式——提供一种形式的序列化的主模板和提供另一种形式的全特化:

\begin{cpp}
template <bool>
struct serializer
{
	template <typename T>
	static void serialize(std::ostream& os, T const& value)
	{
		os << value;
	}
};

template<>
struct serializer<true>
{
	template <typename T>
	static void serialize(std::ostream& os, T const& value)
	{
		value.write(os);
	}
};
\end{cpp}

serializer类模板有一个模板参数,是非类型模板参数,也是一个匿名模板参数(不会在实现中使用)。这个类模板包含一个成员函数,实际上是具有单一类型模板参数的成员函数模板,这个参数定义了要序列化的值的类型,主模板使用<{}<操作符将值输出到提供的流。另一方面,serializer类模板的全特化使用成员函数write来完成同样的工作。这里,我们全特化了序列化器类模板,而不是serialize成员函数模板。

现在只剩下实现所需的独立函数serialize了,其实现将基于serializer<T>::serialize函数:

\begin{cpp}
template <typename T>
void serialize(std::ostream& os, T const& value)
{
	serializer<uses_write_v<T>>::serialize(os, value);
}
\end{cpp}

函数模板的签名与serializer类模板中的serialize成员函数的签名相同,主模板和全特化之间的选择可使用变量模板uses\_write\_v完成,其提供了一种方便的方式来访问uses\_write类型特征的数据成员。

这些示例中,我们已经了解了如何实现类型特征,并使用它们在编译时提供的信息来对类型施加限制,或者在一个或另一个实现之间进行选择。类似的目的还有另一种称为SFINAE的元编程技术,我们将在下面的小节中对其进行介绍。


























