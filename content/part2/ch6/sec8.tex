\section{约束变量模板和模板别名}
C++中,除了函数模板和类模板外,还有变量模板和别名模板,这些都需要定义约束。目前讨论的约束模板参数的相同规则也适用于这两个参数。

定义PI常量以显示变量模板的工作方式是一个典型的例子,这是一个简单的定义:

\begin{cpp}
template <typename T>
constexpr T PI = T(3.1415926535897932385L);
\end{cpp}

然而,这只对浮点类型有意义(也可能是其他类型,比如十进制,在C++中还不存在),所以这个定义应该限制为浮点类型:

\begin{cpp}
template <std::floating_point T>
constexpr T PI = T(3.1415926535897932385L);

std::cout << PI<double> << '\n'; // OK
std::cout << PI<int> << '\n'; // error
\end{cpp}

使用PI<double>是正确的,但PI<int>会产生编译器错误,可以以简单易读的方式进行约束。

最后一类模板——别名模板,也可以受到约束。下面的代码片段中,可以看到这样一个例子:

\begin{cpp}
template <std::integral T>
using integral_vector = std::vector<T>;
\end{cpp}

当T是整型时,integral\_vector模板是std::vector<T>的别名。使用以下替代声明也可以达到同样的效果,不过声明更长了:

\begin{cpp}
template <typename T>
requires std::integral<T>
using integral_vector = std::vector<T>;
\end{cpp}

可以像下面这样使用这个integral\_vector别名模板:

\begin{cpp}
integral_vector<int> v1 { 1,2,3 }; // OK
integral_vector<double> v2 {1.0, 2.0, 3.0}; // error
\end{cpp}

定义v1对象没问题,因为int是整型。不过,定义v2会产生编译器错误,因为double不是整型。

若注意了本节中的例子,会注意到其没有使用我们之前在本章中使用的类型特征(以及相关的变量模板),而是使用了几个概念:std::integral和std::float\_point。这些定义在<concepts>头文件中,可避免使用基于C++11(或更新的)类型特征重复定义相同的概念。后面,我们将简要介绍标准概念库的内容。在此之前,先来看看在C++20中还可以使用哪些方法来定义约束。































































