\section{约束类模板}
类模板和类模板特化也可以像函数模板一样受到约束。首先,考虑wrapper类模板,但这一次要求它只适用于整型的模板参数。这可以在C++20中简单地进行指定:

\begin{cpp}
template <std::integral T>
struct wrapper
{
	T value;
};

wrapper<int> a{ 42 }; // OK
wrapper<double> b{ 42.0 }; // error
\end{cpp}

为int实例化模板是可以的,但对double无效,因为其不是整型。

同样用requires子句和类模板特化指定的需求也可以约束。为了展示这一点,考虑这样一个场景:希望特化包装器类模板,但只针对大小为4字节的类型。可以这样实现:

\begin{cpp}
template <std::integral T>
struct wrapper
{
	T value;
};

template <std::integral T>
requires (sizeof(T) == 4)
struct wrapper<T>
{
	union
	{
		T value;
		struct
		{
			uint8_t byte4;
			uint8_t byte3;
			uint8_t byte2;
			uint8_t byte1;
		};
	};
};
\end{cpp}

如何使用这个类模板,如下所示:

\begin{cpp}
wrapper<short> a{ 42 };
std::cout << a.value << '\n';

wrapper<int> b{ 0x11223344 };
std::cout << std::hex << b.value << '\n';
std::cout << std::hex << (int)b.byte1 << '\n';
std::cout << std::hex << (int)b.byte2 << '\n';
std::cout << std::hex << (int)b.byte3 << '\n';
std::cout << std::hex << (int)b.byte4 << '\n';
\end{cpp}

对象a是wrapper<short>;,所以使用主模板。另一方面,对象b是wrapper<int>的实例。由于int的大小为4字节(在大多数平台上),因此使用了特化,我们可以通过byte1、byte2、byte3和byte4成员访问包装值的各个类型。

最后,我们将讨论如何约束变量模板和模板别名。




































































