\section{探索标准概念库}
标准库提供了一组基本概念,可用于定义对函数模板、类模板、变量模板和别名模板的模板实参的需求,正如在本章中看到的那样。C++20中的标准概念分布在多个头文件和命名空间中。我们将在本节中介绍其中一些,可以在\url{https://en.cppreference.com/}上找到所有的概念。

主要的概念集可以在<concepts>头文件和std命名空间中找到。这些概念中的大多数等价于一个或多个现有的类型特征。对于其中一些,它们的实现是定义良好的;另一些,则是不明确的。可分为四类:核心语言概念、比较概念、对象概念和可调用概念。这组概念包含以下内容(但不仅如此):

\begin{longtblr}
  { colspec = {|l|X|}, hlines, rowhead = 1, rows={m}, row{1} = {c, font=\bfseries} }
  概念                      & 描述                               \\
  same\_as                & 类型T与另一个类型U相同。                    \\
  derived\_from           & 类型D从另一个类型B派生。                    \\
  convertible\_to         &
  类型T可以隐式转换为另一类型U。                                           \\
  common\_reference\_with &
  类型T和U具有共同引用类型。                                             \\
  common\_with            &
  类型T和U具有共同类型,这两种类型都可以转换为共同类型。                               \\
  integral                & 类型T是整型。                          \\
  signed\_integral        & 类型T是有符号整型。                       \\
  unsigned\_integral      & 类型T是无符号整型。                       \\
  floating\_point         & 类型T是浮点类型。                        \\
  assignable\_from        &
  U类型的表达式可以赋值给T类型的左值表达式。                                     \\
  swappable               &
  两个相同类型T的值可以交换。                                             \\
  swappable\_with         &
  类型T的值可以与类型U的值相匹配。                                          \\
  destructible            &
  可以安全地销毁类型T的值(析构函数不抛出异常)。                                   \\
  constructible\_from     &
  可以用给定的参数类型集构造类型为T的对象。                                      \\
  default\_initializable  &
  类型T的对象可以是默认可构造(值初始化T(),从空初始化列表T\{\} 直接列表初始化,或默认初始化,如T t;)。 \\
  move\_constructible     &
  可以用移动语义构造类型T的对象。                                           \\
  copy\_constructible     &
  类型T的对象可以复制构造和移动构造。                                         \\
  moveable                &
  可以移动和交换类型为T的对象。                                            \\
  copyable                &
  可以复制、移动和交换类型为T的对象。                                         \\
  regular                 &
  类型T需要同时满足semiregular和equality\_comparable两个概念。             \\
  semiregular             &
  可以复制、移动、交换和默认构造类型为T的对象。                                    \\
  equality\_comparable    &
  类型T的比较运算符==反映相等,当两个值相等时为true。 类似的,=!表示了不相等。                \\
  predicate               &
  可调用类型T是布尔谓词。                                               \\
\end{longtblr}

其中一些概念是使用类型特征定义的,一些是其他概念或概念与类型特征的组合,还有一些具有部分未指定的实现。下面是一些例子:

\begin{cpp}
template < class T >
concept integral = std::is_integral_v<T>;

template < class T >
concept signed_integral = std::integral<T> &&
                          std::is_signed_v<T>;

template <class T>
concept regular = std::semiregular<T> &&
                  std::equality_comparable<T>;
\end{cpp}

C++20还引入了一个新的基于概念的迭代器系统,并在<iterator>头文件中定义了一组概念。下表列出了其中一些概念:

\begin{longtblr}
  { colspec = {|l|X|}, hlines, rowhead = 1, rows={m}, row{1} = {c, font=\bfseries} }
  概念                       & 描述                            \\
  indirectly\_readable     &
  可以通过应用*operator读取类型值。                                    \\
  indirectly\_writable     &
  迭代器类型所引用的对象可以写入。                                         \\
  input\_iterator          &
  类型是输入迭代器(支持读取、前缀递增和后缀递增)。                                \\
  output\_iterator         &
  类型是输出迭代器(支持写入、前缀递增和后缀递增)。                                \\
  forward\_iterator        &
  作为input\_iterator的类型同时也是前向迭代器(支持相等比较和多次传递)               \\
  bidirectional\_iterator  &
  作为forward\_iterator的类型同时,也是双向迭代器(支持向后移动)                 \\
  randon\_access\_iterator &
  作为bidirectional\_iterator类型的同时,也是随机迭代器(支持在常数时间内下标和前进)。   \\
  contiguous\_iterator     &
  作为random\_access\_iterator类型同时,也是连续迭代器(元素存储在连续的内存位置)的要求。 \\
\end{longtblr}

下面是C++标准中random\_access\_iterator概念的定义:

\begin{cpp}
template<typename I>
concept random_access_iterator =
	std::bidirectional_iterator<I> &&
	std::derived_from</*ITER_CONCEPT*/<I>,
					  std::random_access_iterator_tag> &&
	std::totally_ordered<I> &&
	std::sized_sentinel_for<I, I> &&
	requires(I i,
			 const I j,
			 const std::iter_difference_t<I> n)
	{
		{ i += n } -> std::same_as<I&>;
		{ j + n } -> std::same_as<I>;
		{ n + j } -> std::same_as<I>;
		{ i -= n } -> std::same_as<I&>;
		{ j - n } -> std::same_as<I>;
		{ j[n] } -> std::same_as<std::iter_reference_t<I>>;
	};
\end{cpp}

这里使用了几个概念(其中一些没有在这里列出),以及一个requires表达式来确保某些表达式是格式良好的。

此外,<iterator>头文件中,有一组旨在简化通用算法约束的概念。下表列出了其中一些概念:

\begin{table}[!htb]
  \centering
  \begin{talltblr}
    { colspec = {|l|X|}, hlines, rows={m}, row{1} = {c, font=\bfseries} }
    概念                   & 描述                            \\
    indriectly\_movable  &
    值可以从indirectly\_readable类型移动到indirectly\_readable类型。 \\
    indirectly\_copyable &
    值可以从indirectly\_readable类型复制到indirectly\_copyable类型。 \\
    mergeable            &
    通过复制元素将排序序列合并为输出序列的算法。                               \\
    sortable             & 将序列修改为有序序列的算法。                \\
    permutable           & 对元素重新排序的算法。                   \\
  \end{talltblr}
\end{table}

C++20中包含的几个主要特性之一(以及概念、模块和协程)是范围。范围库定义了一系列类和函数,用于简化使用范围操作。其中有一组概念,在<ranges>头文件和std::ranges命名空间中定义。其中的一些概念:

\begin{longtblr}
  { colspec = {|l|X|}, hlines, rowhead = 1, rows={m}, row{1} = {c, font=\bfseries} }
  概念                    & 描述                     \\
  range                 &
  类型R为范围,需要提供了一个开始迭代器和一个结束哨兵。                    \\
  sized\_range          &
  类型R是在常数时间内已知大小的范围。                             \\
  view                  &
  类型T是视图,需要提供了恒定时间的复制、移动和赋值操作。                   \\
  input\_range          &
  类型R是一个范围,其迭代器类型需要满足input\_iterator概念。          \\
  output\_range         &
  类型R是一个范围,其迭代器类型需要满足output\_iterator概念。         \\
  forward\_range        &
  类型R是一个范围,其迭代器类型需要满足forward\_iterator概念。        \\
  bidirectional\_range  &
  类型R是一个范围,其迭代器类型需要满足bidirectional\_iterator概念。  \\
  random\_access\_range &
  类型R是一个范围,其迭代器类型需要满足random\_access\_iterator概念。 \\
  contiguous\_range     &
  类型R是一个范围,其迭代器类型需要满足contiguous\_iterator概念。     \\
\end{longtblr}

以下是其中一些概念的定义:

\begin{cpp}
template< class T >
concept range = requires( T& t ) {
	ranges::begin(t);
	ranges::end (t);
};

template< class T >
concept sized_range = ranges::range<T> &&
	requires(T& t) {
		ranges::size(t);
	};

template< class T >
concept input_range = ranges::range<T> &&
	std::input_iterator<ranges::iterator_t<T>>;
\end{cpp}

如前所述,这里列出的概念要比这里列出的多得多,未来可能还会增加。本节不打算作为标准概念的完整参考,而是作为它们的介绍。感兴趣的读者可以在\url{https://en.cppreference.com/}官方C++参考文档了解更多关于这些概念的信息。至于范围库,我们将在第8章中了解到更多,并探索标准库提供了什么。

