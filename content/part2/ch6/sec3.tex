\section{探索requires表达式}

requires表达式可能是一个复杂表达式,如前面的容器概念示例中所示。requires表达式的形式与函数语法非常相似:

\begin{cpp}
requires (parameter-list) { requirement-seq }
\end{cpp}

参数列表是一个以逗号分隔的参数列表。与函数声明的唯一区别是不允许使用默认值,但此列表中指定的参数不具有存储空间、链接或生命周期。编译器不为它们分配任何内存,只用于定义需求。其有一个作用域,那就是requires表达式的右大括号。

requirements-seq是一系列需求,每个这样的需求都必须以分号结束。这里,有四种类型的需求:

\begin{itemize}
\item
简单需求

\item
类型需求

\item
复合需求

\item
嵌套需求
\end{itemize}

这些需求的参考如下:

\begin{itemize}
\item
模板范围内的参数

\item
requires表达式的参数列表中可以引入局部参数

\item
封闭上下文中可见的其他声明
\end{itemize}

下面的小节中,将探讨所有提到的需求类型。

\subsubsection{6.3.1\hspace{0.2cm}简单需求}

简单需求是一个不求值,只检查正确性的表达式。表达式必须是有效的,以便对需求求值为true。因为它定义了嵌套的需求,所以表达式不能以requires关键字开头。

前面定义算术和容器概念时,已经看到了简单语句的示例。让我们再看几个:

\begin{cpp}
template<typename T>
concept arithmetic = requires
{
	std::is_arithmetic_v<T>;
};

template <typename T>
concept addable = requires(T a, T b)
{
	a + b;
};

template <typename T>
concept logger = requires(T t)
{
	t.error("just");
	t.warning("a");
	t.info("demo");
};
\end{cpp}

第一个概念arithmetic,和之前定义的一样。std::is\_arithmetic\_v<T>表达式是一个简单的需求。当参数列表为空时,可以省略,只检查T类型模板参数是否为算术类型。

addable和logger概念都有一个参数列表,因为正在检查T类型值的操作。表达式a + b是一个简单的要求,因为编译器只检查加号运算符是否为T类型重载。最后一个例子中,确保T类型有三个成员函数,分别是error、warning和info,可以接受一个const char*类型的参数,或可以从const char*构造的其他类型的参数。但作为参数传递的实际值并不重要,因为这些调用从未执行过,只检查其正确性。

简要地阐述最后一个例子,并看看以下代码:

\begin{cpp}
template <logger T>
void log_error(T& logger)
{}

struct console_logger
{
	void error(std::string_view text){}
	void warning(std::string_view text) {}
	void info(std::string_view text) {}
};

struct stream_logger
{
	void error(std::string_view text, bool = false) {}
	void warning(std::string_view text, bool = false) {}
	void info(std::string_view text, bool) {}
};
\end{cpp}

log\_error函数模板需要一个类型满足记录器要求的参数。这里有两个类,分别为console\_logger和stream\_logger。第一个满足logger的要求,但第二个不满足。这是因为info函数不能用const char*类型的单个参数调用。该函数还需要第二个布尔参数。前两个方法,error和warning,为第二个参数定义了默认值,所以可以通过t.error("just")和warning("a")进行调用。

然而,由于第三个成员函数,stream\_logger不是一个满足预期要求的log类,因此不能与log\_error函数一起使用。console\_logger和stream\_logger的使用示例如下:

\begin{cpp}
console_logger cl;
log_error(cl); // OK

stream_logger sl;
log_error(sl); // error
\end{cpp}

下一节中,将讨论第二类需求:类型需求。

\subsubsection{6.3.2\hspace{0.2cm}类型需求}

类型需求是通过关键字typename,后面跟着类型名称引入的。定义容器约束时,已经看到了几个例子。类型名必须有效,需求才为真。类型需求可用于以下几个目的:

\begin{itemize}
\item
验证嵌套类型是否存在(例如typename T::value\_type;)

\item
验证类模板特化是否命名了类型

\item
验证别名模板特化是否命名了类型
\end{itemize}

看几个例子来学习如何使用类型需求。第一个例子中,检查一个类型是否包含内部类型,key\_type和value\_type:

\begin{cpp}
template <typename T>
concept KVP = requires
{
	typename T::key_type;
	typename T::value_type;
};

template <typename T, typename V>
struct key_value_pair
{
	using key_type = T;
	using value_type = V;
	
	key_type key;
	value_type value;
};

static_assert(KVP<key_value_pair<int, std::string>>);
static_assert(!KVP<std::pair<int, std::string>>);
\end{cpp}

类型key\_value\_pair<int, std::string>满足这些类型要求,但std::pair<int, std::string>不满足。pair类型确实有内部类型,但称为first\_type和second\_type。

第二个例子中,检查类模板特化是否命名了类型。类模板是container,特化是container<T>:

\begin{cpp}
template <typename T>
requires std::is_arithmetic_v<T>
struct container
{ /* ... */ };

template <typename T>
concept containerizeable = requires {
	typename container<T>;
};

static_assert(containerizeable<int>);
static_assert(!containerizeable<std::string>);
\end{cpp}

这段代码中,container是一个类模板,只能针对算术类型特化,例如int、long、float或double。因此,特化container<int>存在,但container<std::string>不存在。containerizeable概念指定了类型T定义有效的container特化的需求。因此,containerizeable<int>为true,但containerizeable<std::string>为false。

现在已经理解了简单的需求和类型需求,是时候探索更复杂的需求类别了。

\subsubsection{6.3.3\hspace{0.2cm}复合需求}

简单需求允许验证表达式是否有效,但有时需要验证表达式的某些属性,而不仅仅是其是否有效。这可以包括表达式是否对结果类型抛出异常或要求(例如函数的返回类型)。一般形式如下:

\begin{cpp}
{ expression } noexcept -> type_constraint;
\end{cpp}

noexcept规范和type\_constraint(前导为->)都是可选的。替换过程和约束的检查如下所示:

\begin{enumerate}
\item
替换表达式中的模板参数。

\item
若指定了noexcept,则表达式不能抛出异常;否则,需求为false。

\item
若存在类型约束,则模板参数也会替换为type\_constraint,并且decltype((expression))必须满足type\_constraint所施加的条件;否则,需求为false。
\end{enumerate}

这里讨论几个示例,来学习如何使用复合需求。第一个例子中,检查函数是否标为noexcept:

\begin{cpp}
template <typename T>
void f(T) noexcept {}

template <typename T>
void g(T) {}

template <typename F, typename ... T>
concept NonThrowing = requires(F && func, T ... t)
{
	{func(t...)} noexcept;
};

template <typename F, typename ... T>
	requires NonThrowing<F, T...>
void invoke(F&& func, T... t)
{
	func(t...);
}
\end{cpp}

这里,有两个函数模板:f声明为noexcept;,所以不会抛出异常,而g可能会抛出异常。NonThrowing概念强制要求类型为F的可变函数不能抛出异常。因此,在以下两个调用中,只有第一个是有效的,而第二个将产生编译错误:

\begin{cpp}
invoke(f<int>, 42);
invoke(g<int>, 42); // error
\end{cpp}

Clang生成的错误消息如下表所示:

\begin{shell}
prog.cc:28:7: error: no matching function for call to 'invoke'
      invoke(g<int>, 42);
       ^~~~~~
prog.cc:18:9: note: candidate template ignored: constraints not
satisfied [with F = void (&)(int), T = <int>]
   void invoke(F&& func, T... t)
          ^
prog.cc:17:16: note: because 'NonThrowing<void (&)(int), int>'
evaluated to false
      requires NonThrowing<F, T...>
                  ^
prog.cc:13:20: note: because 'func(t)' may throw an exception
      {func(t...)} noexcept;
                       ^
\end{shell}

因为g<int>可能抛出异常,导致nonthrows <F, T…>计算为false,所以invoke(g<int>,42)无效。

对于第二个示例,我们将定义一个概念,为timer类提供需求。它要求存在一个名为start的函数,可以在没有参数的情况下调用它,并且它返回void。还要求存在第二个名为stop的函数,可以在没有任何参数的情况下调用,并且返回一个可以转换为long long的值。概念定义如下:

\begin{cpp}
template <typename T>
concept timer = requires(T t)
{
	{t.start()} -> std::same_as<void>;
	{t.stop()} -> std::convertible_to<long long>;
};
\end{cpp}

注意,类型约束不能是任何编译时布尔表达式,而是实际的类型需求,所以使用其他概念来指定返回类型。std::same\_as和std::convertible\_to都是在标准库<concepts>头文件中可用的概念。现在,考虑以下计时器的类实现:

\begin{cpp}
struct timerA
{
	void start() {}
	long long stop() { return 0; }
};

struct timerB
{
	void start() {}
	int stop() { return 0; }
};

struct timerC
{
	void start() {}
	void stop() {}
	long long getTicks() { return 0; }
};

static_assert(timer<timerA>);
static_assert(timer<timerB>);
static_assert(!timer<timerC>);
\end{cpp}

本例中,timerA满足timer概念,包含两个必需的方法:返回void的start方法和返回long long的stop方法。类似地,timerB也满足timer概念,具有相同的方法,即使stop返回int。然而,int类型可隐式转换为long long类型,所以需要满足类型要求。最后,timerC也有相同的方法,但返回void,所以停止返回类型的类型要求不满足,因此,timer概念的约束条件不满足。

最后一类需要研究的需求是嵌套需求。

\subsubsection{6.3.4\hspace{0.2cm}嵌套需求}

最后一类需求是嵌套需求。嵌套需求是通过requires关键字引入的(简单需求不使用requires关键字引入),其形式如下:

\begin{cpp}
requires constraint-expression;
\end{cpp}

表达式必须由替换的参数来满足,将模板参数替换为约束表达式,只是为了检查表达式是否满足。

下面的例子中,要定义一个函数,对可变数量的参数执行加法。但我们想提一些条件:

\begin{itemize}
\item
有多个参数。

\item
所有的参数都有相同的类型。

\item
表达式arg1 + arg2 +…+ argn是有效的。
\end{itemize}

这里,我们定义了一个名为HomogenousRange的概念,如下所示:

\begin{cpp}
template<typename T, typename... Ts>
inline constexpr bool are_same_v =
	std::conjunction_v<std::is_same<T, Ts>...>;
	
template <typename ... T>
concept HomogenousRange = requires(T... t)
{
	(... + t);
	requires are_same_v<T...>;
	requires sizeof...(T) > 1;
};
\end{cpp}

这个概念包含一个简单的需求和两个嵌套的需求。一个嵌套需求使用are\_same\_v变量模板,其值由一个或多个类型特征(std::is\_same)连接决定的,另一个是编译时布尔表达式size…(T) > 1。

使用这个概念,可以这样定义add可变函数模板:

\begin{cpp}
template <typename ... T>
requires HomogenousRange<T...>
auto add(T&&... t)
{
	return (... + t);
}

add(1, 2); // OK
add(1, 2.0); // error, types not the same
add(1); // error, size not greater than 1
\end{cpp}

前面举例的第一个调用是正确的,因为有两个参数,而且都是int类型。第二次调用产生错误,因为实参的类型不同(int和double)。类似地,第三个调用也会产生错误,因为只提供了一个参数。

HomogenousRange概念也可以在几个static\_assert的辅助下测试,如下所示:

\begin{cpp}
static_assert(HomogenousRange<int, int>);
static_assert(!HomogenousRange<int>);
static_assert(!HomogenousRange<int, double>);
\end{cpp}

我们已经介绍了可用于定义约束的所有需求表达式的类别,约束也可以组合,这就是我们接下来要讨论的内容。
















