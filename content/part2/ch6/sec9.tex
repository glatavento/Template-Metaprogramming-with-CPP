\section{更多指定约束的方法}
本章中,已经讨论了requires子句和requires表达式。虽然两者都是用新的requires关键字引入的,但它们是不同的东西:

\begin{itemize}
\item
requires子句决定函数是否参与重载解析,这取决于编译时布尔表达式的值。

\item
requires表达式确定一个或多个表达式的集合是否格式良好,而不会对程序的行为产生任何副作用。requires表达式是一个布尔表达式,可以与requires子句一起使用。
\end{itemize}

再来看一个例子:

\begin{cpp}
template <typename T>
concept addable = requires(T a, T b) { a + b; };
                       // [1] requires expression
template <typename T>
requires addable<T> // [2] requires clause
auto add(T a, T b)
{
	return a + b;
}
\end{cpp}

[1]以requires关键字开头的构造是一个requires表达式。验证表达式a + b对于任何T类型都是格式良好的。另一方面,[2]的构造是一个requires子句。若布尔表达式addable<T>的值为true,则该函数参与重载解析;否则,不会。

虽然requires子句应该使用概念,但也可以使用requires表达式。基本上,概念定义中可以放在=标记右侧的东西,都可以与requires子句一起使用:

\begin{cpp}
template <typename T>
	requires requires(T a, T b) { a + b; }
auto add(T a, T b)
{
	return a + b;
}
\end{cpp}

虽然这是完全合法的代码,但是否是使用约束的好方法却有有争议。我建议避免创建以requires requires开头的构造,因为其可读性较差,可能会造成混乱。此外,命名概念可以在其他地方使用,而若需要用于多个函数,则必须复制带有requires表达式的requires子句。

现在已经了解了如何使用约束和概念以多种方式约束模板实参,让我们看看如何简化函数模板语法,并约束模板实参。




































































