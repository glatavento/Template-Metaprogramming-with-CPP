\section{组合约束}

我们已经看到了多个约束模板参数的例子,但目前为止,我们使用的都是单一约束。不过,可以使用\&\&和||操作符组合约束。使用\&\&操作符组合两个约束称为合取,使用||操作符组合两个约束称为析取。

要使一个合取为真,两个约束都必须为真。与逻辑与操作类似,两个约束从左到右求值,若左边的约束为假,则不对右边的约束求值。来看一个例子:

\begin{cpp}
template <typename T>
requires std::is_integral_v<T> && std::is_signed_v<T>
T decrement(T value)
{
	return value--;
}
\end{cpp}

这段代码中,有一个函数模板,返回接收到的参数的递减值,但只接受带符号的整数值。这是通过连接两个约束来指定的,std::is\_integral\_v<T> \&\& std::is\_signed\_v<T>。使用不同的方法定义连接可以获得相同的结果:

\begin{cpp}
template <typename T>
concept Integral = std::is_integral_v<T>;

template <typename T>
concept Signed = std::is_signed_v<T>;

template <typename T>
concept SignedIntegral = Integral<T> && Signed<T>;

template <SignedIngeral T>
T decrement(T value)
{
	return value--;
}
\end{cpp}

可以看到这里定义了三个概念:一个约束整型,一个约束有符号类型,一个约束整型和有符号类型。

析取的工作原理与此类似。要使析取为真,至少有一个约束条件必须为真。若左边的约束为真,则不对右边的约束求值。再看一个例子,若还记得本章第一节的add函数模板,我们用std::is\_arithmetic类型特征来约束它。还可以使用std::is\_integral和std::is\_float\_point得到相同的结果:

\begin{cpp}
template <typename T>
requires std::is_integral_v<T> || std::is_floating_point_v<T>
T add(T a, T b)
{
	return a + b;
}
\end{cpp}

表达式std::is\_integral\_v<T> || std::is\_float\_point\_v<T>定义了两个原子约束的析取。稍后我们将更详细地讨论这种约束。目前,原子约束是bool类型的表达式,不能分解为更小的部分。类似地,对于我们之前所做的,也可以建立一个析取的概念并使用:

\begin{cpp}
template <typename T>
concept Integral = std::is_integral_v<T>;

template <typename T>
concept FloatingPoint = std::is_floating_point_v<T>;

template <typename T>
concept Number = Integral<T> || FloatingPoint<T>;

template <Number T>
T add(T a, T b)
{
	return a + b;
}
\end{cpp}

如前所述,合取和析取是短路的。这对检查程序的正确性有重要的意义。考虑形式为A<T> \&\& B<T>的连接,则首先检查并求值A<T>;若为假,则不再检查第二个约束B<T>。

类似地,对于A<T> || B<T>的析取,在检查A<T>后,若计算结果为真,则不会检查第二个约束B<T>。若希望检查两个合取的格式是否良好,要么在之后通过其布尔值进行确定,要么不使用\&\&和||操作符。只有当\&\&和||分别嵌套在括号中或作为\&\&或||的操作数出现时,才会形成合取或析取。否则,这些操作符将视为逻辑操作符。这里,用例子来解释一下:

\begin{cpp}
template <typename T>
requires A<T> || B<T>
void f() {}

template <typename T>
requires (A<T> || B<T>)
void f() {}

template <typename T>
requires A<T> && (!A<T> || B<T>)
void f() {}
\end{cpp}

所有这些示例中,||定义了一个析取。然而,当在强制转换表达式或逻辑NOT中使用时,\&\&和||定义了一个逻辑表达式:

\begin{cpp}
template <typename T>
requires (!(A<T> || B<T>))
void f() {}

template <typename T>
requires (static_cast<bool>(A<T> || B<T>))
void f() {}
\end{cpp}

这种情况下,首先检查整个表达式的正确性,然后确定其布尔值。后面的例子中,两个表达式!(A<T> || B<T>)和static\_cast<bool>(A<T> || B<T>)都需要包装在另一组括号中,因为requires子句的表达式不能以!标记或进行转换。

合取和析取符不能用于约束模板参数包,但有一种变通方法可以实现它。考虑add函数模板的可变参数实现,要求所有参数必须是整型。可以尝试以以下形式编写约束:

\begin{cpp}
template <typename ... T>
requires std::is_integral_v<T> && ...
auto add(T ... args)
{
	return (args + ...);
}
\end{cpp}

这将产生编译器错误,因为在此上下文中不允许使用省略号。为了避免这个错误,可以把表达式用括号括起来:

\begin{cpp}
	template <typename ... T>
	requires std::is_integral_v<T> && ...
	auto add(T ... args)
	{
		return (args + ...);
	}
\end{cpp}

表达式(std::is\_integral\_v<T> \&\&…)现在是一个折叠表达式,不是一个合取,这样就得到了一个原子约束。编译器将首先检查整个表达式的正确性,然后确定其布尔值。要构建一个合取,首先需要定义一个概念:

\begin{cpp}
template <typename T>
concept Integral = std::is_integral_v<T>;
\end{cpp}

接下来我们需要做的是更改requires子句,以便使用新定义的概念,而不是布尔变量std::is\_integral\_v<T>:

\begin{cpp}
template <typename ... T>
requires (Integral<T> && ...)
auto add(T ... args)
{
	return (args + ...);
}
\end{cpp}

看起来变化不大,由于使用了概念,验证正确性和确定布尔值将分别针对每个模板参数进行。若某个类型的约束不满足,其余部分将短路,验证将停止。

在本节前面,我两次使用了原子约束这个术语。有人会问,什么是原子约束?它是bool类型的表达式,不能进一步分解。原子约束是在约束规范化过程中形成的,编译器将约束分解为原子约束的合取和析取。其工作原理如下:

\begin{itemize}
\item
表达式E1 \&\& E2可分解为E1和E2的合取。

\item
表达式E1 || E2可分解为E1和E2的析取。

\item
概念C<A1, A2,…An>在将所有模板参数替换为其原子约束后,其定义被替换。
\end{itemize}

原子约束用于确定约束的部分排序,这些约束又决定函数模板和类模板特化部分的排序,以及重载解析中非模板函数的下一个候选。











