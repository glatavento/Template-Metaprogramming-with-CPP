\section{使用概念来约束auto参数}
第2章中,讨论了C++14中引入的泛型Lambda,以及C++20中引入的Lambda模板。对至少一个参数使用auto说明符的Lambda称为泛型Lambda。编译器生成的函数对象,将具有模板化的调用操作符。这里有一个例子:

\begin{cppcode}
auto lsum = [](auto a, auto b) {return a + b; };
\end{cppcode}

C++20标准将此特性推广到所有函数,可以在函数参数列表中使用auto。这具有将函数转换为模板函数的效果:

\begin{cppcode}
auto add(auto a, auto b)
{
	return a + b;
}
\end{cppcode}

这是一个函数,接受两个参数并返回和(或者更准确地说,是对两个值应用operator+的结果)。使用auto作为函数参数的函数称为缩写函数模板,是函数模板的简写语法。上述函数的等效模板如下所示:

\begin{cppcode}
template<typename T, typename U>
auto add(T a, U b)
{
	return a + b;
}
\end{cppcode}

可以像调用模板函数一样调用这个函数,编译器将通过用实际类型替换模板参数的方式,生成适当的实例化。考虑以下的调用方式:

\begin{cppcode}
add(4, 2); // returns 6
add(4.0, 2); // returns 6.0
\end{cppcode}

可以用cppinsights.io网站检查编译器生成的代码,以添加基于这两个调用的缩写函数模板。生成以下特化:

\begin{cppcode}
template<>
int add<int, int>(int a, int b)
{
	return a + b;
}

template<>
double add<double, int>(double a, int b)
{
	return a + static_cast<double>(b);
}
\end{cppcode}

由于缩写函数模板只不过是具有简化语法的常规函数模板,这样的函数可以由用户显式特化:

\begin{cppcode}
template<>
auto add(char const* a, char const* b)
{
	return std::string(a) + std::string(b);
}
\end{cppcode}

这是char const*类型的全特化。这种特化能够调用add("4","2"),尽管结果是一个std::string值。

这类简化的函数模板称为无约束函数模板,模板参数没有限制,但可以用概念为它们的参数提供约束。使用概念的简短函数模板称为约束函数模板。下面,会看到一个add函数约束整型的例子:

\begin{cppcode}
auto add(std::integral auto a, std::integral auto b)
{
	return a + b;
}
\end{cppcode}

若再次考虑前面看到的相同调用,第一个将会成功,但第二个将会产生编译器错误,因为没有重载接受double和int值:

\begin{cppcode}
add(4, 2); // OK
add(4.2, 0); // error
\end{cppcode}

约束auto也可以用于可变参数缩写函数模板。示例如下所示:

\begin{cppcode}
auto add(std::integral auto ... args)
{
	return (args + ...);
}
\end{cppcode}

最后,受约束的auto也可以用于通用的Lambda。若希望本节开头所示的泛型Lambda仅用于整型,可以按以下方式对其进行约束:

\begin{cppcode}
auto lsum = [](std::integral auto a, std::integral auto b)
{
	return a + b;
};
\end{cppcode}

随着本节的结束,我们已经了解了C++20中与概念和约束相关的语言特性。接下来要讨论的是标准库提供的概念集。






























