\section{模板中约束的顺序}
当编译器遇到函数调用或类模板实例化时,就需要确定哪种重载(对于函数)或特化(对于类)是最佳匹配。函数可以用不同的类型约束重载,类模板也可以用不同的类型约束特化。为了决定哪个是最佳匹配,编译器需要找出哪个是最匹配约束的,在替换所有模板参数后的同时,计算结果为true。为了弄清楚这一点,需要进行约束归一化。这是将约束表达式转换为原子约束合取和析取的过程,如前一节末尾所述。

若A包含B,则一个原子约束A包含另一个原子约束B。一个约束声明D1的约束包含另一个声明D2的约束,则该约束声明D1至少与D2一样具有约束。若D1至少和D2一样受约束,反之则不成立,所以D1比D2的约束更强。编译器会选择约束更强的重载作为最佳匹配。

为了理解约束如何影响重载解析,这里讨论几个示例。首先,从以下两个重载开始:

\begin{cppcode}
int add(int a, int b)
{
	return a + b;
}

template <typename T>
T add(T a, T b)
{
	return a + b;
}
\end{cppcode}

第一个重载是非模板函数,接受两个int参数并返回它们的和。第二个是我们在本章中已经看到的模板实现。

现在,来看下如下的调用:

\begin{cppcode}
add(1.0, 2.0); // [1]
add(1, 2); // [2]
\end{cppcode}

第一个调用([1])接受两个double值,因此只有模板重载匹配,其double类型的实例化将调用。add函数的第二次调用([2])接受两个整数参数,两个重载都可能匹配。编译器将选择最具体的一个,即非模板重载。

若两个重载都是模板,但其中一个是受约束的,该怎么办?

\begin{cppcode}
template <typename T>
T add(T a, T b)
{
	return a + b;
}

template <typename T>
requires std::is_integral_v<T>
T add(T a, T b)
{
	return a + b;
}
\end{cppcode}

第一个重载是前面看到的函数模板。第二个具有相同的实现,指定了模板参数的要求,该参数仅限于整型。若考虑前面代码片段中的相同两个调用,对于[1]处具有两个double值的调用,只有第一个重载匹配良好。对于[2]处的调用,有两个整数值,两个重载都很好的匹配。然而,第二个重载的约束更强(与第一个没有约束的重载相比,只有一个约束),因此编译器将选择这个重载。

下一个示例中,这两个重载都有约束。第一次重载要求模板实参的大小为4,第二次重载要求模板实参必须是整型:

\begin{cppcode}
template <typename T>
requires (sizeof(T) == 4)
T add(T a, T b)
{
	return a + b;
}

template <typename T>
requires std::is_integral_v<T>
T add(T a, T b)
{
	return a + b;
}
\end{cppcode}

考虑以下对这个重载函数模板的调用:

\begin{cppcode}
add((short)1, (short)2); // [1]
add(1, 2); // [2]
\end{cppcode}

[1]的调用使用short类型的参数。这是一个大小为2的整型,只有第二个重载匹配。但[2]处的调用使用int类型的参数,这是一个大小为4的整型所以,这两个重载都是很好的匹配。这是一种模棱两可的情况,编译器无法在两者之间进行选择,所以会产生编译错误。

但若稍微改变这两个重载,会发生什么呢?

\begin{cppcode}
template <typename T>
requires std::is_integral_v<T>
T add(T a, T b)
{
	return a + b;
}

template <typename T>
requires std::is_integral_v<T> && (sizeof(T) == 4)
T add(T a, T b)
{
	return a + b;
}
\end{cppcode}

这两种重载都要求模板参数必须是整型,但第二次重载还要求整型的大小必须是4字节。对于第二个重载,使用两个原子约束的结合。我们将讨论同样的两次调用,使用short参数和int参数。

对于[1]上的调用,传递两个short值,只有第一个重载是良好的匹配,因此将调用这个重载。对于接受两个int参数的[2]调用,两个重载都是匹配的。第二种情况更受约束,所以编译器无法决定哪个匹配更好,并将发出一个模糊调用的错误。这可能会让开发者感到惊讶,因为在开始时,我说过将从重载集中选择最受约束的重载。因为我们使用类型特征来约束这两个函数,所以其在我们的示例中它不起作用。若使用概念,行为就不一样了:

\begin{cppcode}
template <typename T>
concept Integral = std::is_integral_v<T>;

template <typename T>
requires Integral<T>
T add(T a, T b)
{
	return a + b;
}

template <typename T>
requires Integral<T> && (sizeof(T) == 4)
T add(T a, T b)
{
	return a + b;
}
\end{cppcode}

现在,不再有歧义,编译器将从重载集中选择第二个重载作为最佳匹配。这说明编译器优先处理概念,使用概念使用约束有不同的方法,但前面的定义只是用概念替换了类型特征,所以可以说是演示这种行为的更好选择,而不是下一个实现:

\begin{cppcode}
template <Integral T>
T add(T a, T b)
{
	return a + b;
}

template <Integral T>
requires (sizeof(T) == 4)
T add(T a, T b)
{
	return a + b;
}
\end{cppcode}

本章中讨论的所有示例都涉及到约束函数模板,但可以约束非模板成员函数,以及类模板和类模板特化。我们将在下一节中讨论这些。





























