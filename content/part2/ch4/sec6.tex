\section{decltype说明符}
此说明符在C++11中引入,用于返回表达式的类型,通常在模板中与auto说明符一起使用。可以用于声明依赖于模板参数的函数模板的返回类型,或者包装另一个函数,并返回执行包装函数结果的返回类型。

decltype说明符不限制在模板代码中使用。可以用于不同的表达式,并且根据表达式产生不同的结果。规则如下:

\begin{enumerate}
\item
若表达式是标识符或类成员访问,则结果是由表达式命名的实体类型。若实体不存在,或者是具有重载集的函数(存在多个同名函数),编译器将报错。

\item
若表达式是函数调用或重载操作符函数,则结果为函数的返回类型。若重载的操作符括在括号中,则忽略这些操作符。

\item
若表达式是左值,则结果类型是对表达式类型的左值引用。

\item
若表达式为其他类型,则结果类型为表达式的类型。
\end{enumerate}

为了更好地理解这些规则,来看一组示例。考虑在decltype表达式中使用的以下函数和变量:

\begin{cpp}
int f() { return 42; }
int g() { return 0; }
int g(int a) { return a; }

struct wrapper
{
	int val;
	int get() const { return val; }
};

int a = 42;
int& ra = a;
const double d = 42.99;
long arr[10];
long l = 0;
char* p = nullptr;
char c = 'x';
wrapper w1{ 1 };
wrapper* w2 = new wrapper{ 2 };
\end{cpp}

下面的代码显示了decltype说明符的多种用法。适用于每种情况的规则,以及推导出的类型,都在注释中的每行指定:

\begin{cpp}
decltype(a) e1; // R1, int
decltype(ra) e2 = a; // R1, int&
decltype(f) e3; // R1, int()
decltype(f()) e4; // R2, int
decltype(g) e5; // R1, error
decltype(g(1)) e6; // R2, int
decltype(&f) e7 = nullptr; // R4, int(*)()
decltype(d) e8 = 1; // R1, const double
decltype(arr) e9; // R1, long[10]
decltype(arr[1]) e10 = l; // R3, long&
decltype(w1.val) e11; // R1, int
decltype(w1.get()) e12; // R1, int
decltype(w2->val) e13; // R1, int
decltype(w2->get()) e14; // R1, int
decltype(42) e15 = 1; // R4, int
decltype(1 + 2) e16; // R4, int
decltype(a + 1) e17; // R4, int
decltype(a = 0) e18 = a; // R3, int&
decltype(p) e19 = nullptr; // R1, char*
decltype(*p) e20 = c; // R3, char&
decltype(p[0]) e21 = c; // R3, char&
\end{cpp}

我们不详细说明所有这些声明。根据指定的规则,其中大多数都相对容易遵循。不过,为了说明一些推导出的类型,有几点注意事项值得考虑:

\begin{itemize}
\item
decltype(f)只使用重载集命名函数,因此适用规则1。decltype(g)也命名一个函数,但它有一个重载集。因此,应用规则1,编译器会报错。

\item
decltype(f())和decltype(g(1))都对表达式使用函数调用,因此适用规则2,即使g有重载集,声明也是正确的。

\item
decltype(\&f)使用函数f的地址,因此应用规则4,生成int(*)()。

\item
decltype(1+2)和decltype(a+1)使用重载加法操作符返回右值,因此适用规则4,结果是int。然而,decltype(a = 1)使用返回左值的赋值操作符,因此应用规则3,生成左值引用int\&。
\end{itemize}

decltype说明符定义了一个未求值的上下文。所以与该说明符一起使用的表达式不会进行计算,因为该说明符只查询其操作数的属性。可以在下面的代码片段中看到这一点,其中赋值a=1与decltype说明符一起使用来声明变量e,但在声明之后,a是其初始化时的值:

\begin{cpp}
int a = 42;
decltype(a = 1) e = a;
std::cout << a << '\n'; // prints 42
\end{cpp}

关于模板实例化的规则有一个例外。当与decltype说明符一起使用的表达式包含模板时,需要在编译时计算表达式之前实例化模板:

\begin{cpp}
template <typename T>
struct wrapper
{
	T data;
};

decltype(wrapper<double>::data) e1; // double

int a = 42;
decltype(wrapper<char>::data, a) e2; // int&
\end{cpp}

e1的类型是double,并且实例化了wrapper<double>来推导这个类型。另一方面,e2的类型是int\&(a是左值)。即使该类型仅从变量a推导出来(使用了逗号操作符),wrapper<char>依旧会在这里进行实例化。

上述规则并不是唯一用于确定类型的规则,还有几个用于数据成员访问的方法。具体如下:

\begin{itemize}
\item
decltype表达式中使用的对象的const或volatile说明符不构成推导的类型。

\item
对象或指针表达式是左值还是右值并不影响推导的类型。

\item
若数据成员访问表达式括号括起来,例如decltype((expression)),则前两条规则不适用。对象的const或volatile限定符确实会影响推导的类型,包括对象的值。
\end{itemize}

下面的代码片段演示了前两条规则:

\begin{cpp}
struct foo
{
	int a = 0;
	volatile int b = 0;
	const int c = 42;
};

foo f;
foo const cf;
volatile foo* pf = &f;

decltype(f.a) e1 = 0; // int
decltype(f.b) e2 = 0; // int volatile
decltype(f.c) e3 = 0; // int const

decltype(cf.a) e4 = 0; // int
decltype(cf.b) e5 = 0; // int volatile
decltype(cf.c) e6 = 0; // int const

decltype(pf->a) e7 = 0; // int
decltype(pf->b) e8 = 0; // int volatile
decltype(pf->c) e9 = 0; // int const

decltype(foo{}.a) e10 = 0; // int
decltype(foo{}.b) e11 = 0; // int volatile
decltype(foo{}.c) e12 = 0; // int const
\end{cpp}

右边的注释提到了每种情况的推导类型。当表达式加圆括号时,这两个规则会颠倒过来。来看看下面的代码:

\begin{cpp}
foo f;
foo const cf;
volatile foo* pf = &f;

int x = 1;
int volatile y = 2;
int const z = 3;

decltype((f.a)) e1 = x; // int&
decltype((f.b)) e2 = y; // int volatile&
decltype((f.c)) e3 = z; // int const&

decltype((cf.a)) e4 = x; // int const&
decltype((cf.b)) e5 = y; // int const volatile&
decltype((cf.c)) e6 = z; // int const&

decltype((pf->a)) e7 = x; // int volatile&
decltype((pf->b)) e8 = y; // int volatile&
decltype((pf->c)) e9 = z; // int const volatile&

decltype((foo{}.a)) e10 = 0; // int&&
decltype((foo{}.b)) e11 = 0; // int volatile&&
decltype((foo{}.c)) e12 = 0; // int const&&
\end{cpp}

decltype用于声明变量e1到e9的所有表达式都是左值,推导出的类型是一个左值引用。用于声明变量e10、e11和e12的表达式是右值,推导的类型是一个右值引用。此外,cf是一个常量对象,foo::a的类型是int,结果类型是const int\&。类似地,foo::b的类型为volatile int,结果类型是const volatile int\&。这些只是本段代码中的几个示例,但其他示例遵循相同的推导规则。

因为decltype是一个类型说明符,所以多余的const和volatile限定符以及引用说明符将忽略:

\begin{cpp}
int a = 0;
int& ra = a;
int const c = 42;
int volatile d = 99;

decltype(ra)& e1 = a; // int&
decltype(c) const e2 = 1; // int const
decltype(d) volatile e3 = 1; // int volatile
\end{cpp}

现在,已经了解了decltype说明符的工作方式。其真正目的是在模板中使用decltype,其中函数的返回值取决于模板参数,并且在实例化之前是未知的。为了理解这个场景,先从下面的函数模板示例开始,该函数模板返回两个值中的最小值:

\begin{cpp}
template <typename T>
T minimum(T&& a, T&& b)
{
	return a < b ? a : b;
}
\end{cpp}

可以这样使用:

\begin{cpp}
auto m1 = minimum(1, 5); // OK
auto m2 = minimum(18.49, 9.99);// OK
auto m3 = minimum(1, 9.99);
                    // error, arguments of different type
\end{cpp}

因为提供的参数是相同类型的,所以前两次调用都是正确的。第三次调用将产生编译器错误,因为参数具有不同的类型,所以需要将整数值强制转换为double类型。当然,还有另一种选择:可以编写一个函数模板,接受两个可能不同类型的参数,并返回这两个参数中的最小值。如下所示:

\begin{cpp}
template <typename T, typename U>
??? minimum(T&& a, U&& b)
{
	return a < b ? a : b;
}
\end{cpp}

问题是,这个函数的返回类型是什么呢?根据所使用的标准版本,可以以不同的方式实现。

C++11中,可以使用带有尾随返回类型的auto说明符,使用decltype说明符从表达式推导返回类型:

\begin{cpp}
template <typename T, typename U>
auto minimum(T&& a, U&& b) -> decltype(a < b ? a : b)
{
	return a < b ? a : b;
}
\end{cpp}

若使用C++14或标准的更新版本,则可以简化此语法。后面的返回类型将不再需要:

\begin{cpp}
template <typename T, typename U>
decltype(auto) minimum(T&& a, U&& b)
{
	return a < b ? a : b;
}
\end{cpp}

还可以进一步简化,简单地使用auto作为返回类型:

\begin{cpp}
template <typename T, typename U>
auto minimum(T&& a, U&& b)
{
	return a < b ? a : b;
}
\end{cpp}

虽然decltype(auto)和auto在本例中具有相同的效果,但情况并非总是如此。看看下面的例子,有一个函数返回一个引用,而另一个函数完美地转发了相关参数:

\begin{cpp}
template <typename T>
T const& func(T const& ref)
{
	return ref;
}

template <typename T>
auto func_caller(T&& ref)
{
return func(std::forward<T>(ref));
}

int a = 42;
decltype(func(a)) r1 = func(a); // int const&
decltype(func_caller(a)) r2 = func_caller(a); // int
\end{cpp}

函数func返回一个引用,而func\_caller应该完全转发到这个函数。使用auto作为返回类型,在前面的代码片段中推断为int(参见变量r2)。为了完美地转发返回类型,必须对其使用decltype(auto):

\begin{cpp}
template <typename T>
decltype(auto) func_caller(T&& ref)
{
	return func(std::forward<T>(ref));
}

int a = 42;
decltype(func(a)) r1 = func(a); // int const&
decltype(func_caller(a)) r2 = func_caller(a); // int const&
\end{cpp}

这一次,结果和预期的一样,并且这个代码片段中的r1和r2的类型都是int const\&。

正如在本节中看到的,decltype是一个类型说明符,用于推断表达式的类型。可以在不同的上下文中使用,但其目的是让模板确定函数的返回类型,并确保完美转发。与decltype一起出现的另一个特性是std::declval,我们将在下一节中对其进行讨论。









