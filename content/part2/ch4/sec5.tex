\section{转发引用}
C++11中添加到该语言的最重要的特性之一是移动语义,通过避免不必要的复制来帮助提高性能。移动语义由C++11的另一个特性支持,称为右值引用。讨论这些之前,值得一提的是,在C++中有两种值:

\begin{itemize}
\item
左值是指向内存位置的值,可以用\&操作符获取它们的地址。左值可以出现在赋值表达式的左边和右边。

\item
右值是非左值的值,它们的定义互斥。右值不指向内存位置,并且不能使用\&操作符获取它们的地址。右值是字面量和临时对象,只能出现在赋值表达式的右侧。

\begin{note}
C++11中,还有一些其他的值类别,glvalue, prvalue和xvalue。在这里讨论它们会让我们的注意力过于分散,可以在\url{https://en.cppreference.com/w/cpp/ language/value_category}上阅读更多关于它们的信息。
\end{note}

\end{itemize}

引用是已经存在的对象或函数的别名。就像我们有两种值一样,C++11中有两种引用:

\begin{itemize}
\item
用\&表示的左值引用,例如\&x,是对左值的引用。

\item
用\&\&表示的右值引用,例如在\&\&x中,是对右值的引用。
\end{itemize}

来看一些例子,可以更好地理解这些概念:

\begin{cpp}
struct foo
{
	int data;
};

void f(foo& v)
{ std::cout << "f(foo&)\n"; }

void g(foo& v)
{ std::cout << "g(foo&)\n"; }

void g(foo&& v)
{ std::cout << "g(foo&&)\n"; }

void h(foo&& v)
{ std::cout << "h(foo&&)\n"; }

foo x = { 42 }; // x is lvalue
foo& rx = x; // rx is lvalue
\end{cpp}

这里有三个函数:f,接受一个左值引用(即int\&);g,有两个重载,一个用于左值引用,另一个用于右值引用(即int\&\&);h,取一个右值引用。还有两个变量,x和rx。x是一个左值,类型为foo,可以用\&x看到它的地址。左值也是rx,它是一个左值引用,类型是foo\&。现在,来看看如何调用f、g和h函数:

\begin{cpp}
f(x); // f(foo&)
f(rx); // f(foo&)
f(foo{42}); // error: a non-const reference
            // may only be bound to an lvalue
\end{cpp}

因为x和rx都是左值,所以将它们传递给f没问题,因为这个函数接受一个左值引用。然而,foo\{42\}是一个临时对象,并且因为它不存在于调用f的上下文之外,所以其是一个右值,将它传递给f将导致编译器错误,因为函数的形参是foo\&类型的,非常量引用只能绑定到左值。若函数f的签名更改为f(int const \&v),就没问题了。接下来,讨论g函数:

\begin{cpp}
g(x); // g(foo&)
g(rx); // g(foo&)
g(foo{ 42 }); // g(foo&&)
\end{cpp}

前面的代码段中,使用x或rx调用g将解析为第一个重载,该重载接受左值引用。但使用foo\{42\}(foo\{42\}是一个临时对象,因此是一个右值)将解析为第二个重载,将接受一个右值引用。看当我们想对h函数进行相同的调用时会发生什么:

\begin{cpp}
h(x); // error, cannot bind an lvalue to an rvalue ref
h(rx); // error
h(foo{ 42 }); // h(foo&&)
h(std::move(x)); // h(foo&&)
\end{cpp}

该函数可以接受一个右值引用,但将x或rx传递给它会导致编译器错误,因为左值不能绑定到右值引用。表达式foo\{42\}是一个右值,可以作为参数传递。若把其语义从左值改为右值(可在std::move的帮助下完成),也可以把左值x传递给函数h。std::move并不真正的移动,只是做了一种从左值到右值的转换而已。

理解将右值传递给函数有两个目的:要么对象是临时的,函数可以对它做任何事情,要么函数应该拥有接收对象。这就是移动构造函数和移动赋值操作符的目的,其实很少会看到其他函数接受右值引用。在最后一个例子的函数h中,参数v是一个左值,但绑定为右值。变量x存在于对h之外,将其传递给std::move可转换为右值。在调用h返回,其仍然作为左值存在,但需要假设函数h对它做了一些事情,它的状态并不确定。

因此,右值引用的目的是启用移动语义。还有另一个功能,那就是实现完美转发。为了理解这一点,来看看以下对函数g和h的修改:

\begin{cpp}
void g(foo& v) { std::cout << "g(foo&)\n"; }
void g(foo&& v) { std::cout << "g(foo&&)\n"; }

void h(foo& v) { g(v); }
void h(foo&& v) { g(v); }
\end{cpp}

这个代码片段中,g的实现与前面看到的相同。h也有两个重载,一个接受左值引用并调用g,另一个接受右值引用并调用g。换句话说,函数h只是将参数转发给g。现在,看看以下调用方式:

\begin{cpp}
foo x{ 42 };
h(x); // g(foo&)
h(foo{ 42 }); // g(foo&)
\end{cpp}

由此,可以期望调用h(x)将使g的重载使用左值引用,而对h(foo\{42\})的调用将使g的重载的使用右值引用。但事实上,它们都会调用g的第一个重载,因此将g(foo\&)打印到控制台。当理解了引用的工作原理,解释就很简单了:在上下文中h(foo\&\& v),参数v实际上是一个左值(有名字,可以取它的地址),所以用它调用g会调用接受左值引用的重载。为了使其如预期那样工作,需要改变h函数的实现:

\begin{cpp}
void h(foo& v) { g(std::forward<foo&>(v)); }
void h(foo&& v) { g(std::forward<foo&&>(v)); }
\end{cpp}

std::forward是一个允许正确转发值的函数。函数的作用为:

\begin{itemize}
\item
若参数是一个左值引用,函数的行为就像调用std::move(将语义从左值更改为右值)。

\item
若实参是右值引用,则什么也不做。
\end{itemize}

目前为止,我们讨论的所有内容都与模板无关,而模板才是本书的主题,而函数模板也可以用于接收左值和右值引用,了解在非模板场景中的工作原理非常重要。在模板中,右值引用的工作方式略有不同,有时右值引用,但实际上有时是左值引用。

表现出这种行为的引用称为转发引用,也称为\textbf{通用引用}。在C++11之后不久,由Scott Meyers创造的一个术语,当时标准中还没有这种类型的引用术语。因为通用引用这个术语不能很好地描述它们的语义,为了解决这个问题,C++标准委员会在C++14中将这些引用称为\textbf{转发引用}。为了忠实于标准术语,我们在本书中称其为转发引用。

开始讨论转发引用之前,我们考虑以下重载函数模板和类模板:

\begin{cpp}
template <typename T>
void f(T&& arg) // forwarding reference
{ std::cout << "f(T&&)\n"; }

template <typename T>
void f(T const&& arg) // rvalue reference
{ std::cout << "f(T const&&)\n"; }

template <typename T>
void f(std::vector<T>&& arg) // rvalue reference
{ std::cout << "f(vector<T>&&)\n"; }

template <typename T>
struct S
{
	void f(T&& arg) // rvalue reference
	{ std::cout << "S.f(T&&)\n"; }
};
\end{cpp}

可以像下面这样调用这些函数:

\begin{cpp}
int x = 42;
f(x); // [1] f(T&&)
f(42); // [2] f(T&&)

int const cx = 100;
f(cx); // [3] f(T&&)
f(std::move(cx)); // [4] f(T const&&)

std::vector<int> v{ 42 };
f(v); // [5] f(T&&)
f(std::vector<int>{42});// [6] f(vector<T>&&)

S<int> s;
s.f(x); // [7] error
s.f(42); // [8] S.f(T&&)
\end{cpp}

这段代码中,可以注意到:

\begin{itemize}
\item
{}[1]和[2]处用左值或右值使用f时,将解析为第一个重载:f(T\&\&)。

\item
{}[3]处调用左值为常数的f也会解析为第一个重载,但[4]处调用右值为常数的f会解析为第二个重载f(T const\&\&),这个匹配度更高。

\item
{}[5]处用左值std::vector对象调用f会解析为第一个重载,[6]处用右值std::vector对象调用f会解析为第三个重载,f(vector<T>\&\&)。

\item
{}[7]处用左值调用S::f是错误的,因为左值不能绑定到右值引用,[8]处用右值调用是正确的。
\end{itemize}

本例中的所有f函数重载都采用右值引用。第一次重载中的\&\&并不一定是右值引用。若传递的是右值,则表示右值引用;若传递的是左值,则表示左值引用。这样的引用称为转发引用,但转发引用只出现在模板形参右值引用的上下文中,必须是T\&\&形式。T const\&\&或std::vector<T>\&\&不是转发引用,而是正常的右值引用。类似地,类模板S的f函数成员中的T\&\&也是右值引用,因为f不是模板,而是类模板的非模板成员函数,所以转发引用的规则在这里不适用。

转发引用是函数模板参数推断的一种特殊情况,这是之前在本章中讨论过的主题。其目的是实现模板的完美转发,这是C++11的新特性,引用折叠实现的。在展示是如何解决完美转发问题之前,让我们先看看个例子。

C++11之前,不可能将一个引用引用于另一个引用。现在C++11中,对于typedef和模板来说,这是可能的:

\begin{cpp}
using lrefint = int&;
using rrefint = int&&;
int x = 42;
lrefint& r1 = x; // type of r1 is int&
lrefint&& r2 = x; // type of r2 is int&
rrefint& r3 = x; // type of r3 is int&
rrefint&& r4 = 1; // type of r4 is int&&
\end{cpp}

规则非常简单:对右值引用的右值引用折叠为右值引用,其他组合都折算为左值引用。可以用表格形式表示如下:

\begin{table}[H]
\centering
	\begin{tabular}{|l|l|l|}
		\hline
		\textbf{类型} & \textbf{引用类型} & \textbf{变量类型} \\ \hline
		T\&           & T\&                        & T\&                       \\ \hline
		T\&           & T\&\&                      & T\&                       \\ \hline
		T\&\&         & T\&                        & T\&                       \\ \hline
		T\&\&         & T\&\&                      & T\&\&                     \\ \hline
	\end{tabular}
\end{table}

\begin{center}
表 4.2
\end{center}

下表所示的其他组合都不涉及引用折叠规则,只有当这两种类型都是引用时才适用:

\begin{table}[H]
\centering
	\begin{tabular}{|l|l|l|}
		\hline
		\textbf{类型} & \textbf{引用类型} & \textbf{变量类型} \\ \hline
		T             & T                          & T                         \\ \hline
		T             & T\&                        & T\&                       \\ \hline
		T             & T\&\&                      & T\&\&                     \\ \hline
		T\&           & T                          & T\&                       \\ \hline
		T\&\&         & T                          & T\&\&                     \\ \hline
	\end{tabular}
\end{table}

\begin{center}
表 4.3
\end{center}

转发引用不仅适用于模板,也适用于自动推导规则。当使用auto\&\&时,其意味着转发引用。这同样不适用于其他任何方式,比如auto const\&\&这样的cv限定方式。下面是一些例子:

\begin{cpp}
int x = 42;
auto&& rx = x; // [1] int&
auto&& rc = 42; // [2] int&&
auto const&& rcx = x; // [3] error

std::vector<int> v{ 42 };
auto&& rv = v[0]; // [4] int&
\end{cpp}

前两个例子中,rx和rc都是转发引用,分别绑定到左值和右值。然而,rcx是一个右值引用,因为auto const\&\&并不表示转发引用,所以尝试将其绑定为左值是错误的。类似地,rv是一个转发引用,并绑定为左值。

如前所述,转发引用的目的是实现完美转发。我们已经在前面的非模板上下文中看到了完美转发的概念,其工作方式与模板类似。为了演示这一点,这里将函数h重新定义为模板函数:

\begin{cpp}
void g(foo& v) { std::cout << "g(foo&)\n"; }
void g(foo&& v) { std::cout << "g(foo&&)\n"; }

template <typename T> void h(T& v) { g(v); }
template <typename T> void h(T&& v) { g(v); }

foo x{ 42 };
h(x); // g(foo&)
h(foo{ 42 }); // g(foo&)
\end{cpp}

g重载的实现是相同的,但h重载现在是函数模板。用一个左值和一个右值调用h实际上解析为对g的相同调用,第一个重载取一个左值。因为在函数h的上下文中,所以v是一个左值,将它传递给g将调用取左值的重载。

这个问题的解决方案与讨论模板之前已经看到的解决方案相同,这里有一个区别:不再需要两个重载,而是需要一个转发引用:

\begin{cpp}
template <typename T>
void h(T&& v)
{
	g(std::forward<T>(v));
}
\end{cpp}

这个实现使用std::forward将左值作为左值传递,将右值作为右值传递,同样适用于可变函数模板。下面是std::make\_unique函数的概念实现,其创建了一个std::unique\_ptr对象:

\begin{cpp}
	void g(foo& v) { std::cout << "g(foo&)\n"; }
	void g(foo&& v) { std::cout << "g(foo&&)\n"; }
	
	template <typename T> void h(T& v) { g(v); }
	template <typename T> void h(T&& v) { g(v); }
	
	foo x{ 42 };
	h(x); // g(foo&)
	h(foo{ 42 }); // g(foo&)
\end{cpp}

总结一下本节,请记住转发引用(也称为通用引用)基本上是针对函数模板参数的特殊推导规则。基于引用折叠规则工作,目的是实现完全转发。这是通过保留值语义将引用传递给另一个函数:右值应作为右值传递,左值应作为左值传递。

本章中我们将讨论的下一个主题是decltype说明符。













