\section{函数模板的参数推导}
编译器有时可以从函数调用的上下文推导出模板参数,从而可以不用显式地指定模板参数类型。模板参数推断的规则更复杂,我们将在本节中探讨这个主题。

从一个简单的例子开始:

\begin{cppcode}
template <typename T>
void process(T arg)
{
	std::cout << "process " << arg << '\n';
}

int main()
{
	process(42); // [1] T is int
	process<int>(42); // [2] T is int, redundant
	process<short>(42); // [3] T is short
}
\end{cppcode}

这段代码中,process是一个具有单一类型模板形参的函数模板。process(42)和process<int>(42)等价,在第一种情况下,编译器能够从传递给函数的参数值推断类型模板形参T的类型为int。

当编译器试图推断模板参数时,会将模板形参的类型与用于调用函数的实参的类型的匹配。有一些规则可以控制这种匹配。编译器可以匹配以下内容:

\begin{itemize}
  \item 形式为T, T const, T volatile的类型(包括cv-类型和non-类型):

\begin{cppcode}
struct account_t
{
	int number;
};

template <typename T>
void process01(T) { std::cout << "T\n"; }

template <typename T>
void process02(T const) { std::cout << "T const\n"; }

template <typename T>
void process03(T volatile) { std::cout << "T volatile\n";
}

int main()
{
	account_t ac{ 42 };
	process01(ac); // T
	process02(ac); // T const
	process03(ac); // T volatile
}
\end{cppcode}
  \item 指针(T*)、左值引用(T\&)和右值引用(T\&\&):

\begin{cppcode}
template <typename T>
void process04(T*) { std::cout << "T*\n"; }

template <typename T>
void process04(T&) { std::cout << "T&\n"; }

template <typename T>
void process05(T&&) { std::cout << "T&&\n"; }

int main()
{
	account_t ac{ 42 };
	process04(&ac); // T*
	process04(ac); // T&
	process05(ac); // T&&
}
\end{cppcode}
  \item 数组,如T[5]或C[5][n],其中C是类类型,n是非类型模板参数:

\begin{cppcode}
template <typename T>
void process06(T[5]) { std::cout << "T[5]\n"; }

template <size_t n>
void process07(account_t[5][n])
{ std::cout << "C[5][n]\n"; }

int main()
{
	account_t arr1[5] {};
	process06(arr1); // T[5]
	
	account_t ac{ 42 };
	process06(&ac); // T[5]
	
	account_t arr2[5][3];
	process07(arr2); // C[5][n]
}
\end{cppcode}
  \item 函数指针,形式为T(*)(), C(*)(T)和T(*)(U),其中C是类类型,T和U是类型模板参数:

\begin{cppcode}
template<typename T>
void process08(T(*)()) { std::cout << "T (*)()\n"; }

template<typename T>
void process08(account_t(*)(T))
{ std::cout << "C (*) (T)\n"; }

template<typename T, typename U>
void process08(T(*)(U)) { std::cout << "T (*)(U)\n"; }

int main()
{
	account_t (*pf1)() = nullptr;
	account_t (*pf2)(int) = nullptr;
	double (*pf3)(int) = nullptr;
	
	process08(pf1); // T (*)()
	process08(pf2); // C (*)(T)
	process08(pf3); // T (*)(U)
}
\end{cppcode}
  \item 具有以下形式之一的成员函数的指针,T (C::*)(), T (C::*)(U), T (U::*)(), T (U::*)(V), C (T::*)(), C (T::*)(U)和D (C::*)(T),其中C和D是类类型,T, U和V是类型模板参数:

\begin{cppcode}
struct account_t
{
	int number;
	int get_number() { return number; }
	int from_string(std::string text) {
		return std::atoi(text.c_str()); }
};

struct transaction_t
{
	double amount;
};

struct balance_report_t {};
struct balance_t
{
	account_t account;
	double amount;
	
	account_t get_account() { return account; }
	int get_account_number() { return account.number; }
	bool can_withdraw(double const value)
		{return amount >= value; };
	transaction_t withdraw(double const value) {
		amount -= value; return transaction_t{ -value }; }
	balance_report_t make_report(int const type)
	{return {}; }
};

template<typename T>
void process09(T(account_t::*)())
{ std::cout << "T (C::*)()\n"; }

template<typename T, typename U>
void process09(T(account_t::*)(U))
{ std::cout << "T (C::*)(U)\n"; }

template<typename T, typename U>
void process09(T(U::*)())
{ std::cout << "T (U::*)()\n"; }

template<typename T, typename U, typename V>
void process09(T(U::*)(V))
{ std::cout << "T (U::*)(V)\n"; }

template<typename T>
void process09(account_t(T::*)())
{ std::cout << "C (T::*)()\n"; }

template<typename T, typename U>
void process09(transaction_t(T::*)(U))
{ std::cout << "C (T::*)(U)\n"; }

template<typename T>
void process09(balance_report_t(balance_t::*)(T))
{ std::cout << "D (C::*)(T)\n"; }

int main()
{
	int (account_t::* pfm1)() = &account_t::get_number;
	int (account_t::* pfm2)(std::string) =
		&account_t::from_string;
	int (balance_t::* pfm3)() =
		&balance_t::get_account_number;
	bool (balance_t::* pfm4)(double) =
		&balance_t::can_withdraw;
	account_t (balance_t::* pfm5)() =
		&balance_t::get_account;
	transaction_t(balance_t::* pfm6)(double) =
		&balance_t::withdraw;
	balance_report_t(balance_t::* pfm7)(int) =
		&balance_t::make_report;
		
	process09(pfm1); // T (C::*)()
	process09(pfm2); // T (C::*)(U)
	process09(pfm3); // T (U::*)()
	process09(pfm4); // T (U::*)(V)
	process09(pfm5); // C (T::*)()
	process09(pfm6); // C (T::*)(U)
	process09(pfm7); // D (C::*)(T)
}
\end{cppcode}
  \item 数据成员的指针,如T C::*、C T::*和T U::*,其中C是类类型,T和U是类型模板形参:

\begin{cppcode}
template<typename T>
void process10(T account_t::*)
{ std::cout << "T C::*\n"; }

template<typename T>
void process10(account_t T::*)
{ std::cout << "C T::*\n"; }

template<typename T, typename U>
void process10(T U::*) { std::cout << "T U::*\n"; }

int main()
{
	process10(&account_t::number); // T C::*
	process10(&balance_t::account); // C T::*
	process10(&balance_t::amount); // T U::*
}
\end{cppcode}
  \item 模板中参数列表包含至少一个类型模板形参;一般形式为C<T>,其中C是类类型,T是类型模板形参:

\begin{cppcode}
template <typename T>
struct wrapper
{
	T data;
};

template<typename T>
void process11(wrapper<T>) { std::cout << "C<T>\n"; }

int main()
{
	wrapper<double> wd{ 42.0 };
	process11(wd); // C<T>
}
\end{cppcode}
  \item 模板中参数列表包含至少一个非类型模板实参;一般形式为C<i>,其中C是类类型,i是非类型模板参数:

\begin{cppcode}
template <size_t i>
struct int_array
{
	int data[i];
};

template<size_t i>
void process12(int_array<i>) { std::cout << "C<i>\n"; }

int main()
{
	int_array<5> ia{};
	process12(ia); // C<i>
}
\end{cppcode}
  \item 双重模板参数,其实参列表包含至少一个类型模板形参;一般形式为TT<T>,其中TT是模板模板形参,T是类型模板:

\begin{cppcode}
template<template<typename> class TT, typename T>
void process13(TT<T>) { std::cout << "TT<T>\n"; }

int main()
{
	wrapper<double> wd{ 42.0 };
	process13(wd); // TT<U>
}
\end{cppcode}
  \item 双重模板参数,其实参列表包含至少一个非类型模板实参;一般形式为TT<i>,其中TT是模板模板形参,i是非类型模板实参:

\begin{cppcode}
template<template<size_t> typename TT, size_t i>
void process14(TT<i>) { std::cout << "TT<i>\n"; }
int main()
{
	int_array<5> ia{};
	process14(ia); // TT<i>
}
\end{cppcode}
  \item 双重模板参数,参数列表中没有依赖于模板形参的模板实参;形式为TT<C>,其中TT是模板模板形参,C是类类型:

\begin{cppcode}
template<template<typename> typename TT>
void process15(TT<account_t>) { std::cout << "TT<C>\n"; }

int main()
{
	wrapper<account_t> wa{ {42} };
	process15(wa); // TT<C>
}
\end{cppcode}
\end{itemize}

尽管编译器能够推导出许多类型的模板参数,但能做的事也不多:

\begin{itemize}
  \item 编译器不能从非类型模板实参的类型,推断出类型模板实参的类型。下面的例子中,process是一个有两个模板形参的函数模板:一个类型模板为T,一个类型为T的非类型模板i,调用带有五个double数组的函数不允许编译器确定T的类型,即使这是指定数组大小的值的类型:

\begin{cppcode}
template <typename T, T i>
void process(double arr[i])
{
	using index_type = T;
	std::cout << "processing " << i
	          << " doubles" << '\n';
	          
	std::cout << "index type is "
              << typeid(T).name() << '\n';
}

int main()
{
	double arr[5]{};
	process(arr); // error
	process<int, 5>(arr); // OK
}
\end{cppcode}
  \item 编译器不能根据默认值的类型确定模板参数的类型。前面的代码中用函数模板过程举例说明了这一点,该过程只有一个类型模板形参,但有两个函数形参,都是T类型,并且都有默认值。

process()调用(没有实参)失败,因为编译器不能从函数形参的默认值推断类型模板形参T的类型。process<int>()没问题,因为template参数显式提供。process(6)也没问题,因为第一个函数形参的类型可以从提供的实参中推导出来,所以类型模板参数也可以推导出来:

\begin{cppcode}
template <typename T>
void process(T a = 0, T b = 42)
{
	std::cout << a << "," << b << '\n';
}

int main()
{
	process(); // [1] error
	process<int>(); // [2] OK
	process(10); // [3] OK
}
\end{cppcode}
  \item 尽管编译器可以从函数指针或成员函数指针推导出函数模板实参,但这种功能有几个限制:不能从指向函数模板的指针推导出实参,也不能从指向函数的指针推导出实参,该函数的重载集包含多个匹配所需类型的重载函数。

前面的代码中,函数模板调用接受一个指向函数的指针,该函数有两个参数,第一个是类型模板形参T,第二个是int型,并返回void。这个函数模板不能传递指向alpha([1])的指针,这是一个函数模板,不能传递给beta([2]),并且有多个可以匹配类型T的重载。可以用指向gamma的指针调用([3]),编译器将正确地推断出第二个重载的类型:

\begin{cppcode}
template <typename T>
void invoke(void(*pfun)(T, int))
{
	pfun(T{}, 42);
}

template <typename T>
void alpha(T, int)
{ std::cout << "alpha(T,int)" << '\n'; }

void beta(int, int)
{ std::cout << "beta(int,int)" << '\n'; }

void beta(short, int)
{ std::cout << "beta(short,int)" << '\n'; }

void gamma(short, int, long long)
{ std::cout << "gamma(short,int,long long)" << '\n'; }

void gamma(double, int)
{ std::cout << "gamma(double,int)" << '\n'; }

int main()
{
	invoke(&alpha); // [1] error
	invoke(&beta); // [2] error
	invoke(&gamma); // [3] OK
}
\end{cppcode}
  \item 编译器的另一个限制是对数组的主维进行参数推断,这不是函数形参类型的一部分。这种限制的例外情况是维度引用引用或指针类型。下面的代码段演示了这些限制:

\begin{itemize}
  \item {}[1]处调用process1()会生成一个错误,因为它引用数组的主维度,所以编译器无法推断非类型模板参数Size的值。
  \item {}[2]的点上调用process2()是正确的,因为非类型模板参数Size指的是数组的第二维度。
  \item 另一方面,process3()(在[3])和process4()([4])都没问题,因为函数参数要么是一个引用,要么是指向一维数组的指针:

\begin{cppcode}
template <size_t Size>
void process1(int a[Size])
{ std::cout << "process(int[Size])" << '\n'; };

template <size_t Size>
void process2(int a[5][Size])
{ std::cout << "process(int[5][Size])" << '\n'; };

template <size_t Size>
void process3(int(&a)[Size])
{ std::cout << "process(int[Size]&)" << '\n'; };

template <size_t Size>
void process4(int(*a)[Size])
{ std::cout << "process(int[Size]*)" << '\n'; };

int main()
{
	int arr1[10];
	int arr2[5][10];
	process1(arr1); // [1] error
	process2(arr2); // [2] OK
	process3(arr1); // [3] OK
	process4(&arr1); // [4] OK
}
\end{cppcode}

\end{itemize}
  \item 若在函数模板形参列表中的表达式中使用了非类型模板实参,则编译器无法推断其值。

下面的代码中,ncube是一个类模板,其非类型模板参数N表示多个维度。函数模板process也有一个非类型的模板形参N,用在与其单个形参类型相同的模板形参列表中的表达式中,所以编译器不能从函数参数的类型推断出N的值(如[1]所示),这必须显式指定(如[2]所示):

\begin{cppcode}
template <size_t N>
struct ncube
{
	static constexpr size_t dimensions = N;
};

template <size_t N>
void process(ncube<N - 1> cube)
{
	std::cout << cube.dimensions << '\n';
}

int main()
{
	ncube<5> cube;
	process(cube); // [1] error
	process<6>(cube); // [2] OK
}
\end{cppcode}
\end{itemize}

本节中讨论的模板参数推导的所有规则也适用于可变参数函数模板,但目前讨论的所有内容都是在函数模板上下文中进行的。模板参数推断也适用于类模板,我们将在下一节探讨这个主题。













































