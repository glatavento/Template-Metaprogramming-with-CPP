\section{名称绑定和依赖名称}

“名称绑定”指的是查找模板中使用的每个名称的声明的过程。模板中使用两种名称:依赖名称和非依赖名称。前者是依赖模板参数的类型或值的名称,可以是类型参数、非类型形参或模板参数。不依赖于模板参数的名称称为非依赖名称。依赖名称和非依赖名称的查找方式不同:

\begin{itemize}
  \item 依赖名称,在模板实例化时执行。
  \item 非依赖名称,则在模板定义时执行。
\end{itemize}

首先,来看看非依赖名称,位于模板定义的前面。先来看一下下面的例子:

\begin{cppcode}
template <typename T>
struct parser; // [1] template declaration
void handle(double value) // [2] handle(double) definition
{
	std::cout << "processing a double: " << value << '\n';
}

template <typename T>
struct parser // [3] template definition
{
	void parse()
	{
		handle(42); // [4] non-dependent name
	}
};

void handle(int value) // [5] handle(int) definition
{
	std::cout << "processing an int: " << value << '\n';
}

int main()
{
	parser<int> p; // [6] template instantiation
	p.parse();
}
\end{cppcode}

注释中有几个参考点。[1]声明了一个名为parser的类模板,[2]定义一个名为handle的函数,该函数以double作为参数。类模板的定义在[3]。该类包含一个名为run的方法,该方法调用一个名为handle的函数,其参数值为42,位于[4]。

handle是一个非依赖名称,因为它不依赖于任何模板参数,所以此处执行名称查找和绑定。handle必须是[3]已知的函数,[2]上定义的函数是唯一匹配的。定义类模板之后([5]),就有了函数handle的重载定义,该函数handle以整数作为参数。这是handle(42)更好的匹配候选,但它是在执行名称绑定之后出现的,因此将被忽略。main函数中([6]),有int类型的解析器类模板的实例化。调用run函数时,会将"processing adouble: 42"输出至控制台。

下一个示例就来介绍依赖名称的概念:

\begin{cppcode}
template <typename T>
struct handler // [1] template definition
{
	void handle(T value)
	{
		std::cout << "handler<T>: " << value << '\n';
	}
};

template <typename T>
struct parser // [2] template definition
{
	void parse(T arg)
	{
		arg.handle(42); // [3] dependent name
	}
};

template <>
struct handler<int> // [4] template specialization
{
	void handle(int value)
	{
		std::cout << "handler<int>: " << value << '\n';
	}
};

int main()
{
	handler<int> h; // [5] template instantiation
	parser<handler<int>> p; // [6] template instantiation
	p.parse(h);
}
\end{cppcode}

这个示例与前面的示例略有不同。parser类模板非常相似,但handle函数已经成为另一个类模板的成员。

注释[1],有一个名为handler的类模板的定义。包含一个名为handle的公共方法,该方法接受T类型的参数,并将其值输出到控制台。接下来,[2]有parser类模板的定义。这与前一个类似,除了[3]在其参数上调用一个名为handle的方法。因为实参的类型是模板形参T,其使handle成为依赖名称。依赖名称在模板实例化时查找,因此句柄此时没有绑定。[4]有int类型的处理程序类模板的模板特化,这是与依赖名称更好的匹配。因此,当模板实例化发生在[6]时,handler<int>::handle会绑定到[3],使用的依赖名称的名称。运行此程序,控制台将会输出"handler<int>: 42"。

现在已经了解了名称绑定是如何发生的,接下来来了解它与模板实例化之间的关系。

\subsection{两阶段的名称查找}

上一节的关键内容是,名称查找对于依赖名称(依赖于模板参数的名称)和非依赖名称(不依赖于模板参数的名称,加上模板名称和当前模板实例化中定义的名称)不同。当编译器遍历模板定义时,需要判断名称是依赖的还是非依赖的,高阶名称查找依赖于这种分类,并且发生在模板定义点(对于非依赖名称)或模板实例化点(对于依赖名称)。因此,模板的实例化会分为两个阶段:

\begin{itemize}
  \item 第一个阶段发生在定义时,检查模板语法并将名称分类为依赖或非依赖。
  \item 第二个阶段发生在实例化时,此时模板实参替换为模板参数。依赖名称的绑定这时发生。
\end{itemize}

这个分为两步的过程称为\textbf{两阶段名称查找},来看一个例子:

\begin{cppcode}
template <typename T>
struct base_parser
{
	void init()
	{
		std::cout << "init\n";
	}
};

template <typename T>
struct parser : base_parser<T>
{
	void parse()
	{
		init(); // error: identifier not found
		std::cout << "parse\n";
	}
};

int main()
{
	parser<int> p;
	p.parse();
}
\end{cppcode}

代码中有两个类模板:base_parser和parser,前者包含名为init的公共方法,后者派生自base_parser,并包含一个名为parse的方法。parse成员函数调用了一个名为init的函数,目的是在这里调用的是基类方法init。然而,编译器将报错,因为它无法找到init。发生这种情况的原因是init是一个不依赖的名称(不依赖于模板参数)。因此,必须在定义解析器模板时就知道。尽管base_parser<T>::init存在,但编译器不能假定它是我们想要的,因为主模板base_parser可以稍后进行特化,而init可以定义为其他东西(例如:类型、变量或另一个函数)。因此,名称查找不会发生在基类中,而只发生在其外围作用域中,并且parser中没有名为init的函数。

这个问题可以通过将init设置为依赖名称来解决。这可以通过添加前缀this->或base_parser<T>::来实现。将init转换为依赖名称,其名称绑定将从模板定义点移动到模板实例化点。下面的代码段中,是通过this指针来解决问题的:

\begin{cppcode}
template <typename T>
struct parser : base_parser<T>
{
	void parse()
	{
		this->init(); // OK
		std::cout << "parse\n";
	}
};
\end{cppcode}

继续这个例子,在定义parser类模板之后,int类型的base_parser的特化可用。如下所示:

\begin{cppcode}
template <>
struct base_parser<int>
{
	void init()
	{
		std::cout << "specialized init\n";
	}
};
\end{cppcode}

此外,看看parser类模板的如下用法:

\begin{cppcode}
int main()
{
	parser<int> p1;
	p1.parse();
	parser<double> p2;
	p2.parse();
}
\end{cppcode}

运行这个程序时,下面的文本将输出到控制台:

\begin{cppcode}
specialized init
parse
init
parse
\end{cppcode}

出现这种行为的原因是p1是parser<int>的实例,并且其基类base_parser<int>实现了init函数,并将特化的init打印到控制台。另一方面,p2是parser<double>的实例。由于double类型的base_parser的特化不可用,因此将调用主模板中的init函数,并且只将init输出到控制台。

下一个主题就是如何使用依赖名称(即类型)。

\subsection{依赖类型的名称}

目前的例子中,依赖名称是函数或成员函数。但在某些情况下,依赖名称是类型:

\begin{cppcode}
template <typename T>
struct base_parser
{
	using value_type = T;
};

template <typename T>
struct parser : base_parser<T>
{
	void parse()
	{
		value_type v{}; // [1] error
		// or
		base_parser<T>::value_type v{}; // [2] error
		std::cout << "parse\n";
	}
};
\end{cppcode}

这个代码段中,base_parser是类模板,为T定义了名为value_type的类型别名。parser类模板派生自base_parser,需要在其parser方法中使用这种类型。然而,value_type和base_parser<T>::value_type都不起作用,编译器会报错。value_type无效,因为它是一个不依赖的名称,因此不会在基类中进行查找,只能在外围作用域中查找。base_parser<T>::value_type也不能工作,因为编译器不能假设这是一个实际类型。base_parser的特化可能紧随其后,所以value_type不一定是一种类型。

为了解决这个问题,需要告诉编译器这个名称指向的类型。否则,编译器默认会假定它不是类型。这可通过typename关键字在定义点完成,如下所示:

\begin{cppcode}
template <typename T>
struct parser : base_parser<T>
{
	void parse()
	{
		typename base_parser<T>::value_type v{}; // [3] OK
		std::cout << "parse\n";
	}
};
\end{cppcode}

实际上,这条规则有两个例外:

\begin{itemize}
  \item 指定基类时
  \item 初始化类成员时
\end{itemize}

来看看这两个例外:

\begin{cppcode}
struct dictionary_traits
{
	using key_type = int;
	using map_type = std::map<key_type, std::string>;
	static constexpr int identity = 1;
};

template <typename T>
struct dictionary : T::map_type // [1]
{
	int start_key { T::identity }; // [2]
	typename T::key_type next_key; // [3]
};

int main()
{
	dictionary<dictionary_traits> d;
}
\end{cppcode}

dictionay_traits是一个类,用作dictionary类template的模板参数。这个类派生于T::map_type(参见第[1]行),但不需要使用typename关键字。dictionary类定义了一个名为start_key的成员,是一个int型,初始化值为T::identity(参见第[2]行)。同样,这里不需要typename关键字。然而,若想定义类型T::key_type的另一个成员(见[3]行),需要使用typename。

C++20中,对使用typename的要求已经放宽了,从而更容易使用类型名。编译器现在能够推断出我们在许多上下文中引用的是类型名。例如,在[3]行上那样定义成员变量时,不再需要使用typename关键字作为前缀了。

C++20中,typename在以下情形中是隐式的(可以由编译器推导):

\begin{itemize}
  \item 使用声明时
  \item 数据成员声明中
  \item 函数参数的声明或定义中
  \item 尾部返回类型中
  \item 模板类型参数的默认类型中
  \item static_cast、const_cast、reinterpret_cast或dynamic_cast语句的type-id中
\end{itemize}

以下代码段举例说明了其中一些情况:

\begin{cppcode}
template <typename T>
struct dictionary : T::map_type
{
	int start_key{ T::identity };
	T::key_type next_key; // [1]

	using value_type = T::map_type::mapped_type; // [2]

	void add(T::key_type const&, value_type const&) {} // [3]
};
\end{cppcode}

这个代码段中,[1]、[2]和[3]标记的所有行中,C++20之前,需要typename关键字来指示类型名称(例如T::key_type或T::map_type::mapped_type)。当使用C++20编译时,这就不再需要了。

\begin{note}
第2章中,我们已经看到关键字typename和class可以用来引入类型模板参数,而且是可互换的。这里的关键字typename虽然具有类似的使用方式,但不能用class替换。
\end{note}

不仅类型可以是依赖的名称,其他模板也可以。

\subsection{依赖模板的名称}

某些情况下,依赖名称是模板,例如函数模板或类模板。但编译器的默认行为是将依赖项名称解释为非类型,这会导致使用小于比较操作符时出现错误:

\begin{cppcode}
template <typename T>
struct base_parser
{
	template <typename U>
	void init()
	{
		std::cout << "init\n";
	}
};

template <typename T>
struct parser : base_parser<T>
{
	void parse()
	{
		// base_parser<T>::init<int>(); // [1] error
		base_parser<T>::template init<int>(); // [2] OK
		std::cout << "parse\n";
	}
};
\end{cppcode}

这类似于前面的代码,base_parser中的init函数也是一个模板。尝试使用base_parser<T>::init<int>(),如[1]所示,将导会致编译器报错,所以必须使用template关键字告诉编译器依赖名称是模板,如[2]所示。

template关键字只能跟随作用域解析操作符(::)、通过指针(->)进行成员访问和成员访问(.)。正确用法的例子为X::template foo<T>(),this->template foo<T>()和obj.template foo()。

依赖名称不一定是函数模板,也可以是一个类模板:

\begin{cppcode}
template <typename T>
struct base_parser
{
	template <typename U>
	struct token {};
};

template <typename T>
struct parser : base_parser<T>
{
	void parse()
	{
		using token_type =
			base_parser<T>::template token<int>; // [1]
		token_type t1{};

		typename base_parser<T>::template token<int> t2{};
		                     // [2]
		std::cout << "parse\n";
	}
};
\end{cppcode}

token类是base_parser类模板的内部类模板,既可以在[1]行中使用,定义了类型别名(然后用于实例化对象);也可以在[2]行中使用,直接用于声明变量。typename关键字在[1]中是不必要的,其中using声明表示正在处理类型,因为编译器会假定其为非类型名称,所以在[2]中是必需的。

观察当前模板实例化的某些上下文中,并不需要使用typename和template关键字。

\subsection{实例化}

类模板定义的上下文中,可以避免使用typename和template关键字来消除依赖名称的歧义。在类模板定义的上下文中,编译器能够推导出一些依赖名称(例如嵌套类的名称)来引用当前实例化,所以一些错误可以在定义时(而不是实例化时)就可以找出来。

根据C++标准§13.8.2.1-依赖类型,可以引用当前实例化名称的完整列表如下所示:

\begin{table}[!htb]
	\centering
	\begin{talltblr} {|X[1]|X[2]|}
		内容 & 名称 \\
		类模板定义 & {嵌套类\\ 类模板的成员\\ 嵌套类的成员\\ 注入模板的类名\\ 注入的嵌套类的类名} \\
		主类模板定义或定义主类模板的成员 & 类模板的名称,后面跟着主模板的模板实参列表,其中每个实参都等效于其对应的形参 \\
		嵌套类或类模板的定义 & 用作当前实例化成员的嵌套类的名称 \\
		偏特化的定义或偏特化成员的定义 & 类模板的名称,后面跟着偏特化的模板参数列表,其中每个参数等效于其相应的形参 \\
	\end{talltblr}
  \end{table}

以下是将名称作为当前实例化的部分规则:

\begin{itemize}
  \item 当前实例化或其非依赖基类中找到的非限定名称(不在作用域解析操作符::)的右侧
  \item 限定名(出现在范围解析操作符的右侧::),若其限定符(出现在范围解析操作符左侧的部分)命名当前实例化,并且在当前实例化或其非依赖基类中找到
  \item 类成员访问表达式中使用的名称,其中对象表达式是当前实例化,且名称在当前实例化或其非依赖基类中找到
\end{itemize}

\begin{note}
若基类是依赖类型(依赖于模板形参)并且不在当前实例化中,那么基类就是依赖类。否则,基类为非依赖类。
\end{note}

这些规则听起来可能有点难以理解,以下几个例子可能有助于对其进行理解:

\begin{cppcode}
template <typename T>
struct parser
{
	parser* p1; // parser is the CI
	parser<T>* p2; // parser<T> is the CI
	::parser<T>* p3; // ::parser<T> is the CI
	parser<T*> p4; // parser<T*> is not the CI

	struct token
	{
		token* t1; // token is the CI
		parser<T>::token* t2; // parser<T>::token is the CI
		typename parser<T*>::token* t3;
		// parser<T*>::token is not the CI
	};
};

template <typename T>
struct parser<T*>
{
	parser<T*>* p1; // parser<T*> is the CI
	parser<T>* p2; // parser<T> is not the CI
};
\end{cppcode}

主模板parser中,名称parser、parser<T>和::parser<T>都指向当前实例化,但parser<T*>没有。token类是主模板parser的嵌套类。该类的范围内,token和parser<T>::token都表示当前实例化,但对于parser<T*>::token则不是这样。该代码段还包含指针类型T*的主模板的偏特化。在这个偏特化的上下文中,parser<T*>是当前实例化,而parser<T>不是。

依赖名称对于模板编程很重要。本节的关键内容是将名称分为依赖名称(依赖于模板参数的名称)和非依赖名称(不依赖于模板参数的名称)。名称绑定发生在非依赖类型的定义点和依赖类型的实例化点。某些情况下,需要关键字typename和template来消除名称使用的歧义,并告诉编译器名称指的是类型或模板。然而,在类模板定义的上下文中,编译器能够找出一些依赖名称指向当前实例化,这使它能够更快地找到错误。

下一节中,我们将把注意力转移到模板递归上。























