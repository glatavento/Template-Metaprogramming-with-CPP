\section{类模板的参数推导}
C++17之前,模板参数推断只适用于函数,而不适用于类。所以当一个类模板实例化时,必须提供所有的模板参数。下面的代码段展示了几个例子:

\begin{cppcode}
template <typename T>
struct wrapper
{
	T data;
};

std::pair<int, double> p{ 42, 42.0 };
std::vector<int> v{ 1,2,3,4,5 };
wrapper<int> w{ 42 };
\end{cppcode}

通过为函数模板利用模板实参推断,一些标准类型具有辅助函数,这些函数可以创建类型的实例,而不需要显式地指定模板实参。例如:std::make_pair用于std::pair, std::make_unique用于std::unique_ptr。这些辅助函数模板会使用auto关键字,避免了为类模板指定模板实参。这里有一个例子:

\begin{cppcode}
auto p = std::make_pair(42, 42.0);
\end{cppcode}

尽管并非所有标准类模板都有这样一个用于创建实例的辅助函数,但编写自己的帮助函数并不难。下面的代码中,可以看到一个make_vector函数模板用于创建std::vector<T>实例,还有一个make_wrapper函数模板用于创建wrapper<T>实例:

\begin{cppcode}
template <typename T, typename... Ts,
typename Allocator = std::allocator<T>>
auto make_vector(T&& first, Ts&&... args)
{
	return std::vector<std::decay_t<T>, Allocator> {
		std::forward<T>(first),
		std::forward<Ts>(args)...
	};
}

template <typename T>
constexpr wrapper<T> make_wrapper(T&& data)
{
	return wrapper{ data };
}

auto v = make_vector(1, 2, 3, 4, 5);
auto w = make_wrapper(42);
\end{cppcode}

C++17标准通过为类模板提供模板参数推导,简化了类模板的使用。因此使用C++17,本节中的第一段代码可以进行如下的简化:

\begin{cppcode}
std::pair p{ 42, 42.0 }; // std::pair<int, double>
std::vector v{ 1,2,3,4,5 }; // std::vector<int>
wrapper w{ 42 }; // wrapper<int>
\end{cppcode}

本例中,因为编译器能够从初始化式的类型推断出模板参数,编译器从变量的初始化表达式中推导出。当然,编译器也能够从new表达式和函数样式的强制转换表达式中推导出模板参数。下面举几个例子:

\begin{cppcode}
template <typename T>
struct point_t
{
	point_t(T vx, T vy) : x(vx), y(vy) {}
	private:
	T x;
	T y;
};

auto p = new point_t(1, 2); // [1] point<int>
							// new expression
							
std::mutex mt;
auto l = std::lock_guard(mt); // [2]
// std::lock_guard<std::mutex>
// function-style cast expression
\end{cppcode}

类模板的模板参数推导方式与函数模板不同,当在变量声明或函数样式强制转换中遇到类模板的名称时,编译器将构建一组推导指南。

有虚构函数代表一个虚构的类的构造函数签名模板类型,还可以提供推导参考,这些参考线将添加到编译器生成的参考列表中。若在虚构函数模板的构造集上重载解析失败(因为这些函数表示构造函数,返回类型不是匹配过程的一部分),则程序格式错误并报错。否则,所选函数模板特化的返回类型将变成推导出类模板的特化。

为了更好地理解这一点,来看看推导指南实际的样子。下面的代码中,可以看到编译器为std::pair类生成的一些指南。实际的列表更长,这里只列出了一些:

\begin{cppcode}
template <typename T1, typename T2>
std::pair<T1, T2> F();

template <typename T1, typename T2>
std::pair<T1, T2> F(T1 const& x, T2 const& y);

template <typename T1, typename T2, typename U1,
typename U2>
std::pair<T1, T2> F(U1&& x, U2&& y);
\end{cppcode}

这组隐式推导指南由类模板的构造函数生成。包括默认构造函数、复制构造函数、移动构造函数和所有转换构造函数,参数以精确的顺序复制。若构造函数是显式的,推导指南也是显式的。但若类模板没有用户定义的构造函数,则为假设的默认构造函数创建推导指南。总会为假设的复制构造函数创建推导指南。

用户定义的演绎指南可以在源代码中提供,其语法类似于带有尾部返回类型,但没有auto关键字的函数。推导指南可以是函数也可以是函数模板,但这些必须在与类模板相同的命名空间中提供。若要为std::pair类添加用户定义的推导指南,则必须在std命名空间中完成。下面是一个例子:

\begin{cppcode}
namespace std
{
	template <typename T1, typename T2>
	pair(T1&& v1, T2&& v2) -> pair<T1, T2>;
}
\end{cppcode}

目前,推导指南都针对的是函数模板,但它们不一定是函数模板,也可以是正则函数:

\begin{cppcode}
std::pair p1{1, "one"}; // std::pair<int, const char*>
std::pair p2{"two", 2}; // std::pair<const char*, int>
std::pair p3{"3", "three"};
// std::pair<const char*, const char*>
\end{cppcode}

使用编译器退化的std::pair类推导指南,推导出的类型为std::pair<int, const char*> (p1), std::pair<const char*,int> (p2), std::pair<const char*,const char*> (p3)。换句话说,编译器在使用字面值字符串的地方推导出的类型是const char*(正如预料的那样)。可以通过提供几个用户定义的推导指南,告诉编译器推导std::string,而非const char*:

\begin{cppcode}
namespace std
{
	template <typename T>
	pair(T&&, char const*) -> pair<T, std::string>;
	
	template <typename T>
	pair(char const*, T&&) -> pair<std::string, T>;
	
	pair(char const*, char const*) ->
	pair<std::string, std::string>;
}
\end{cppcode}

前两个是函数模板,但第三个是常规函数。有了这些指南,从前面的例子中为p1、p2和p3推导出的类型分别是std::pair<int, std::string>, std::pair<std::string, int>和std::pair<std::string, std::string>。

再看一个用户定义指南的例子,这次是一个用户定义的类。考虑以下对范围建模的类模板:

\begin{cppcode}
template <typename T>
struct range_t
{
	template <typename Iter>
	range_t(Iter first, Iter last)
	{
		std::copy(first, last, std::back_inserter(data));
	}
	private:
	std::vector<T> data;
};
\end{cppcode}

这个实现没有太多内容,但已经足够了。假设想从一个整数数组中构造一个range对象:

\begin{cppcode}
int arr[] = { 1,2,3,4,5 };
range_t r(std::begin(arr), std::end(arr));
\end{cppcode}

运行这段代码将报错。不同的编译器会生成不同的错误消息,也许Clang提供的错误消息最能描述问题:

\begin{shell}
error: no viable constructor or deduction guide for deduction
of template arguments of 'range_t'
range_t r(std::begin(arr), std::end(arr));
         ^
note: candidate template ignored: couldn't infer template
argument 'T'
range_t(Iter first, Iter last)
         ^
note: candidate function template not viable: requires 1
argument, but 2 were provided
struct range_t
\end{shell}

然而,不管实际的错误消息是什么,其含义都是一样的:range_t的模板参数推导失败。为了使推导工作,需要提供自定义的推导指南:

\begin{cppcode}
template <typename Iter>
range_t(Iter first, Iter last) ->
range_t<
typename std::iterator_traits<Iter>::value_type>;
\end{cppcode}

该推导指南指示的是,当遇到对带有两个迭代器实参的构造函数的调用时,模板形参T的值应该推导为迭代器特征的值类型。迭代器特征将在第5章中讨论,有了这个功能,前面的代码段运行起来就没有问题了,编译器会按预期的那样推导出r变量的类型为range_t<int>。

本节开始时,提供了以下示例,其中w的类型推断为wrapper<int>:

\begin{cppcode}
wrapper w{ 42 }; // wrapper<int>
\end{cppcode}

C++17中,若没有用户定义的推导指南,这实际上不正确。因为是wrapper<T>是一个聚合类型,并且类模板参数推导在C++17中从聚合初始化中不起作用。所以为了使上一行代码工作,需要提供如下的推导指南:

\begin{cppcode}
template <typename T>
wrapper(T) -> wrapper<T>;
\end{cppcode}

C++20中取消了对这种用户定义的推导指南的要求,标准提供了对聚合类型的支持(只要任何依赖基类都没有虚函数或虚基类,并且变量可从一个非空的初始化器列表中进行初始化)。

只有在没有提供模板参数的情况下,类模板参数推导才有效。因此,下面的p1和p2声明都是有效的,并且发生类模板参数推导。对于p2,推导出的类型是std::pair<int, std::string>(假设用户定义的指南可用)。然而,p3和p4的声明会产生错误,因为类模板参数推导没有发生。而模板参数列表是存在的(<>和<int>),但不包含所有必需的参数:

\begin{cppcode}
std::pair<int, std::string> p1{ 1, "one" }; // OK
std::pair p2{ 2, "two" }; // OK
std::pair<> p3{ 3, "three" }; // error
std::pair<int> p4{ 4, "four" }; // error
\end{cppcode}

类模板参数推断可能并不总是产生预期的结果:

\begin{cppcode}
std::vector v1{ 42 };
std::vector v2{ v1, v1 };
std::vector v3{ v1 };
\end{cppcode}

v1的推导类型是std::vector<int>,v2的推导类型是std::vector<std::vector<int>>。但编译器应该为v3类型推断出什么?有两个选项:std::vector<std::vector<int>>和std::vector<int>。若期望是前者,那么将失望地发现编译器实际上推导出后者。这是因为推导依赖于参数的数量和类型。

当参数数量大于1时,将使用接受初始化列表的构造函数。对于v2变量,即std::initializer_list<std::vector<int>>。当参数的数量为1时,则考虑参数的类型。若参数的类型是std::vector的(特化)-考虑到这种显式情况-则使用复制构造函数,推导的类型是实参的声明类型。这就是变量v3的情况,其中推导出的类型是std::vector<int>。否则,使用接受初始化列表(包含单个元素)的构造函数,例如变量v1,其推导出的类型为std::vector<int>。在cppinsights.io可视化工具的帮助下,可以显示生成的代码(对于前面的代码段)。为了简单起见,删除了allocator参数:

\begin{cppcode}
std::vector<int> v1 =
std::vector<int>{std::initializer_list<int>{42}};
std::vector<vector<int>> v2 =
std::vector<vector<int>>{
	std::initializer_list<std::vector<int>>{
		std::vector<int>(v1),
			std::vector<int>(v1)
		}
	};

std::vector<int> v3 = std::vector<int>{v1};
\end{cppcode}

类模板参数推导是C++17的一个特性,C++20改进了对聚合类型的处理。当编译器能够推导出不必要的模板参数时,器有助于避免编写不必要的显式模板参数,即使在某些情况下,编译器可能需要用户定义的推导指南才能进行推导。这还避免了创建工厂函数的需要,例如std::make_pair或std::make_tuple,这是在类模板可用之前从模板参数推导中受益的一种方式。

关于模板参数推导的内容比我们到目前为止讨论的内容要多。函数模板参数推断有一种特殊情况,称为转发引用。这个主题将在下面的章节中进行讨论。







