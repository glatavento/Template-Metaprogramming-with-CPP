\section{模板间的“友情”}
定义类时,可以使用protected和private访问说明符限制对其成员数据和成员函数的访问。若成员是private,则只能在类中访问它。若成员是protected,则可以从具有public访问权限或protected访问权限的派生类访问。但类可以通过friend关键字将其private成员或protected成员的访问权授予其他函数或类,这些授予特殊访问权限的函数或类称为友元函数或友元类。先来看一个简单的例子:

\begin{cpp}
struct wrapper
{
	wrapper(int const v) :value(v) {}
private:
	int value;
	
	friend void print(wrapper const & w);
};

void print(wrapper const& w)
{ std::cout << w.value << '\n'; }

wrapper w{ 42 };
print(w);
\end{cpp}

wrapper类有一个名为value的private数据成员。有一个名为print的独立函数,接受类型wrapper的参数,并将包装的值输出到控制台。为了能够访问它,将该函数声明为wrapper类的友元。

我们将不关注友情对非模板的作用。读者需要熟悉该特性,才能继续在模板上下文中对其进行讨论。当涉及到模板时,事情变得有点复杂。我们将通过几个例子来研究这个问题:

\begin{cpp}
struct wrapper
{
	wrapper(int const v) :value(v) {}
private:
	int value;
	
	template <typename T>
	friend void print(wrapper const&);
	
	template <typename T>
	friend struct printer;
};

template <typename T>
void print(wrapper const& w)
{ std::cout << w.value << '\n'; }

template <typename T>
struct printer
{
	void operator()(wrapper const& w)
	{ std::cout << w.value << '\n'; }
};

wrapper w{ 42 };
print<int>(w);
print<char>(w);
printer<int>()(w);
printer<double>()(w);
\end{cpp}

print函数现在是一个函数模板,有一个类型模板参数,但实际上并没有使用。可能有点奇怪,但这是一个有效的代码,需要通过指定模板参数来使用。但不管模板参数如何,print的模板实例化都可以访问wrapper类的私有成员。注意用于将其声明为wrapper类的友元的语法:使用模板语法。这同样适用于类模板printer,可声明为wrapper类的友元,任何模板实例化,无论模板参数如何,都可以访问其私有部分。

若想限制对这些模板的某些实例的访问呢?比如只有int类型的特化?然后,可以将这些特殊化声明为友元:

\begin{cpp}
struct wrapper;

template <typename T>
void print(wrapper const& w);

template <typename T>
struct printer;

struct wrapper
{
	wrapper(int const v) :value(v) {}
private:
	int value;
	
	friend void print<int>(wrapper const&);
	friend struct printer<int>;
};

template <typename T>
void print(wrapper const& w)
{ std::cout << w.value << '\n'; /* error */ }

template <>
void print<int>(wrapper const& w)
{ std::cout << w.value << '\n'; }

template <typename T>
struct printer
{
	void operator()(wrapper const& w)
	{ std::cout << w.value << '\n'; /* error*/ }
};

template <>
struct printer<int>
{
	void operator()(wrapper const& w)
	{ std::cout << w.value << '\n'; }
};

wrapper w{ 43 };
print<int>(w);
print<char>(w);
printer<int>()(w);
printer<double>()(w);
\end{cpp}

代码段中,wrapper类与前面相同。对于print函数模板和printer类模板,都有一个主模板和int类型的全特化。只有int实例化声明为wrapper类的友元。所以其他的特化,想试图访问主模板中wrapper类的私有部分时,会产生编译器错误。

这些例子中,将友情授予私有部分的类是非模板类,但类模板也可以声明友元。先从类模板和非模板函数的情况开始:

\begin{cpp}
template <typename T>
struct wrapper
{
	wrapper(T const v) :value(v) {}
private:
	T value;
	
	friend void print(wrapper<int> const&);
};

void print(wrapper<int> const& w)
{ std::cout << w.value << '\n'; }

void print(wrapper<char> const& w)
{ std::cout << w.value << '\n'; /* error */ }
\end{cpp}

此实现中,wrapper类模板声明以wrapper<int>作为参数的print重载作为友元。这个重载函数中,可以访问私有数据成员,但不能在任何其他重载中访问。当友元函数或类是模板,而只想用一个特化来访问私有部分时,也会发生类似的情况:

\begin{cpp}
template <typename T>
struct printer;

template <typename T>
struct wrapper
{
	wrapper(T const v) :value(v) {}
private:
	T value;
	
	friend void print<int>(wrapper<int> const&);
	friend struct printer<int>;
};

template <typename T>
void print(wrapper<T> const& w)
{ std::cout << w.value << '\n'; /* error */ }

template<>
void print(wrapper<int> const& w)
{ std::cout << w.value << '\n'; }

template <typename T>
struct printer
{
	void operator()(wrapper<T> const& w)
	{ std::cout << w.value << '\n'; /* error */ }
};

template <>
struct printer<int>
{
	void operator()(wrapper<int> const& w)
	{ std::cout << w.value << '\n'; }
};
\end{cpp}

wrapper类模板的这种实现,使print函数模板和printer类模板的int特化成为友元。试图访问主模板(或任何其他特化)中的私有数据成员值时,编译器将报错。

若是wrapper类模板允许友元访问print函数模板或printer类模板的实例化,代码可以这样写:

\begin{cpp}
template <typename T>
struct printer;

template <typename T>
struct wrapper
{
	wrapper(T const v) :value(v) {}
private:
	T value;
	
	template <typename U>
	friend void print(wrapper<U> const&);
	
	template <typename U>
	friend struct printer;
};

template <typename T>
void print(wrapper<T> const& w)
{ std::cout << w.value << '\n'; }

template <typename T>
struct printer
{
	void operator()(wrapper<T> const& w)
	{ std::cout << w.value << '\n'; }
};
\end{cpp}

注意声明友元时,语法是template<typename U>,而不是template<typename T>。模板参数的名称U可以是除T名称之外的名称,这将掩盖wrapper类模板的模板参数名称,这是一个错误。使用这种语法,print或printer的特化都可以访问wrapper类模板特化的私有成员。若希望只有满足包装器类template参数的友元特化才可以访问私有部分,那么必须使用以下语法:

\begin{cpp}
template <typename T>
struct wrapper
{
	wrapper(T const v) :value(v) {}
private:
	T value;
	
	friend void print<T>(wrapper<T> const&);
	friend struct printer<T>;
};
\end{cpp}

这类似于之前只授予int特化访问权限时看到的情况,只不过现在适用于匹配T的特化。

除了这些情况外,类模板还可以将友元访问权授予类型模板参数:

\begin{cpp}
template <typename T>
struct connection
{
	connection(std::string const& host, int const port)
		:ConnectionString(host + ":" + std::to_string(port))
	{}
private:
	std::string ConnectionString;
	friend T;
};

struct executor
{
	void run()
	{
		connection<executor> c("localhost", 1234);
		std::cout << c.ConnectionString << '\n';
	}
};
\end{cpp}

connection类模板有一个名为ConnectionString的私有数据成员,类型模板参数T是类的友元。executor类使用实例化connection<executor>,所以executor类型是模板参数,并受益于与connection类的友元权限,它可以访问私有数据成员ConnectionString。
 
从所有这些示例中可以看出,与模板之间的友谊与非模板实体之间的友情略有不同。朋友可以访问类的所有非公共成员,所以交朋友需要小心。另一方面,若需要将访问权限授予一些私有成员,而不是全部成员,则可以借助“客户-律师”模式。此模式允许控制对类的私有部分的访问粒度。可以在这里了解关于这个模式的信息: \url{https://en.wikibooks.org/wiki/More_C%2B%2B_Idioms/Friendship_and_the_Attorney-Client}。


