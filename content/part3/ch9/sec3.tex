\section{约束算法}
标准库提供了一百多种通用算法,它们有一个共同点:在迭代器的帮助下处理抽象范围。可以接受迭代器作为参数,有时返回迭代器。这使得重复使用标准容器或数组非常麻烦。这里有一个例子:

\begin{cpp}
auto l_odd = [](int const n) {return n % 2 == 1; };

std::vector<int> v{ 1, 1, 2, 3, 5, 8, 13 };
std::vector<int> o;
auto e1 = std::copy_if(v.begin(), v.end(),
						std::back_inserter(o),
						l_odd);
						
int arr[] = { 1, 1, 2, 3, 5, 8, 13 };
auto e2 = std::copy_if(std::begin(arr), std::end(arr),
						std::back_inserter(o),
						l_odd);
\end{cpp}

这段代码中,有一个vector v和一个array arr,我们将这两个vector中的奇数元素复制到第二个vector o中。为此,使用std::copy\_if算法,其接受开始和结束输入迭代器(定义输入范围)、第二个范围的输出迭代器(复制的元素将插入其中)和一个一元谓词(在本例中为lambda表达式),并返回的是指向最后一个复制元素之后的目标范围的迭代器。

若查看std::copy\_if算法的声明,会发现有以下两个重载:

\begin{cpp}
template <typename InputIt, typename OutputIt,
		  typename UnaryPredicate>
constexpr OutputIt copy_if(InputIt first, InputIt last,
						   OutputIt d_first,
						   UnaryPredicate pred);
						   
template <typename ExecutionPolicy,
		  typename ForwardIt1, typename ForwardIt2,
		  typename UnaryPredicate>
ForwardIt2 copy_if(ExecutionPolicy&& policy,
				   ForwardIt1 first, ForwardIt1 last,
				   ForwardIt2 d_first,
				   UnaryPredicate pred);
\end{cpp}

第一个重载就是这里使用和描述的重载,第二个重载是在C++17中引入的。允许指定执行策略,如并行或串行。这基本上支持标准算法的并行执行。但这与本章的主题无关,我们将不再深入探讨。

大多数标准算法在std::ranges命名空间中都有一个新的约束版本。这些算法存在于<algorithm>,<numeric>和<memory>头文件中,并且具有以下特征:

\begin{itemize}
  \item 与现有算法的名称相同。
  \item 具有重载,允许指定一个范围,可以使用开始迭代器和结束哨兵,也可以作为单个范围参数。
  \item 修改了返回类型,提供关于执行的更多信息。
  \item 支持应用于已处理元素的投影。投影可以是一个可调用的实体,可以是指向成员的指针、Lambda表达式或函数指针。在算法逻辑使用范围元素之前,将这样的投影应用到范围元素。
\end{itemize}

下面是如何声明std::ranges::copy\_if算法的重载:

\begin{cpp}
template <std::input_iterator I,
		  std::sentinel_for<I> S,
		  std::weakly_incrementable O,
		  class Proj = std::identity,
		  std::indirect_unary_predicate<
		    std::projected<I, Proj>> Pred>
requires std::indirectly_copyable<I, O>
constexpr copy_if_result<I, O> copy_if(I first, S last,
									   O result,
								   	   Pred pred,
									   Proj proj = {} );
									   
template <ranges::input_range R,
		  std::weakly_incrementable O,
		  class Proj = std::identity,
		  std::indirect_unary_predicate<
		  std::projected<ranges::iterator_t<R>, Proj>> Pred>
requires std::indirectly_copyable<ranges::iterator_t<R>, O>
constexpr copy_if_result<ranges::borrowed_iterator_t<R>, O>
		  copy_if(R&& r,
				  O result,
				  Pred pred,
				  Proj proj = {});
\end{cpp}

若这些看起来比较难读,因为参数太多、约束和更长的类型名。然而,好的方面是它们使代码更容易编写。下面是之前的代码片段,使用std::ranges::copy\_if:

\begin{cpp}
std::vector<int> v{ 1, 1, 2, 3, 5, 8, 13 };
std::vector<int> o;
auto e1 = std::ranges::copy_if(v, std::back_inserter(o),
								l_odd);
								
int arr[] = { 1, 1, 2, 3, 5, 8, 13 };
auto e2 = std::ranges::copy_if(arr, std::back_inserter(o),
								l_odd);

auto r = std::ranges::views::iota(1, 10);
auto e3 = std::ranges::copy_if(r, std::back_inserter(o),
								l_odd);
\end{cpp}

这些例子展示了两件事:如何从std::vector对象和数组复制元素,以及如何从视图(范围适配器)复制元素。这里没有显示的是投影,这一点在前面简要提到过。

投影是一个可调用的实体,基本上是一个功能适配器。它影响谓词,提供了一种执行函数组合的方法,并且没有提供改变算法的方法。假设有以下类型:

\begin{cpp}
struct Item
{
	int id;
	std::string name;
	double price;
};
\end{cpp}

为了便于解释,再考虑以下元素序列:

\begin{cpp}
std::vector<Item> items{
	{1, "pen", 5.49},
	{2, "ruler", 3.99},
	{3, "pensil case", 12.50}
};
\end{cpp}

投影允许对谓词执行组合。例如,想要将所有名称以字母p开头的项复制到第二个vector中。可以这样:

\begin{cpp}
std::vector<Item> copies;
std::ranges::copy_if(
	items,
	std::back_inserter(copies),
	[](Item const& i) {return i.name[0] == 'p'; });
\end{cpp}

然而,也可以写成如下等价的方式:

\begin{cpp}
std::vector<Item> copies;
std::ranges::copy_if(
	items,
	std::back_inserter(copies),
	[](std::string const& name) {return name[0] == 'p'; },
	&Item::name);
\end{cpp}

本例中,投影是指向成员的指针表达式\&Item::name,在执行谓词(Lambda表达式)之前应用于每个Item元素。当有可重用的函数对象或Lambda表达式,而不想再写一个来传递不同类型的参数时,这会很有用。

以这种方式,项目不能用于将范围从一种类型转换为另一种类型。例如,不能将Item名称从std::vector<Item>复制到std::vector<std::string>。这需要使用std::ranges::transform范围适配器,如下所示:

\begin{cpp}
std::vector<std::string> names;
std::ranges::copy_if(
	items | rv::transform(&Item::name),
	std::back_inserter(names),
	[](std::string const& name) {return name[0] == 'p'; });
\end{cpp}

有许多受约束的算法,但不会在这里列出它们。有兴趣的读者可以直接在标准中或在\url{https://en.cppreference.com/w/cpp/algorithm/ranges}页面上找到它们。

本章的最后一个主题是编写自定义范围适配器。




























