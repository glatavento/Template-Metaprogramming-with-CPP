\section{从抽象范围到范围库}
前一章中,已经多次使用范围这个术语。范围是一个元素序列的抽象,由两个迭代器分隔(一个指向序列的第一个元素,一个指向最后一个元素)。容器如std::vector、std::list和std::map都是范围抽象的具体实现,其拥有元素的所有权,并且使用各种数据结构(如数组、链表或树)实现。标准算法是通用的,可以对std::vector、std::list或std::map的内部实现一无所知,在迭代器的帮助下可以进行范围抽象,而这有一个缺点:需要从容器中检索开始和结束迭代器:

\begin{cpp}
// sorts a vector
std::vector<int> v{ 1, 5, 3, 2, 4 };
std::sort(v.begin(), v.end());

// counts even numbers in an array
std::array<int, 5> a{ 1, 5, 3, 2, 4 };
auto even = std::count_if(
	a.begin(), a.end(),
	[](int const n) {return n % 2 == 0; });
\end{cpp}

极少数情况下,只需要处理容器元素的一部分。大多数情况下,需要一遍遍地写v.begin()和v.end(),包括对cbegin()/cend()、rbegin()/rend()或独立函数std::begin()/std::end()的调用。理想情况下,我们希望能够编写如下代码:

\begin{cpp}
// sorts a vector
std::vector<int> v{ 1, 5, 3, 2, 4 };
sort(v);

// counts even numbers in an array
std::array<int, 5> a{ 1, 5, 3, 2, 4 };
auto even = std::count_if(
	a,
	[](int const n) {return n % 2 == 0; });
\end{cpp}

另一方面,我们经常需要组合操作。即使使用标准算法,也会涉及许多操作和过于冗长的代码。考虑下面的例子:给定一个整数序列,希望将除前两个以外的所有偶数的平方按它们的值降序(而不是在序列中的位置)输出到控制台。有多种方法可以解决这个问题,以下是一个可能的解决方案:

\begin{cpp}
std::vector<int> v{ 1, 5, 3, 2, 8, 7, 6, 4 };

// copy only the even elements
std::vector<int> temp;
std::copy_if(v.begin(), v.end(),
			std::back_inserter(temp),
			[](int const n) {return n % 2 == 0; });
			
// sort the sequence
std::sort(temp.begin(), temp.end(),
		[](int const a, int const b) {return a > b; });
		
// remove the first two
temp.erase(temp.begin() + temp.size() - 2, temp.end());

// transform the elements
std::transform(temp.begin(), temp.end(),
				temp.begin(),
				[](int const n) {return n * n; });

// print each element
std::for_each(temp.begin(), temp.end(),
				[](int const n) {std::cout << n << '\n'; });
\end{cpp}

熟悉标准算法的开发者可以轻松地阅读这段代码,但仍然有很多东西需要写,还需要一个临时容器和重复的开始/结束调用。因此,我想大多数人会更容易理解以下版本的代码,并且喜欢这样进行书写:

\begin{cpp}
std::vector<int> v{ 1, 5, 3, 2, 8, 7, 6, 4 };
sort(v);
auto r = v
		| filter([](int const n) {return n % 2 == 0; })
		| drop(2)
		| reverse
		| transform([](int const n) {return n * n; });
		
for_each(r, [](int const n) {std::cout << n << '\n'; });
\end{cpp}

这是C++20标准在范围库的帮助下提供的功能。这有两个主要组成部分:

\begin{itemize}
\item
视图或范围适配器,表示非拥有的可迭代序列。使我们能够更容易地组合操作,例如在最后一个示例中。

\item
约束算法,能够操作具体的范围(标准容器或范围),不是使用一对迭代器分隔的抽象范围(尽管这也是可能的)。
\end{itemize}

下一节中,我们将探讨范围库的这两种功能,先从范围开始。

