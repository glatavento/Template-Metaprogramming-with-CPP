\section{范围和视图}

术语范围是指,由开始迭代器和结束迭代器限定的元素序列的抽象,所以范围表示元素的可迭代序列。这样的序列可以用几种方式定义:

\begin{itemize}
\item
一个开始迭代器和一个结束哨兵,这样的序列从开始迭代到结束。哨兵是表示序列结束的对象。可以具有与迭代器类型相同的类型,也可以具有不同的类型。

\item
一个开始对象和一个大小(元素的数量),表示一个所谓的计数序列。这样的序列从一开始就要迭代N次(其中N表示大小)。

\item
有一个开始和一个谓词,表示所谓的有条件终止序列。这样的序列从开始迭代,直到谓词返回false为止。

\item
只有一个开始值,表示所谓的无界序列。这样的序列可以无限迭代。
\end{itemize}

所有这些类型的可迭代序列都是范围。因为范围是一种抽象,C++20标准库定义了一系列概念来描述范围类型的需求,可以在<ranges>头文件和std::ranges命名空间中使用。下表列出了范围概念的列表:

\begin{table}[H]
\centering
	\begin{tabular}{|l|l|}
		\hline
		\textbf{名称} &
		\textbf{描述} \\ \hline
		range &
		\begin{tabular}[c]{@{}l@{}}通过提供开始迭代器和结束哨兵,定义类型R为范围的需求。\\ 迭代器和哨兵可以是不同的类型。\end{tabular} \\ \hline
		bgorrowed\_range &
		\begin{tabular}[c]{@{}l@{}}定义R类型的要求,以便函数可以按值接受该类型的对象\\ 并返回从该对象获得的迭代器,而不会出现悬空的危险。\end{tabular} \\ \hline
		sized\_range &
		\begin{tabular}[c]{@{}l@{}}将类型R的要求定义为在常数时间内知道其大小的范围。\end{tabular} \\ \hline
		common\_range &
		\begin{tabular}[c]{@{}l@{}}将类型R的要求定义为迭代器类型和哨兵类型相同的范围。\end{tabular} \\ \hline
		view &
		\begin{tabular}[c]{@{}l@{}}定义类型R的需求,该类型R是一个具有固定时间\\ 复制、移动和赋值操作的范围。\end{tabular} \\ \hline
		viewable\_range &
		\begin{tabular}[c]{@{}l@{}}定义可转换为视图的范围类型R的需求。\end{tabular} \\ \hline
		input\_range &
		\begin{tabular}[c]{@{}l@{}}要求范围类型具有满足input\_iterator概念的迭代器类型。\end{tabular} \\ \hline
		output\_range &
		\begin{tabular}[c]{@{}l@{}}要求范围类型具有满足output\_iterator概念的迭代器类型。\end{tabular} \\ \hline
		forward\_range &
		\begin{tabular}[c]{@{}l@{}}要求范围类型具有满足forward\_iterator概念的迭代器类型。\end{tabular} \\ \hline
		\begin{tabular}[c]{@{}l@{}}bidirectional\_\\ range\end{tabular} &
		\begin{tabular}[c]{@{}l@{}}要求范围类型具有满足bidirectional\_iterator概念的迭代器类型。\end{tabular} \\ \hline
		\begin{tabular}[c]{@{}l@{}}random\_access\_\\ range\end{tabular} &
		\begin{tabular}[c]{@{}l@{}}要求范围类型具有满足random\_access\_iterator概念的迭代器类型。\end{tabular} \\ \hline
		contiguous\_range &
		\begin{tabular}[c]{@{}l@{}}要求范围类型具有满足contiguous\_iterator概念的迭代器类型。\end{tabular} \\ \hline
	\end{tabular}
\end{table}

\begin{center}
表 9.1
\end{center}

标准库为容器和数组定义了一组访问函数。这包括std::begin和std::end代替成员函数begin和end, std::size代替成员函数size等,这些称为范围访问函数。类似地,范围库定义了一组范围访问函数。其为范围设计的,可以在<ranges>和<iterator>头文件和std::ranges命名空间中使用。下表列出了它们:

\begin{table}[H]
\centering
	\begin{tabular}{|l|l|l|}
		\hline
		\textbf{Rnage的访问范围} &
		\textbf{\begin{tabular}[c]{@{}l@{}}容器/数组的\\ 等效范围访问\end{tabular}} &
		\textbf{描述} \\ \hline
		\begin{tabular}[c]{@{}l@{}}begin/end\\ cbegin/cend\end{tabular} &
		\begin{tabular}[c]{@{}l@{}}begin/end\\ cbegin/cend\end{tabular} &
		\begin{tabular}[c]{@{}l@{}}返回一个迭代器和一个常量迭代器,分别指向\\ 范围的开始/结束。\end{tabular} \\ \hline
		\begin{tabular}[c]{@{}l@{}}rbegin/rend\\ crbegin/crend\end{tabular} &
		\begin{tabular}[c]{@{}l@{}}rbegin/rend\\ crbegin/crend\end{tabular} &
		\begin{tabular}[c]{@{}l@{}}分别返回一个指向范围开始/结束的反向迭代器\\ 和常量反向迭代器。\end{tabular} \\ \hline
		size/ssize &
		size/ssize &
		\begin{tabular}[c]{@{}l@{}}将范围的大小返回为整数或带符号的整数值。\end{tabular} \\ \hline
		empty &
		empty &
		\begin{tabular}[c]{@{}l@{}}返回一个布尔值,指示范围是否为空。\end{tabular} \\ \hline
		data/cdata &
		data &
		\begin{tabular}[c]{@{}l@{}}分别返回指向连续范围和只读连续范围开头的\\ 指针。\end{tabular} \\ \hline
	\end{tabular}
\end{table}

\begin{center}
表 9.2
\end{center}

下面的代码段演示了其中一些函数的使用:

\begin{cpp}
std::vector<int> v{ 8, 5, 3, 2, 4, 7, 6, 1 };
auto r = std::views::iota(1, 10);

std::cout << "size(v)=" << std::ranges::size(v) << '\n';
std::cout << "size(r)=" << std::ranges::size(r) << '\n';

std::cout << "empty(v)=" << std::ranges::empty(v) << '\n';
std::cout << "empty(r)=" << std::ranges::empty(r) << '\n';

std::cout << "first(v)=" << *std::ranges::begin(v) << '\n';
std::cout << "first(r)=" << *std::ranges::begin(r) << '\n';

std::cout << "rbegin(v)=" << *std::ranges::rbegin(v)
		  << '\n';
std::cout << "rbegin(r)=" << *std::ranges::rbegin(r)
		  << '\n';
		  
std::cout << "data(v)=" << *std::ranges::data(v) << '\n';
\end{cpp}

这段代码中,使用了一个名为std::views::iota的类型。正如命名空间所示,这是一个视图。视图是带有限制的范围,视图是具有非所有语义的轻量级对象,以一种不需要复制或改变序列的方式呈现底层元素序列(范围)的视图。关键特征是惰性求值。所以不管它们应用了什么转换,只在请求(迭代)元素时执行,而不是在创建时执行。

C++20中提供了一系列视图,C++23中也包含了新视图。视图可以在<ranges>头文件和std::ranges命名空间的形式中使用,std::ranges::abc\_view,例如std::ranges::iota\_view。为了使用方便,在std::views命名空间中,还存在一个形式为std::views::abc的变量模板,例如std::views::iota。这就是我们在前面的例子中看到的。下面是两个使用iota的等效示例:

\begin{cpp}
// using the iota_view type
for (auto i : std::ranges::iota_view(1, 10))
	std::cout << i << '\n';

// using the iota variable template
for (auto i : std::views::iota(1, 10))
	std::cout << i << '\n';
\end{cpp}

iota视图是工厂视图的一部分。这些工厂视图是新生成范围的视图,范围库中有以下工厂视图:

\begin{table}[H]
\centering
	\begin{tabular}{|l|l|l|}
		\hline
		\textbf{类型} &
		\textbf{变量} &
		\textbf{描述} \\ \hline
		\begin{tabular}[c]{@{}l@{}}ranges::empty\_\\ view\end{tabular} &
		ranges::views::empty &
		\begin{tabular}[c]{@{}l@{}}生成不包含T类型元素的视图。\end{tabular} \\ \hline
		\begin{tabular}[c]{@{}l@{}}ranges::single\_\\ view\end{tabular} &
		ranges::views::single &
		\begin{tabular}[c]{@{}l@{}}生成具有单个T类型元素的视图。\end{tabular} \\ \hline
		\begin{tabular}[c]{@{}l@{}}ranges::iota\_\\ view\end{tabular} &
		ranges::views::iota &
		\begin{tabular}[c]{@{}l@{}}生成一个连续元素序列的视图,从\\开始值到结束值(有界视图)或无限\\ (无界视图)。\end{tabular} \\ \hline
		\begin{tabular}[c]{@{}l@{}}ranges::basic\_\\ iostream\_view\end{tabular} &
		ranges::views::istream &
		\begin{tabular}[c]{@{}l@{}}通过反复应用操作符>{}>生成一个元\\ 素序列的视图。\end{tabular} \\ \hline
	\end{tabular}
\end{table}

\begin{center}
表 9.3
\end{center}

为什么empty\_view和single\_view有用,答案应该不难找到。模板代码中,可以处理空范围或只有一个元素的范围是有效输入的范围。一个函数模板不用多次重载来处理这些特殊情况;反之,可以接受一个empty\_view或single\_view范围。下面的代码使用工厂视图的几个示例:

\begin{cpp}
constexpr std::ranges::empty_view<int> ev;
static_assert(std::ranges::empty(ev));
static_assert(std::ranges::size(ev) == 0);
static_assert(std::ranges::data(ev) == nullptr);

constexpr std::ranges::single_view<int> sv{42};
static_assert(!std::ranges::empty(sv));
static_assert(std::ranges::size(sv) == 1);
static_assert(*std::ranges::data(sv) == 42);
\end{cpp}

对于iota\_view,已经看过一些有界视图的例子。下面的代码段再次展示了一个例子,不仅使用了iota生成的有界视图,还使用了同样由iota生成的无界视图:

\begin{cpp}
auto v1 = std::ranges::views::iota(1, 10);
std::ranges::for_each(
	v1,
	[](int const n) {std::cout << n << '\n'; });
	
auto v2 = std::ranges::views::iota(1) |
		  std::ranges::views::take(9);
std::ranges::for_each(
	v2,
	[](int const n) {std::cout << n << '\n'; });
\end{cpp}

最后一个示例中,使用了另一个名为take\_view的视图。这将生成另一个视图(我们的示例中是iota生成的无界视图)的前N个元素(我们的示例中是9)的视图。首先,举一个使用第四个视图工厂basic\_iostream\_view的例子。假设在文本中有一个商品价格列表,用空格分隔,需要输出这些价格的总和。有不同的解决方法,但这里给出了一个可能的解决方案:

\begin{cpp}
auto text = "19.99 7.50 49.19 20 12.34";
auto stream = std::istringstream{ text };
std::vector<double> prices;

double price; // 高亮显示
while (stream >> price) // 高亮显示
{ // 高亮显示
	prices.push_back(price); // 高亮显示
} // 高亮显示

auto total = std::accumulate(prices.begin(), prices.end(),
							 0.0);
std::cout << std::format("total: {}\n", total);
\end{cpp}

高亮显示的部分可以用以下两行代码替换,使用basic\_iostream\_view,或者更准确地说,使用istream\_view别名模板:

\begin{cpp}
for (double const price :
		std::ranges::istream_view<double>(stream))
{
	prices.push_back(price);
}
\end{cpp}

istream\_view范围工厂所做的是在istringstream对象上重复应用操作符>{}>,并在每次应用时产生一个值。不能指定分隔符,只适用于空格。若喜欢使用标准算法,可以使用range::for\_each约束算法来产生相同的结果:

\begin{cpp}
std::ranges::for_each(
	std::ranges::istream_view<double>(stream),
	[&prices](double const price) {
		prices.push_back(price); });
\end{cpp}

目前为止,本章给出的示例包括filter, take, drop和reverse等视图。这些只是C++20中可用的一些标准视图,C++23中会添加了更多。下表列出了整个标准视图集:

\begin{table}[H]
\centering
	\begin{tabular}{|l|l|l|l|}
		\hline
		\textbf{\begin{tabular}[c]{@{}l@{}}类型\\ (ranges\\ 命名空间中)\end{tabular}} &
		\textbf{\begin{tabular}[c]{@{}l@{}}变量\\ (ranges::view\\ 命名空间中)\end{tabular}} &
		\textbf{\begin{tabular}[c]{@{}l@{}}C++版本\end{tabular}} &
		\textbf{描述} \\ \hline
		fileter\_view &
		fileter &
		C++20 &
		\begin{tabular}[c]{@{}l@{}}表示范围适配器的类型,该范围适配器提供下\\ 级范围的视图,该范围仅包括满足谓词的元素。\end{tabular} \\ \hline
		\begin{tabular}[c]{@{}l@{}}transform\_\\ view\end{tabular} &
		transform &
		C++20 &
		\begin{tabular}[c]{@{}l@{}}一种类型,表示提供底层范围视图的范围适配\\ 器,并将转换应用到范围的每个元素。\end{tabular} \\ \hline
		split\_view &
		split &
		C++20 &
		\begin{tabular}[c]{@{}l@{}}表示范围适配器的类型,该类型提供通过在指\\ 定分隔符上分割范围而产生的范围序列的视图。\\ 该范围不能是输入范围,并且视图的惰性语义\\ 不会被观察到。\end{tabular} \\ \hline
		\begin{tabular}[c]{@{}l@{}}lazy\_split\_\\ view\end{tabular} &
		lazy\_split &
		C++20 &
		\begin{tabular}[c]{@{}l@{}}与split\_view相同,还适用于输入范围,并观\\ 察范围的惰性机制。\end{tabular} \\ \hline
		reverse\_view &
		reverse &
		C++20 &
		\begin{tabular}[c]{@{}l@{}}一种类型,表示范围适配器,该适配器以相反\\ 的顺序提供基础范围元素的视图。\end{tabular} \\ \hline
		keys\_view &
		keys &
		C++20 &
		\begin{tabular}[c]{@{}l@{}}表示范围适配器的类型,提供从底层视图的类\\ 元组值(std::pari和std::tuple)的第一个元素投\\ 影的视图。\end{tabular} \\ \hline
		values\_view &
		values &
		C++20 &
		\begin{tabular}[c]{@{}l@{}}表示范围适配器的类型,提供从底层视图的类\\ 元组值(std::pair和std::tuple)的第二个元素投\\ 影出来的视图。\end{tabular} \\ \hline
		elements\_view &
		elements &
		C++20 &
		\begin{tabular}[c]{@{}l@{}}表示范围适配器的类型,该范围适配器提供从\\ 非视图类元组值的第n个元素投影的视图。\end{tabular} \\ \hline
		zip\_view &
		zip &
		C++23 &
		\begin{tabular}[c]{@{}l@{}}一种表示范围适配器的类型,提供一个由一个\\ 或多个底层视图构建的视图,将每个视图的第\\ n个元素投影到一个元组中。\end{tabular} \\ \hline
		\begin{tabular}[c]{@{}l@{}}zip\_\\ transform\_\\ view\end{tabular} &
		\begin{tabular}[c]{@{}l@{}}zip\_\\ transform\end{tabular} &
		C++23 &
		\begin{tabular}[c]{@{}l@{}}一种类型,表示一个范围适配器,该适配器提\\ 供一个由一个或多个底层视图和一个可调用对\\ 象构建的视图,其元素是通过将可调用对象应\\ 用于每个底层视图的第n个元素来计算的。\end{tabular} \\ \hline
		\begin{tabular}[c]{@{}l@{}}adjacent\_\\ view\end{tabular} &
		adjacent &
		C++23 &
		\begin{tabular}[c]{@{}l@{}}一种表示范围适配器的类型,该范围适配器提\\ 供类似元组值的视图,通过获取底层视图的N\\ 个连续元素进行投影。\end{tabular} \\ \hline
		\begin{tabular}[c]{@{}l@{}}adjacent\_\\ transform\_\\ view\end{tabular} &
		\begin{tabular}[c]{@{}l@{}}adjacent\_\\ transform\end{tabular} &
		C++23 &
		\begin{tabular}[c]{@{}l@{}}表示范围适配器的类型,该类型通过将可调用\\ 对象应用到底层视图的N个连续元素来提供投\\ 影值的视图。\end{tabular} \\ \hline
	\end{tabular}
\end{table}

\begin{center}
表 9.4
\end{center}

除了上表中列出的视图(范围适配器)之外,还有一些在某些特定场景中可能有用的视图。为了完整起见,下表将列出了这些参数:

\begin{table}[H]
\centering
	\begin{tabular}{|l|l|l|l|}
		\hline
		\textbf{\begin{tabular}[c]{@{}l@{}}类型\\ (ranges\\ 命名空间中)\end{tabular}} &
		\textbf{\begin{tabular}[c]{@{}l@{}}变量\\ (ranges::view\\ 命名空间中)\end{tabular}} &
		\textbf{\begin{tabular}[c]{@{}l@{}}C++版本\end{tabular}} &
		\textbf{描述} \\ \hline
		&
		all &
		C++20 &
		\begin{tabular}[c]{@{}l@{}}一个对象,创建一个包含范围参数的所有元素\\ 的视图。\end{tabular} \\ \hline
		&
		all\_t &
		C++20 &
		\begin{tabular}[c]{@{}l@{}}可以安全转换为视图的范围的视图类型的别名\\ 模板。\end{tabular} \\ \hline
		&
		counted &
		C++20 &
		\begin{tabular}[c]{@{}l@{}}一个对象,创建一个包含N个范围元素的视\\ 图,从给定迭代器所表示的元素开始。\end{tabular} \\ \hline
		ref\_view &
		&
		C++20 &
		\begin{tabular}[c]{@{}l@{}}将引用封装到另一个范围的视图类型。\end{tabular} \\ \hline
		owning\_view &
		&
		C++20 &
		\begin{tabular}[c]{@{}l@{}}存储给定范围的视图类型,拥有存储范围和\\ 移动语义的唯一所有权。\end{tabular} \\ \hline
		common\_view &
		common &
		C++20 &
		\begin{tabular}[c]{@{}l@{}}一种对迭代器和哨兵类型对采用不同类型的\\ 视图的类型,对迭代器和哨兵类型使用相同\\ 类型的视图。\end{tabular} \\ \hline
	\end{tabular}
\end{table}

\begin{center}
表 9.5
\end{center}

我们已经列举了所有标准范围适配器,来看一下使用适配器的更多示例。

\subsubsection{9.2.1\hspace{0.2cm}更多的例子}

本节之前,我们看到了以下示例(使用显式命名空间):

\begin{cpp}
namespace rv = std::ranges::views;
std::ranges::sort(v);
auto r = v
		| rv::filter([](int const n) {return n % 2 == 0; })
		| rv::drop(2)
		| rv::reverse
		| rv::transform([](int const n) {return n * n; });
\end{cpp}

这实际上是以下内容的短而易读版本:

\begin{cpp}
std::ranges::sort(v);auto r =
	rv::transform(
		rv::reverse(
			rv::drop(
				rv::filter(
					v,
					[](int const n) {return n % 2 == 0; }),
					2)),
		[](int const n) {return n * n; });
\end{cpp}

第一个版本是可能的,因为管道操作符(|)可重载,以更易于阅读的形式简化视图的组合。有些范围适配器接受一个参数,有些可能接受多个参数。适用规则如下:

\begin{itemize}
\item
若范围适配器a有一个参数,一个视图V,那么a(V)和v| a是等价的。这样的范围适配器是reverse\_view:

\begin{cpp}
std::vector<int> v{ 1, 5, 3, 2, 8, 7, 6, 4 };
namespace rv = std::ranges::views;
auto r1 = rv::reverse(v);
auto r2 = v | rv::reverse;
\end{cpp}

\item
若一个范围适配器a有多个参数,一个视图V和args…,那么a (V, args…),a (args…)(V)和V| a (args…)是等价的。这样的范围适配器是take\_view:

\begin{cpp}
std::vector<int> v{ 1, 5, 3, 2, 8, 7, 6, 4 };
namespace rv = std::ranges::views;
auto r1 = rv::take(v, 2);
auto r2 = rv::take(2)(v);
auto r3 = v | rv::take(2);
\end{cpp}

\end{itemize}

已经看过滤器、变换、反转和删除的例子。下面,我们会通过一系列例子来演示表8.7中的视图的使用。以下所有示例中,我们将把rv视为std::ranges::views命名空间的别名:

\begin{itemize}
\item
输出序列中最后两个奇数,顺序相反:

\begin{cpp}
std::vector<int> v{ 1, 5, 3, 2, 4, 7, 6, 8 };
for (auto i : v |
	rv::reverse |
	rv::filter([](int const n) {return n % 2 == 1; }) |
	rv::take(2))
{
	std::cout << i << '\n'; // prints 7 and 3
}
\end{cpp}

\item
从不包括第一个连续奇数的范围中打印小于10的连续数的子序列:

\begin{cpp}
std::vector<int> v{ 1, 5, 3, 2, 4, 7, 16, 8 };
for (auto i : v |
	rv::take_while([](int const n){return n < 10; }) |
	rv::drop_while([](int const n){return n % 2 == 1; })
) {
	std::cout << i << '\n'; // prints 2 4 7
}
\end{cpp}

\item
分别输出元组序列中的第一个元素、第二个元素和第三个元素:

\begin{cpp}
std::vector<std::tuple<int,double,std::string>> v =
{
	{1, 1.1, "one"},
	{2, 2.2, "two"},
	{3, 3.3, "three"}
};

for (auto i : v | rv::keys)
	std::cout << i << '\n'; // prints 1 2 3
	
for (auto i : v | rv::values)
	std::cout << i << '\n'; // prints 1.1 2.2 3.3
	
for (auto i : v | rv::elements<2>)
	std::cout << i << '\n'; // prints one two three
\end{cpp}

\item
输出一个由整数vector中的所有元素:

\begin{cpp}
std::vector<std::vector<int>> v {
	{1,2,3}, {4}, {5, 6}
};

for (int const i : v | rv::join)
	std::cout << i << ' '; // prints 1 2 3 4 5 6
\end{cpp}

\item
输出由整数vector中的所有元素,但在每个vector的元素之间插入一个0。范围适配器join\_with是C++23的新功能,编译器可能还不支持:

\begin{cpp}
std::vector<std::vector<int>> v{
	{1,2,3}, {4}, {5, 6}
};
for(int const i : v | rv::join_with(0))
	std::cout << i << ' '; // print 1 2 3 0 4 0 5 6
\end{cpp}

\item
输出句子中的单个单词,其中分隔符为空格:

\begin{cpp}
std::string text{ "this is a demo!" };
constexpr std::string_view delim{ " " };
for (auto const word : text | rv::split(delim))
{
	std::cout << std::string_view(word.begin(),
								  word.end())
			  << '\n';
}
\end{cpp}

\item
从一个整数数组的元素和一个双精度vector创建元组视图:

\begin{cpp}
std::array<int, 4> a {1, 2, 3, 4};
std::vector<double> v {10.0, 20.0, 30.0};

auto z = rv::zip(a, v)
// { {1, 10.0}, {2, 20.0}, {3, 30.0} }
\end{cpp}

\item
创建包含整数数组的相乘元素和双精度vector的视图:

\begin{cpp}
std::array<int, 4> a {1, 2, 3, 4};
std::vector<double> v {10.0, 20.0, 30.0};

auto z = rv::zip_transform(
	std::multiplies<double>(), a, v)
// { {1, 10.0}, {2, 20.0}, {3, 30.0} }
\end{cpp}

\item
输出整数序列的相邻元素对:

\begin{cpp}
std::vector<int> v {1, 2, 3, 4};
for (auto i : v | rv::adjacent<2>)
{
	// prints: (1, 2) (2, 3) (3, 4)
	std::cout << std::format("({},{})",
							 i.first, i.second)";
}
\end{cpp}

\item
输出从一个整数序列中每三个连续值相乘得到的值:

\begin{cpp}
std::vector<int> v {1, 2, 3, 4, 5};
for (auto i : v | rv::adjacent_transform<3>(
	std::multiplies()))
{
	std::cout << i << ' '; // prints: 3 24 60
}
\end{cpp}

\end{itemize}

这些示例希望能够帮助理解每个可用视图的可能用例,可以在本书附带的源代码中找到更多示例,也可以在扩展阅读部分找到更多示例。下一节中,我们将讨论范围库的另一部分,约束算法。












