\section{奇异递归模板模式(CRTP)}

这个模式有一个相当奇怪的名字:奇异递归模板模式,简称CRTP。之所以称为"奇异",是因为奇怪和不直观。1995年,James Coplien在c++ Report的一个专栏中首次描述了这个模式(并创造了它的名字)。这种模式如下:

\begin{itemize}
\item
定义(静态)接口的基类模板。

\item
派生类本身就是基类模板的模板参数。

\item
基类的成员函数调用其类型模板参数(即派生类)的成员函数。
\end{itemize}

来看看模式实现在实际中是什么样子的。我们将把前面带有游戏单位的例子转换成使用CRTP的版本,模式实现如下所示:

\begin{cpp}
template <typename T>
struct game_unit
{
	void attack()
	{
		static_cast<T*>(this)->do_attack();
	}
};

struct knight : game_unit<knight>
{
	void do_attack()
	{ std::cout << "draw sword\n"; }
};

struct mage : game_unit<mage>
{
	void do_attack()
	{ std::cout << "spell magic curse\n"; }
};
\end{cpp}

game\_unit类现在是一个模板类,但包含相同的成员函数attack。在内部,它将this指针上转换为T*,然后调用一个名为do\_attack的成员函数。knight和mage类派生自game\_unit类,并将自己作为类型模板参数T进行传递。它们都提供了一个名为do\_attack的成员函数。

注意,基类模板中的成员函数和派生类中调用的成员函数具有不同的名称。否则,若具有相同的名称,因为它们不再是虚函数了,派生类成员函数将隐藏基类成员,、。

fight函数(获取游戏单位集合并调用攻击函数)也需要改变。需要作为函数模板实现,实现如下所示:

\begin{cpp}
template <typename T>
void fight(std::vector<game_unit<T>*> const & units)
{
	for (auto unit : units)
	{
		unit->attack();
	}
}
\end{cpp}

使用此函数与以前略有不同:

\begin{cpp}
knight k;
mage m;
fight<knight>({ &k });
fight<mage>({ &m });
\end{cpp}

我们已经将运行时多态性移到了编译时,所以fight函数不能多态地处理knight和mage的对象。相反,我们得到了两个不同的重载,一个可以处理knight对象,一个可以处理mage对象,这就是静态多态。

尽管这个模式看起来并不复杂,但此需要问问自己:这个模式实际上有什么用处?可以使用CRT解决的问题,包括:

\begin{itemize}
\item
限制类型可以实例化的次数

\item
增加公共功能并避免代码重复

\item
实现复合设计模式
\end{itemize}

下面的小节中,我们将逐一研究这些问题,并了解如何使用CRTP解决。

\subsubsection{7.2.1\hspace{0.2cm}限制实例化对象的次数}

假设在创造骑士和法师的游戏中,需要一些道具在有限的实例中可用。例如,有一种特殊的剑叫做Excalibur,它应该只有一个实例。另一方面,有一本魔法咒语书,但在游戏中一次不能超过三个实例。怎么解决这个问题?显然,剑的问题可以用单例模式解决。但当我们需要把这个数限制到一个更高的值,但仍然是有限的时候,该怎么办呢?单例模式不会有什么帮助(除非将其转换为“多例”),但CRTP可以。

首先,从基类模板开始。这个类模板所做的事就是记录实例化的次数。计数器是静态数据成员,在构造函数中自增,在析构函数中自减。当该计数超过定义的限制时,将触发异常:

\begin{cpp}
template <typename T, size_t N>
struct limited_instances
{
	static std::atomic<size_t> count;
	limited_instances()
	{
		if (count >= N)
			throw std::logic_error{ "Too many instances" };
		++count;
	}
	~limited_instances() { --count; }
};

template <typename T, size_t N>
std::atomic<size_t> limited_instances<T, N>::count = 0;
\end{cpp}

模板的第二部分包括定义派生类。针对上述问题,具体实现如下:

\begin{cpp}
struct excalibur : limited_instances<excalibur, 1>
{};

struct book_of_magic : limited_instances<book_of_magic, 3>
{};
\end{cpp}

我们可以实例化excalibur一次。当第二次尝试做同样的事情时(当第一个实例仍然存在时),将抛出异常:

\begin{cpp}
excalibur e1;
try
{
	excalibur e2;
}
catch (std::exception& e)
{
	std::cout << e.what() << '\n';
}
\end{cpp}

类似地,可以实例化book\_of\_magic三次,第四次尝试这样做时将抛出异常:

\begin{cpp}
book_of_magic b1;
book_of_magic b2;
book_of_magic b3;
try
{
	book_of_magic b4;
}
catch (std::exception& e)
{
	std::cout << e.what() << '\n';
}
\end{cpp}

接下来,来看一个更常见的场景,向类型添加通用功能。

\subsubsection{7.2.2\hspace{0.2cm}增加功能}

奇异递归模板模式可以解决的另一种情况是,通过基类中仅依赖于派生类成员的泛型函数为派生类提供公共功能。我们通过一个例子来理解这个用例。

假设一些游戏单位拥有step\_forth和step\_back等成员函数,它们将向前或向后移动一个位置。这些类看起来如下所示:

\begin{cpp}
struct knight
{
	void step_forth();
	void step_back();
};

struct mage
{
	void step_forth();
	void step_back();
};
\end{cpp}

然而,这可能是一种要求,即所有可以来回移动一步的东西都能够前进或后退任意数量的步骤。但这个功能可以基于step\_forward和step\_back函数来实现,这将有助于避免在每个游戏单元类中出现重复的代码,所以这个问题的CRTP实现如下所示:

\begin{cpp}
template <typename T>
struct movable_unit
{
	void advance(size_t steps)
	{
		while (steps--)
			static_cast<T*>(this)->step_forth();
	}

	void retreat(size_t steps)
	{
		while (steps--)
			static_cast<T*>(this)->step_back();
	}
};

struct knight : movable_unit<knight>
{
	void step_forth()
	{ std::cout << "knight moves forward\n"; }
	
	void step_back()
	{ std::cout << "knight moves back\n"; }
};

struct mage : movable_unit<mage>
{
	void step_forth()
	{ std::cout << "mage moves forward\n"; }
	
	void step_back()
	{ std::cout << "mage moves back\n"; }
};
\end{cpp}

可以通过调用基类advance和retreat成员函数来推进和后退单位,如下所示:

\begin{cpp}
knight k;
k.advance(3);
k.retreat(2);

mage m;
m.advance(5);
m.retreat(3);
\end{cpp}

可以认为,使用非成员函数模板也可以实现相同的结果。为了便于讨论,这种解决方案的实现如下所示:

\begin{cpp}
struct knight
{
	void step_forth()
	{ std::cout << "knight moves forward\n"; }
	
	void step_back()
	{ std::cout << "knight moves back\n"; }
};

struct mage
{
	void step_forth()
	{ std::cout << "mage moves forward\n"; }
	
	void step_back()
	{ std::cout << "mage moves back\n"; }
};

template <typename T>
void advance(T& t, size_t steps)
{
	while (steps--) t.step_forth();
}

template <typename T>
void retreat(T& t, size_t steps)
{
	while (steps--) t.step_back();
}
\end{cpp}

使用端的代码需要很小的修改:

\begin{cpp}
knight k;
advance(k, 3);
retreat(k, 2);

mage m;
advance(m, 5);
retreat(m, 3);
\end{cpp}

这两者之间的选择可能取决于问题的性质和偏好。而CRTP有一个优点,那就是很好地描述了派生类的接口(比如例子中的knight和mage)。对于非成员函数,并不一定知道这个功能,可能来自需要包含的头文件。使用CRTP时,类接口对于使用者来说是可见的。

针对在这里讨论的最后一个场景,来看看CRTP如何帮助实现复合设计模式。

\subsubsection{7.2.3\hspace{0.2cm}实现复合设计模式}

著名的《设计模式:可重用面向对象软件的元素》中,四人组(Erich Gamma、Richard Helm、Ralph Johnson和John Vlissides)描述了一种称为复合的结构模式,其能够将对象组合成更大的结构,并统一对待单个对象和组合。当希望表示对象的部分-整体层次结构,并且希望忽略单个对象和单个对象的组合之间的差异时,可以使用此模式。

为了将这种模式付诸实践,再次考虑游戏场景。英雄有特殊的能力,可以做不同的行动,其中之一是与另一个英雄结盟。这可以很容易地进行如下建模:

\begin{cpp}
struct hero
{
	hero(std::string_view n) : name(n) {}
	void ally_with(hero& u)
	{
		connections.insert(&u);
		u.connections.insert(this);
	}
private:
	std::string name;
	std::set<hero*> connections;
	friend std::ostream& operator<<(std::ostream& os,
									hero const& obj);
};

std::ostream& operator<<(std::ostream& os,
hero const& obj)
{
	for (hero* u : obj.connections)
		os << obj.name << " --> [" << u->name << "]" << '\n';
		
	return os;
}
\end{cpp}

这些英雄由hero类来表示,这个英雄类包含一个名字,一个到其他hero对象的连接列表,以及一个成员函数ally\_with,它定义了两个英雄之间的联盟。可以这样使用:

\begin{cpp}
hero k1("Arthur");
hero k2("Sir Lancelot");
hero k3("Sir Gawain");

k1.ally_with(k2);
k2.ally_with(k3);

std::cout << k1 << '\n';
std::cout << k2 << '\n';
std::cout << k3 << '\n';
\end{cpp}

运行代码段的输出:

\begin{shell}
Arthur --> [Sir Lancelot]

Sir Lancelot --> [Arthur]
Sir Lancelot --> [Sir Gawain]

Sir Gawain --> [Sir Lancelot]
\end{shell}

目前为止,一切都很简单。但要求是英雄们可以聚集在一起组成政党。一个英雄可以与一个团体结盟,一个团体可以与一个英雄结盟,也可以与整个团体结盟。突然间,需要的功能激增:

\begin{cpp}
struct hero_party;

struct hero
{
	void ally_with(hero& u);
	void ally_with(hero_party& p);
};

struct hero_party : std::vector<hero>
{
	void ally_with(hero& u);
	void ally_with(hero_party& p);
};
\end{cpp}

这就是复合设计模式可以统一对待英雄和团队,并避免不必要的代码重复的地方。通常有不同的实现方法,但其中一种方法是使用重复出现的模板模式。实现需要定义公共接口的基类。我们的例子中,这将是一个类模板,只有一个名为ally\_with的成员函数:

\begin{cpp}
template <typename T>
struct base_unit
{
	template <typename U>
	void ally_with(U& other);
};
\end{cpp}

我们把hero类定义为base\_unit<hero>的派生类。这一次,hero类不再实现自身的ally\_with。但它提供了begin和end方法,用于模拟容器的行为:

\begin{cpp}
struct hero : base_unit<hero>
{
	hero(std::string_view n) : name(n) {}
	
	hero* begin() { return this; }
	hero* end() { return this + 1; }
	
private:
	std::string name;
	std::set<hero*> connections;
	
	template <typename U>
	friend struct base_unit;
	
	template <typename U>
	friend std::ostream& operator<<(std::ostream& os,
									base_unit<U>& object);
};
\end{cpp}

为一组英雄建模的类叫做hero\_party,源于std::vector<hero>(用来定义英雄对象的容器)和base\_unit<hero\_party>。这就是为什么hero类有begin和end函数来对英雄对象执行迭代操作,就像对hero\_party对象所做的那样:

\begin{cpp}
struct hero_party : std::vector<hero>,
					base_unit<hero_party>
{};
\end{cpp}

需要实现基类的ally\_with成员函数,代码如下所示。其所做的是遍历当前对象的所有子对象,并将它们与所提供参数的所有子对象连接起来:

\begin{cpp}
template <typename T>
template <typename U>
void base_unit<T>::ally_with(U& other)
{
	for (hero& from : *static_cast<T*>(this))
	{
		for (hero& to : other)
		{
			from.connections.insert(&to);
			to.connections.insert(&from);
		}
	}
}
\end{cpp}

hero类将基类base\_unit声明为友元,以便访问connections成员。其还将操作符<{}<声明为友元,以便该函数可以访问connections和name私有成员。有关模板及其朋友的更多信息,请参阅第4章的相关章节。输出流操作符的实现如下所示:

\begin{cpp}
template <typename T>
std::ostream& operator<<(std::ostream& os,
base_unit<T>& object)
{
	for (hero& obj : *static_cast<T*>(&object))
	{
		for (hero* n : obj.connections)
			os << obj.name << " --> [" << n->name << "]"
				<< '\n';
	}
	return os;
}
\end{cpp}

定义了所有这些之后,可以编写如下代码:

\begin{cpp}
hero k1("Arthur");
hero k2("Sir Lancelot");

hero_party p1;
p1.emplace_back("Bors");

hero_party p2;
p2.emplace_back("Cador");
p2.emplace_back("Constantine");

k1.ally_with(k2);
k1.ally_with(p1);

p1.ally_with(k2);
p1.ally_with(p2);

std::cout << k1 << '\n';
std::cout << k2 << '\n';
std::cout << p1 << '\n';
std::cout << p2 << '\n';
\end{cpp}

这里,可以让一个hero与另一个hero和一个hero\_party结盟,也可以让一个hero\_party与一个hero或另一个hero\_party结盟。这就是我们的目标,并且能够在不复制hero和hero\_party之间的代码的情况下做到。执行上一段代码的输出如下所示:

\begin{shell}
Arthur --> [Sir Lancelot]
Arthur --> [Bors]

Sir Lancelot --> [Arthur]
Sir Lancelot --> [Bors]

Bors --> [Arthur]
Bors --> [Sir Lancelot]
Bors --> [Cador]
Bors --> [Constantine]

Cador --> [Bors]
Constantine --> [Bors]
\end{shell}

了解了CRTP如何帮助实现不同的目标之后,再来看看CRTP在C++标准库中的使用。

\subsubsection{7.2.4\hspace{0.2cm}标准库中的CRTP}

标准库包含一个名为std::enabled\_shared\_from\_this的辅助类型(<memory>头文件中),允许由std::shared\_ptr管理的对象以安全的方式生成更多std::shared\_ptr实例。std::enabled\_shared\_from\_this类是CRTP模式中的基类。前面的描述可能很抽象,所以我们通过例子来理解它。

假设有一个叫做building的类,正在以以下方式创建std::shared\_ptr对象:

\begin{cpp}
struct building {};

building* b = new building();
std::shared_ptr<building> p1{ b }; // [1]
std::shared_ptr<building> p2{ b }; // [2] bad
\end{cpp}

我们有一个原始指针[1],实例化了一个std::shared\_ptr对象来管理其生命周期。但在[2]处,可为同一个指针实例化了第二个std::shared\_ptr对象。当这两个智能指针对彼此一无所知,因此一旦超出作用域,它们都会删除堆上分配的构建对象。删除已经删除的对象是未定义的行为,可能会导致程序崩溃。

std::enable\_shared\_from\_this类以安全的方式从现有的shared\_ptr对象创建更多的shared\_ptr对象。首先,需要实现CRTP模式:

\begin{cpp}
struct building : std::enable_shared_from_this<building>
{
};
\end{cpp}

有了这个新的实现,可以调用成员函数shared\_from\_this来从一个对象创建更多的std::shared\_ptr实例,这些实例都引用了该对象的同一个实例:

\begin{cpp}
building* b = new building();
std::shared_ptr<building> p1{ b }; // [1]
std::shared_ptr<building> p2{
	b->shared_from_this()}; // [2] OK
\end{cpp}

std::enable\_shared\_from\_this的接口实现如下:

\begin{cpp}
template <typename T>
class enable_shared_from_this
{
	public:
	std::shared_ptr<T> shared_from_this();
	std::shared_ptr<T const> shared_from_this() const;
	std::weak_ptr<T> weak_from_this() noexcept;
	std::weak_ptr<T const> weak_from_this() const noexcept;
	enable_shared_from_this<T>& operator=(
		const enable_shared_from_this<T> &obj ) noexcept;
};
\end{cpp}

前面的例子展示了enable\_shared\_from\_this是如何工作的,但并不有助于理解什么时候使用它合适。因此,让我们修改示例,展示一个实际的示例。

试想现有的建筑可以升级,这是一个需要一些时间和几个步骤的过程。这个任务,以及游戏中的其他任务,都是由一个指定的实体执行的,我们称之为executor。最简单的形式中,这个executor类有一个名为execute的公共成员函数,该函数接受一个函数对象,并在不同的线程上执行它。下面是一种简单的实现:

\begin{cpp}
struct executor
{
	void execute(std::function<void(void)> const& task)
	{
		threads.push_back(std::thread([task]() {
			using namespace std::chrono_literals;
			std::this_thread::sleep_for(250ms);
			task();
		}));
	}

	~executor()
	{
		for (auto& t : threads)
		t.join();
	}
private:
	std::vector<std::thread> threads;
};
\end{cpp}

building类有一个指向executor的指针,该指针从调用端传递过来。其还有一个名为upgrade的成员函数,用于启动执行过程。然而,实际的升级发生在一个不同的、私有的、名为do\_upgrade的函数中,这是从传递给executor的execute成员函数的Lambda表达式中调用的。所有这些都显示在下面的代码中:

\begin{cpp}
struct building
{
	building() { std::cout << "building created\n"; }
	~building() { std::cout << "building destroyed\n"; }
	
	void upgrade()
	{
		if (exec)
		{
			exec->execute([self = this]() {
				self->do_upgrade();
			});
		}
	}
	void set_executor(executor* e) { exec = e; }
private:
	void do_upgrade()
	{
		std::cout << "upgrading\n";
		operational = false;
		
		using namespace std::chrono_literals;
		std::this_thread::sleep_for(1000ms);
		
		operational = true;
		std::cout << "building is functional\n";
	}

	bool operational = false;
	executor* exec = nullptr;
};
\end{cpp}

调用端代码相对简单:创建一个executor,创建一个shared\_ptr管理的building,设置executor引用,并进行升级:

\begin{cpp}
int main()
{
	executor e;
	std::shared_ptr<building> b =
		std::make_shared<building>();
	b->set_executor(&e);
	b->upgrade();
	
	std::cout << "main finished\n";
}
\end{cpp}

若运行这个程序,会得到以下输出:

\begin{shell}
building created
main finished
building destroyed
upgrading
building is functional
\end{shell}

在升级过程开始之前,建筑就摧毁了。这会导致了未定义的行为,尽管这个程序没有崩溃,但实际的程序肯定会崩溃。

这种行为的罪魁祸首是升级代码中的这一行:

\begin{cpp}
exec->execute([self = this]() {
	self->do_upgrade();
});
\end{cpp}

正在创建一个Lambda表达式来捕获this指针,该指针稍后在所指向的对象销毁后使用。为了避免这种情况,需要创建并捕获一个shared\_ptr对象,安全的方法是借助std::enable\_shared\_from\_this类。有两个改变,第一个是从std::enable\_shared\_from\_this类中派生building类:

\begin{cpp}
struct building : std::enable_shared_from_this<building>
{
	/* … */
};
\end{cpp}

第二个变化是要在Lambda捕获中调用shared\_from\_this:

\begin{cpp}
exec->execute([self = shared_from_this()]() {
	self->do_upgrade();
});
\end{cpp}

这对代码的两个小更改,但效果非常显著。Lambda表达式在单独的线程上执行之前,building对象不再销毁(因为现在有一个的共享指针,引用与主函数中创建的共享指针相同的对象)。结果,得到了我们期望的输出(没有对调用端代码进行更改):

\begin{shell}
building created
main finished
upgrading
building is functional
building destroyed
\end{shell}

你可以争辩说,主函数完成后,不应该关心发生了什么。注意,这只是一个演示程序。在实践中,这会发生在其他一些函数中,并且在该函数返回后,程序会继续运行很长时间。

至此,我们结束了关于奇怪的重复模板模式的讨论。接下来,我们将研究混入(Mixins)技术,它经常与CRTP模式混合使用。





















