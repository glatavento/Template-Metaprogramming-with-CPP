\section{标记分派}

标记分派是一种技术,能够在编译时选择一个或另一个函数重载,是std::enable_if和SFINAE的替代方案,易于理解和使用。术语标记描述了一个没有成员(数据)或函数(行为)的空类。这样的类仅用于定义函数的形参(通常是最后一个),以决定是否在编译时选择它,这取决于提供的参数。

标准库包含一个名为std::advance的实用函数:

\begin{cppcode}
template<typename InputIt, typename Distance>
void advance(InputIt& it, Distance n);
\end{cppcode}

C++17中,这也是constexpr(稍后会详细介绍)。该函数将给定的迭代器增加n个元素。但迭代器有几种类型(输入、输出、向前、双向和随机),所以这样的操作可以以不同的方式计算:

\begin{itemize}
  \item 对于输入迭代器,可以调用operator++ n次。
  \item 对于双向迭代器,可以调用operator++ n次(若n是一个正数)或运算符-n次(若n是一个负数)。
  \item 对于随机访问迭代器,它可以使用operator+=直接增加n个元素。
\end{itemize}

所以,可以有三种不同的实现,可以在编译时选择与所调用迭代器的类别最匹配的实现。一种解决方案是标记分派,首先要做的是定义标记,标记是空类。因此,对应于这5种迭代器类型的标记可进行如下定义:

\begin{cppcode}
struct input_iterator_tag {};
struct output_iterator_tag {};
struct forward_iterator_tag : input_iterator_tag {};
struct bidirectional_iterator_tag :
	forward_iterator_tag {};
struct random_access_iterator_tag :
	bidirectional_iterator_tag {};
\end{cppcode}

这正是C++标准库中std名称空间中定义它们的方式,这些标记用于为std::advance的每次重载定义一个附加参数:

\begin{cppcode}
namespace std
{
	namespace details
	{
		template <typename Iter, typename Distance>
		void advance(Iter& it, Distance n,
		std::random_access_iterator_tag)
		{
			it += n;
		}
	
		template <typename Iter, typename Distance>
		void advance(Iter& it, Distance n,
					 std::bidirectional_iterator_tag)
		{
			if (n > 0)
			{
				while (n--) ++it;
			}
			else
			{
				while (n++) --it;
			}
		}

		template <typename Iter, typename Distance>
		void advance(Iter& it, Distance n,
				     std::input_iterator_tag)
		{
			while (n--)
			{
				++it;
			}
		}
	}
}
\end{cppcode}

这些重载定义在std命名空间的单独(内部)命名空间中,这样标准命名空间就不会被不必要的定义所污染。这里可以看到,每个重载都有三个参数:对迭代器的引用、要递增(或递减)的元素数量和标记。

最后要做的是提供一个用于直接使用的高级函数的定义。这个函数没有第三个形参,但是通过确定调用迭代器的类别来调用这些重载中的一个。其实现可能如下所示:

\begin{cppcode}
namespace std
{
	template <typename Iter, typename Distance>
	void advance(Iter& it, Distance n)
	{
		details::advance(it, n,
			typename std::iterator_traits<Iter>::
								iterator_category{});
	}
}
\end{cppcode}

std::iterator_traits类为迭代器类型定义了一种接口,包含几个成员类型,其中一个是iterator_category。这将解析为前面定义的一个迭代器标记,例如std::input_iterator_tag用于输入迭代器,std::random_access_iterator_tag用于随机迭代器。基于所提供的迭代器的类别,实例化这些标记类,在编译时从details命名空间确定适当的重载选择。就可以像下面这样使用std::advance函数了:

\begin{cppcode}
std::vector<int> v{ 1,2,3,4,5 };
auto sv = std::begin(v);
std::advance(sv, 2);

std::list<int> l{ 1,2,3,4,5 };
auto sl = std::begin(l);
std::advance(sl, 2);
\end{cppcode}

std::vector迭代器的类别类型是随机访问。另一方面,std::list的迭代器类别类型是双向的。但可以利用标记调度技术使用依赖于不同优化实现的单个函数。

\subsection{标签分派的替代方案}

C++17之前,标记分派的唯一替代方案是SFINAE和enable_if。这是一种相当传统的技术,在现代C++中有更好的替代方案——constexpr if和概念。

\subsubsection{constexpr if}

C++11引入了constexpr值的概念,其是编译时已知的值,也是可以在编译时求值的函数(若所有输入都是编译时的值)。C++14、C++17和C++20中,许多标准库函数或标准库类的成员函数更改为constexpr。其中之一是std::advance,它在C++17中的实现基于constexpr if特性,也在C++17中添加的。

下面是C++17中的可能实现:

\begin{cppcode}
template<typename It, typename Distance>
constexpr void advance(It& it, Distance n)
{
	using category =
	typename std::iterator_traits<It>::iterator_category;
	static_assert(std::is_base_of_v<std::input_iterator_tag,
	category>);
	auto dist =
	typename std::iterator_traits<It>::difference_type(n);
	if constexpr (std::is_base_of_v<
	std::random_access_iterator_tag,
	category>)
	{
		it += dist;
	}
	else
	{
		while (dist > 0)
		{
			--dist;
			++it;
		}
		if constexpr (std::is_base_of_v<
						std::bidirectional_iterator_tag,
						category>)
		{
			while (dist < 0)
			{
				++dist;
				--it;
			}
		}
	}
}
\end{cppcode}

虽然这个实现仍然使用前面的迭代器标记,但不再用于调用不同的重载函数,而是用于确定一些编译时表达式的值。std::is_base_of类型特性(通过std::is_base_of_v变量模板)用于在编译时确定迭代器类别的类型。

这个实现有几个优点:

\begin{itemize}
  \item 具有算法的单一实现(std命名空间中)
  \item 不需要多个重载,实现细节定义在单独的命名空间中
\end{itemize}

调用端代码不受影响,所以标准库实现者能够用基于constexpr if的新版本替换基于标记分派的旧版本,而不会影响调用std::advance的代码。

然而,在C++20中有一个更好的选择。接下来让我们来探索一下。

\subsubsection{概念}

前一章专门介绍C++20中引入的约束和概念,不仅看到了这些特性是如何工作的,还看到了标准库在几个头文件中定义的一些概念,如<concepts>和<iterator>。其中一些概念指定类型是某个迭代器类别。例如,std::input_iterator指定类型为输入迭代器。类似地,还定义了以下概念:std::output_iterator、std::forward_iterator、std::bidirectional_iterator、std::random_access_iterator和std::contiguous_iterator(最后一个表示迭代器是随机访问迭代器,指的是连续存储在内存中的元素)。

std::input_iterator概念的定义如下:

\begin{cppcode}
template<class I>
	concept input_iterator =
		std::input_or_output_iterator<I> &&
		std::indirectly_readable<I> &&
		requires { typename /*ITER_CONCEPT*/<I>; } &&
		std::derived_from</*ITER_CONCEPT*/<I>,
						  std::input_iterator_tag>;
\end{cppcode}

值得注意的是,这个概念是一组约束,用于验证以下内容:

\begin{itemize}
  \item 迭代器是可解引用的(支持*i)和可递增的(支持++i和i++)
  \item 迭代器类别派生自std::input_iterator_tag。
\end{itemize}

类别检查是在约束内执行的,所以这些概念仍然基于迭代器标记,但技术上与标记调度有很大不同。因此,在C++20中可以有std::advance算法的另一个实现,如下所示:

\begin{cppcode}
template <std::random_access_iterator Iter, class Distance>
void advance(Iter& it, Distance n)
{
	it += n;
}

template <std::bidirectional_iterator Iter, class Distance>
void advance(Iter& it, Distance n)
{
	if (n > 0)
	{
		while (n--) ++it;
	}
	else
	{
		while (n++) --it;
	}
}

template <std::input_iterator Iter, class Distance>
void advance(Iter& it, Distance n)
{
	while (n--)
	{
		++it;
	}
}
\end{cppcode}

有几点需要注意:

\begin{itemize}
  \item advanced函数还有三种不同的重载。
  \item 这些重载定义在std命名空间中,不需要单独的命名空间来隐藏实现细节。
\end{itemize}

尽管再次显式地编写了几个重载,但这种解决方案可以说比基于constexpr if的解决方案更容易阅读和理解,因为代码很好地分离到不同的单元(函数)中,因此更容易理解。

标记分派是在编译时进行重载选择的一项重要技术。但若使用C++17或C++20,也有更好的选择。若编译器支持概念,出于前面提到的原因,应该选择这种替代方案。

下一节讨论的模式是表达式模板。


