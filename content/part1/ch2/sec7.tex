\section{定义变量模板}
变量模板是在C++14中引入,允许在命名空间范围内定义模板变量(可以表示一组全局变量),或者在类范围内定义模板变量(表示静态数据成员)。

变量模板在命名空间范围内声明,如下所示。这是一个典型的例子,可以在文献中找到,这里用它来说明变量模板的好处:

\begin{cpp}
template<class T>
constexpr T PI = T(3.1415926535897932385L);
\end{cpp}

其语法类似于声明变量(或数据成员),但与声明模板的语法相结合。

随之而来的问题是变量模板有什么用处。为了回答这个问题,让我们些写个例子来说明。假设想要写一个函数模板,给定一个球体的半径,返回它的体积。球体体积公式为$\dfrac{4\pi r^{3}}{3}$,可能的实现如下所示:

\begin{cpp}
constexpr double PI = 3.1415926535897932385L;

template <typename T>
T sphere_volume(T const r)
{
	return 4 * PI * r * r * r / 3;
}
\end{cpp}

PI定义为double类型的编译时常数。这将生成编译器警告,若使用float,例如,类型模板参数T:

\begin{cpp}
float v1 = sphere_volume(42.0f); // warning
double v2 = sphere_volume(42.0); // OK
\end{cpp}

这个问题的一种解决方案是,使PI成为模板类的静态数据成员,其类型由类型模板参数决定:

\begin{cpp}
template <typename T>
struct PI
{
	static const T value;
};

template <typename T>
const T PI<T>::value = T(3.1415926535897932385L);

template <typename T>
T sphere_volume(T const r)
{
	return 4 * PI<T>::value * r * r * r / 3;
}
\end{cpp}

尽管使用PI<T>::value并不理想,但也可行,若可以写成PI<T>就更好了。这正是本节开头显示的变量模板PI可以做的事情,下面是完整的解决方案:

\begin{cpp}
template<class T>
constexpr T PI = T(3.1415926535897932385L);

template <typename T>
T sphere_volume(T const r)
{
	return 4 * PI<T> * r * r * r / 3;
}
\end{cpp}

下一个例子展示了变量模板的显式特化:

\begin{cpp}
template<typename T>
constexpr T SEPARATOR = '\n';

template<>
constexpr wchar_t SEPARATOR<wchar_t> = L'\n';

template <typename T>
std::basic_ostream<T>& show_parts(
	std::basic_ostream<T>& s,
	std::basic_string_view<T> const& str)
{
	using size_type =
		typename std::basic_string_view<T>::size_type;
	size_type start = 0;
	size_type end;
	do
	{
		end = str.find(SEPARATOR<T>, start);
		s << '[' << str.substr(start, end - start) << ']'
		<< SEPARATOR<T>;
		start = end+1;
	} while (end != std::string::npos);

	return s;
}

show_parts<char>(std::cout, "one\ntwo\nthree");
show_parts<wchar_t>(std::wcout, L"one line");
\end{cpp}

有一个名为show\_parts的函数模板,会将输入字符串分割为用分隔符分隔的部分后处理。分隔符是在(全局)命名空间范围内定义的变量模板,并且为wchar\_t类型显式特化。

如前所述,变量模板可以是类的成员,可表示静态数据成员,需要使用static关键字进行声明:

\begin{cpp}
struct math_constants
{
	template<class T>
	static constexpr T PI = T(3.1415926535897932385L);
};

template <typename T>
T sphere_volume(T const r)
{
	return 4 * math_constants::PI<T> *r * r * r / 3;
}
\end{cpp}

可以在类中声明一个变量模板,然后在类外提供它的定义。注意,变量模板必须声明为静态const,而不是静态constexpr,因为后者需要在类内初始化:

\begin{cpp}
struct math_constants
{
	template<class T>
	static const T PI;
};

template<class T>
const T math_constants::PI = T(3.1415926535897932385L);
\end{cpp}

变量模板用于简化类型特征的使用。显式特化部分包含一个名is\_float \_point的类型特征示例。下面是其主模板:

\begin{cpp}
template <typename T>
struct is_floating_point
{
	constexpr static bool value = false;
};
\end{cpp}

有几个显式的特化,不在这里进行列出。然而,这种类型的特征可以这样使用:

\begin{cpp}
std::cout << is_floating_point<float>::value << '\n';
\end{cpp}

使用is\_float \_point<float>::value相当麻烦,但是可以通过定义如下的变量模板来避免:

\begin{cpp}
template <typename T>
inline constexpr bool is_floating_point_v =
	is_floating_point<T>::value;
\end{cpp}

is\_floating\_point\_v变量模板有助于编写更简单、更易于阅读的代码。下面是我喜欢的形式:

\begin{cpp}
std::cout << is_floating_point_v<float> << '\n';
\end{cpp}

标准库为::value定义了一系列以\_v结尾的变量模板,就像我们的例子一样(例如std::is\_floating\_point\_v或std::is\_same\_v)。变量模板的实例化类似于函数模板和类模板。

这可以通过显式实例化或显式特化实现,也可以由编译器隐式实现。当在必须存在变量定义的上下文中使用变量模板时,或者需要该变量对表达式求常量时,编译器就会生成定义。

接下来,要介绍的主题为别名模板,其允许我们为类模板定义别名。








