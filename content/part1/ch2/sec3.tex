\section{定义成员函数模板}
我们已经学习了函数模板和类模板。本节中,我们将学习如何在非模板类和类模板中定义成员函数模板。为了理解它们的区别,先来看个例子:

\begin{cppcode}
template <typename T>
class composition
{
public:
	T add(T const a, T const b)
	{
		return a + b;
	}
};
\end{cppcode}

复合类是一个类模板,有一个名为add的成员函数,使用类型形参T。这个类可以这样使用:

\begin{cppcode}
composition<int> c;
c.add(41, 21);
\end{cppcode}

首先需要实例化组合类的一个对象,必须显式地指定类型形参T的实参,因为编译器不能自己推导出来(没有上下文)。当调用add函数时,只提供参数。其类型(由T类型模板形参表示,之前解析为int)是已知的,像c.add<int>(42, 21)这样的调用使编译器报错。add函数不是一个函数模板,而是一个常规函数,它是复合类模板的成员。

下一个示例中,复合类略有变化。先来看看定义:

\begin{cppcode}
class composition
{
public:
	template <typename T>
	T add(T const a, T const b)
	{
		return a + b;
	}
};
\end{cppcode}

composition是非模板类,但add函数是一个函数模板。要调用这个函数,必须执行以下操作:

\begin{cppcode}
composition c;
c.add<int>(41, 21);
\end{cppcode}

为T类型模板形参显式指定int类型是多余的,编译器可以从调用的实参中自行推导出int类型。这里是为了更好地理解这两种实现之间的差异。

除了类模板的成员函数和类成员函数模板这两种情况外,还可以有类模板的成员函数模板。这种情况下,成员函数模板的模板形参必须与类模板的模板形参不同;否则,编译器将报错。回到包装器类模板的示例,并对其进行如下修改:

\begin{cppcode}
template <typename T>
class wrapper
{
public:
	wrapper(T const v) :value(v)
	{}
	
	T const& get() const { return value; }
	
	template <typename U>
	U as() const
	{
		return static_cast<U>(value);
	}
private:
	T value;
};
\end{cppcode}

这个实现还有一个成员,一个名为as的函数。这是一个函数模板,有一个名为U的类型模板形参。该函数用于将包装的值从类型T转换为类型U,并将其返回给调用者:


\begin{cppcode}
wrapper<double> a(42.0);
auto d = a.get(); // double
auto n = a.as<int>(); // int
\end{cppcode}

模板形参的实参在实例化包装器类时指定(double)——尽管在C++17中这是冗余的,并且在调用as函数(int)执行时也可以指定。

继续其他内容(例如实例化、特化和其他形式的模板,包括变量和别名)之前,花点时间了解更多关于模板参数的知识非常重要。







