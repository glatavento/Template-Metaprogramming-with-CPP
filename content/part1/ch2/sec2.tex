\section{定义类模板}
类模板将template关键字和template参数列表放在类声明之前。下面的代码展示了wrapper类模板,其只有一个模板形参(类型为T),用作数据成员、形参和函数返回类型的类型:

\begin{cppcode}
template <typename T>
class wrapper
{
public:
	wrapper(T const v): value(v)
	{ }
	
	T const& get() const { return value; }
	
private:
	T value;
};
\end{cppcode}

只要源码中没有使用类模板,编译器就不会生成代码。所以类模板必须实例化,并且其所有形参必须由用户显式地匹配,或由编译器隐式地匹配。实例化这个类模板的例子如下:

\begin{cppcode}
wrapper a(42); // wraps an int
wrapper<int> b(42); // wraps an int
wrapper<short> c(42); // wraps a short
wrapper<double> d(42.0); // wraps a double
wrapper e("42"); // wraps a char const *
\end{cppcode}

这段代码中的a和e的定义只在C++17及以后的版本中有效,这要归功于\textbf{类模板参数推导}特性。这个特性使我们可以使用类模板,而不指定任何模板参数,编译器能够进行推导即可(在第4章中讨论)。在此之前,所有引用类模板的示例都将显式列出参数,如wrapper<int>或wrapper<char const*>。

类模板可以在不定义的情况下声明,并在允许不完整类型的上下文中使用,例如函数的声明:

\begin{cppcode}
template <typename T>
class wrapper;

void use_foo(wrapper<int>* ptr);
\end{cppcode}

类模板必须在模板实例化发生的地方定义;否则,编译器将报错。

\begin{cppcode}
template <typename T>
class wrapper; // OK

void use_wrapper(wrapper<int>* ptr); // OK

int main()
{
	wrapper<int> a(42); // error, incomplete type
	use_wrapper(&a);
}

template <typename T>
class wrapper
{
	// template definition
};

void use_wrapper(wrapper<int>* ptr)
{
	std::cout << ptr->get() << '\n';
}
\end{cppcode}

声明use_wrapper函数时,只声明了类模板wrapper,而没有定义。但在此上下文中允许使用不完整类型,因此可以使用wrapper<T>。但在main函数中,正在实例化wrapper类模板的对象。因为类模板的定义必须可用,所以这将生成编译器错误。要修复这个特定的示例,必须将main函数的定义移到末尾,在wrapper和use_wrapper定义之后。

本例中,类模板是使用class关键字定义的。C++中,使用class或struct关键字声明类之间几乎没什么区别:

\begin{itemize}
  \item 对于struct,默认的成员访问是public,而使用class是private。
  \item 对于struct,基类继承的默认访问说明符是public,而使用class是private。
\end{itemize}

可以使用struct关键字定义类模板,就像使用class关键字一样。对于用struct或class关键字定义的类模板,也可以观察到用struct或class关键字定义的类之间的差异。

类,不管是不是模板,都可以包含成员函数模板。下一节将讨论如何定义这些成员函数。

















