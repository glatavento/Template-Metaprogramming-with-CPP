\section{通用Lambda和Lambda模板}
Lambda在形式上称为Lambda表达式,是对于定义函数对象的一种简化方法,通常包括传递给算法的谓词或比较函数。我们不会讨论Lambda表达式,但先来看看下面的例子:

\begin{cpp}
int arr[] = { 1,6,3,8,4,2,9 };
std::sort(
	std::begin(arr), std::end(arr),
	[](int const a, int const b) {return a > b; });
	
int pivot = 5;
auto count = std::count_if(
	std::begin(arr), std::end(arr),
	[pivot](int const a) {return a > pivot; });
\end{cpp}

Lambda表达式是语法糖,是定义匿名函数对象的一种简化方式。遇到Lambda表达式时,编译器会生成一个带有函数调用操作符的类。对于前面的例子,如下所示:

\begin{cpp}
struct __lambda_1
{
	inline bool operator()(const int a, const int b) const
	{
		return a > b;
	}
};

struct __lambda_2
{
	__lambda_2(int & _pivot) : pivot{_pivot}
	{}
	
	inline bool operator()(const int a) const
	{
		return a > pivot;
	}
private:
	int pivot;
};
\end{cpp}

这里的名称很随意,每个编译器将生成不同的名称。另外,实现细节可能有所不同,这里看到的是编译器应该生成的最小值。注意,第一个Lambda和第二个Lambda的区别在于,后者包含通过值捕获的状态。

Lambda表达式是在C++11中引入的,该标准的后续版本中进行了多次更新。值得注意的是,本章将讨论两个问题:

\begin{itemize}
\item 
泛型Lambda:在C++14中引入,其允许我们使用auto,而不是显式地指定类型。这将生成的函数对象转换为,具有模板函数调用操作符的函数对象。

\item 
模板Lambda:在C++20中引入,模板Lambda可以使用模板语法显式指定模板化的函数调用操作符。
\end{itemize}

为了理解它们之间的区别,以及泛型Lambda和模板Lambda的作用,来看看以下示例:

\begin{cpp}
auto l1 = [](int a) {return a + a; }; // C++11, regular
                                      // lambda
auto l2 = [](auto a) {return a + a; }; // C++14, generic
                                       // lambda
                                       
auto l3 = []<typename T>(T a)
          { return a + a; }; // C++20, template lambda

auto v1 = l1(42); // OK
auto v2 = l1(42.0); // warning
auto v3 = l1(std::string{ "42" }); // error

auto v5 = l2(42); // OK
auto v6 = l2(42.0); // OK
auto v7 = l2(std::string{"42"}); // OK

auto v8 = l3(42); // OK
auto v9 = l3(42.0); // OK
auto v10 = l3(std::string{ "42" }); // OK
\end{cpp}

这里有三个不同的Lambda表达式: l1是一个常规Lambda, l2是一个泛型Lambda,因为至少有一个参数用auto定义,l3是一个模板Lambda,用模板语法定义,但没有使用template关键字。

可以用一个整数调用l1,也可以使用double类型调用,但这一次编译器将产生一个关于可能丢失数据的警告。尝试用字符串参数调用会产生编译错误,因为std::string不能转换为int。另一方面,l2是泛型。编译器继续为它调用的所有参数类型实例化其特化版本,本例中是int、double和std::string。下面的代码展示了生成的函数的外观(至少在概念上):

\begin{cpp}
struct __lambda_3
{
	template<typename T1>
	inline auto operator()(T1 a) const
	{
		return a + a;
	}

	template<>
	inline int operator()(int a) const
	{
		return a + a;
	}

	template<>
	inline double operator()(double a) const
	{
		return a + a;
	}

	template<>
	inline std::string operator()(std::string a) const
	{
		return std::operator+(a, a);
	}
};
\end{cpp}

可以在这里看到函数调用操作符的主模板,以及提到的三个特化。毫不奇怪,编译器将为第三个Lambda表达式l3生成相同的代码,这是一个模板Lambda,仅在C++20中可用。由此产生的问题是,通用Lambda和Lambda模板有什么不同?为了回答这个问题,稍微修改一下前面的例子:

\begin{cpp}
auto l1 = [](int a, int b) {return a + b; };
auto l2 = [](auto a, auto b) {return a + b; };
auto l3 = []<typename T, typename U>(T a, U b)
          { return a + b; };
          
auto v1 = l1(42, 1); // OK
auto v2 = l1(42.0, 1.0); // warning
auto v3 = l1(std::string{ "42" }, '1'); // error

auto v4 = l2(42, 1); // OK
auto v5 = l2(42.0, 1); // OK
auto v6 = l2(std::string{ "42" }, '1'); // OK
auto v7 = l2(std::string{ "42" }, std::string{ "1" }); // OK

auto v8 = l3(42, 1); // OK
auto v9 = l3(42.0, 1); // OK
auto v10 = l3(std::string{ "42" }, '1'); // OK
auto v11 = l3(std::string{ "42" }, std::string{ "42" }); // OK
\end{cpp}

新Lambda表达式有两个参数,可以用两个整数或一个int和一个double(尽管这同样会产生警告)调用l1,但不能用字符串和char调用,但可以用泛型Lambda l2和模板Lambda l3来完成所有这些。从语义上看,编译器生成的代码对于l2和l3是相同的:

\begin{cpp}
struct __lambda_4
{
	template<typename T1, typename T2>
	inline auto operator()(T1 a, T2 b) const
	{
		return a + b;
	}

	template<>
	inline int operator()(int a, int b) const
	{
		return a + b;
	}

	template<>
	inline double operator()(double a, int b) const
	{
		return a + static_cast<double>(b);
	}

	template<>
	inline std::string operator()(std::string a,
	char b) const
	{
		return std::operator+(a, b);
	}

	template<>
	inline std::string operator()(std::string a,
	std::string b) const
	{
		return std::operator+(a, b);
	}
};
\end{cpp}

我们看到了函数调用操作符的主模板,以及几个完全显式的特化:两个int值,一个double和一个int,一个字符串和一个char,以及两个字符串对象。但若想将泛型Lambda l2的使用限制为相同类型的参数呢?这是不可能的。编译器无法推断我们的意图,所以将为参数列表中出现的auto,生成不同类型的模板参数。然而,C++20的Lambda模板允许指定函数调用操作符的形式。来看看下面的例子:

\begin{cpp}
auto l5 = []<typename T>(T a, T b) { return a + b; };

auto v1 = l5(42, 1); // OK
auto v2 = l5(42, 1.0); // error

auto v4 = l5(42.0, 1.0); // OK
auto v5 = l5(42, false); // error

auto v6 = l5(std::string{ "42" }, std::string{ "1" }); // OK

auto v6 = l5(std::string{ "42" }, '1'); // error
\end{cpp}

使用两个不同类型的参数调用Lambda模板是不可能的,即使可以隐式转换,例如从int转换为double。编译器将会报错。调用模板Lambda时,不可能显式地提供模板参数,例如:l5<double>(42,1.0)会产生一个编译错误。

decltype类型说明符允许告知编译器从表达式推导出类型。C++14中,可以在泛型Lambda中使用它,来说明前面泛型Lambda表达式中的第二个参数具有与第一个参数相同的类型。看起来如下所示:

\begin{cpp}
auto l4 = [](auto a, decltype(a) b) {return a + b; };
\end{cpp}

所以第二个参数b的类型必须转换为第一个参数a的类型,并且可以进行以下方式的调用:

\begin{cpp}
auto v1 = l4(42.0, 1); // OK
auto v2 = l4(42, 1.0); // warning
auto v3 = l4(std::string{ "42" }, '1'); // error
\end{cpp}

因为int可以隐式转换为double,所以第一次调用编译时没有问题。第二个调用编译时发出警告,因为从double转换为int可能会导致数据丢失。但第三个调用会报错,因为char不能隐式转换为std::string。l4 Lambda比前面看到的泛型Lambda l2有所改进,但若参数类型不同,仍然不能帮助完全限制调用。这仅适用于前面所示的Lambda模板。

下一段代码中显示了Lambda模板的另一个示例。这个Lambda只有一个参数std::array,但数组元素的类型和数组的大小指定为Lambda模板的模板参数:

\begin{cpp}
auto l = []<typename T, size_t N>(
            std::array<T, N> const& arr)
{
	return std::accumulate(arr.begin(), arr.end(),
	                       static_cast<T>(0));
};

auto v1 = l(1); // error
auto v2 = l(std::array<int, 3>{1, 2, 3}); // OK
\end{cpp}

试图用std::array对象以外的对象调用这个Lambda会产生编译器错误。编译器生成的函数对象可能如下所示:

\begin{cpp}
struct __lambda_5
{
	template<typename T, size_t N>
	inline auto operator()(
		const std::array<T, N> & arr) const
	{
		return std::accumulate(arr.begin(), arr.end(),
							   static_cast<T>(0));
	}

	template<>
	inline int operator()(
		const std::array<int, 3> & arr) const
	{
		return std::accumulate(arr.begin(), arr.end(),
							   static_cast<int>(0));
	}
};
\end{cpp}

泛型Lambda相对于常规Lambda的好处是递归Lambda。Lambda是匿名的,因此不能直接递归调用。必须定义一个std::function对象,将Lambda表达式赋给它,并在捕获列表中通过引用捕获它。下面是一个递归Lambda的例子,用来计算数字的阶乘:

\begin{cpp}
std::function<int(int)> factorial;
factorial = [&factorial](int const n) {
	if (n < 2) return 1;
	else return n * factorial(n - 1);
};

factorial(5);
\end{cpp}

这可以通过使用泛型Lambdas来简化,不需要std::function及其捕获。递归泛型Lambda的实现如下所示:

\begin{cpp}
auto factorial = [](auto f, int const n) {
	if (n < 2) return 1;
	else return n * f(f, n - 1);
};

factorial(factorial, 5);
\end{cpp}

若理解起来很难,编译器生成的代码应该能帮助搞清楚这是如何进行工作的:

\begin{cpp}
struct __lambda_6
{
	template<class T1>
	inline auto operator()(T1 f, const int n) const
	{
		if(n < 2) return 1;
		else return n * f(f, n - 1);
	}

	template<>
	inline int operator()(__lambda_6 f, const int n) const
	{
		if(n < 2) return 1;
		else return n * f.operator()(__lambda_6(f), n - 1);
	}
};

__lambda_6 factorial = __lambda_6{};
factorial(factorial, 5);
\end{cpp}

泛型Lambda是一个带有模板函数调用操作符的函数对象。用auto指定的第一个参数可以是任何东西,包括Lambda本身。因此,编译器将为生成的类型,提供完全显式特化的调用操作符。

当需要将函数对象作为参数传递给其他函数时,Lambda表达式可以避免编写显式代码。相反,编译器可以生成代码。泛型Lambda,在C++14中引入,避免为不同类型编写相同的Lambda。C++20的Lambda模板允许在模板语法和语义下,指定生成调用操作符的形式。


