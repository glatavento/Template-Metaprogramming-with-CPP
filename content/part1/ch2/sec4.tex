\section{模板参数}

我们已经看到了带有一个或多个参数的模板的多个示例,其中参数表示实例化时提供的类型,这些类型可以由用户显式提供,也可以由编译器在可以推导时隐式提供。这些类型的参数称为类型模板参数,但模板也可以有非类型模板参数和双重模板参数。在接下来的部分中,我们将探讨这些问题。

\subsubsection{2.4.1\hspace{0.2cm}类型模板参数}

如前所述,这些参数表示模板实例化期间作为参数提供的类型,由typename或class关键字引入。使用这两个关键字没什么区别。类型模板参数可以有一个默认值,即类型。这与为函数形参指定默认值的方式相同:

\begin{cpp}
template <typename T>
class wrapper { /* ... */ };

template <typename T = int>
class wrapper { /* ... */ };
\end{cpp}

类型模板参数的名称可以省略,这在转发声明时很有用:

\begin{cpp}
template <typename>
class wrapper;

template <typename = int>
class wrapper;
\end{cpp}

C++11引入了可变参数模板,这些模板的参数数量可变。接受零个或多个参数的模板形参称为形参包,类型模板参数包具有以下形式:

\begin{cpp}
template <typename... T>
class wrapper { /* ... */ };
\end{cpp}

可变参数模板将在第3章中讨论,这里不会详细讨论这类参数。

C++20引入了概念和约束。约束指定了对模板参数的要求,一系列命名的约束称为概念。概念可以指定为类型模板参数,但语法有点不同。使用概念的名称(若有的话,后面跟着尖括号中的模板参数列表)来代替typename或class关键字。具有默认值和约束类型模板参数包的概念:

\begin{cpp}
template <WrappableType T>
class wrapper { /* ... */ };

template <WrappableType T = int>
class wrapper { /* ... */ };

template <WrappableType... T>
class wrapper { /* ... */ };
\end{cpp}

概念和约束将在第6章中讨论,将在那一章学习更多关于这类参数的知识。现在,来看看第二种模板参数,非类型模板参数。


\subsubsection{2.4.2\hspace{0.2cm}非类型模板参数}

模板参数并不总是必须表示类型,也可以是编译时表达式,例如常量、外部函数,或对象的地址或静态类成员的地址。编译时表达式提供的参数称为非类型模板参数,这类参数只能有结构类型。以下是结构类型的描述:

\begin{itemize}
\item 
整数类型

\item 
浮点类型(始于C++20)

\item 
枚举类型

\item 
指针类型(指向对象或函数)

\item 
指向成员类型的指针(指向成员对象或成员函数)

\item 
左值引用类型(指向对象或函数)

\item 
字面量类类型,需要满足以下要求:
\begin{itemize}
\item 
所有基类都是public且不可变的。

\item 
所有非静态数据成员都是public且不可变的。

\item 
所有基类和非静态数据成员的类型也是结构类型或其数组。
\end{itemize}
\end{itemize}

这些类型的cv限定形式也可以用于非类型模板参数。非类型模板参数可以以不同的方式指定,可能的形式如下所示:

\begin{cpp}
template <int V>
class foo { /*...*/ };

template <int V = 42>
class foo { /*...*/ };

template <int... V>
class foo { /*...*/ };
\end{cpp}

这些例子中,非类型模板形参的类型是int。第一个示例和第二个示例类似,只是第二个示例使用了默认值。第三个示例有很大不同,因为参数实际上是一个参数包,这将在下一章讨论。

为了更好地理解非类型模板参数,来看看下面的例子,这里有一个固定大小的数组类,称为buffer:

\begin{cpp}
template <typename T, size_t S>
class buffer
{
	T data_[S];
public:
	constexpr T const * data() const { return data_; }
	
	constexpr T& operator[](size_t const index)
	{
		return data_[index];
	}

	constexpr T const & operator[](size_t const index) const
	{
		return data_[index];
	}
};
\end{cpp}

这个buffer类包含S个T类型元素的内部数组,所以S需要是一个编译类型的值。这个类可以进行如下的实例化:

\begin{cpp}
buffer<int, 10> b1;
buffer<int, 2*5> b2;
\end{cpp}

这两个定义相同,b1和b2都是两个存储10个整数的buffer。此外,其具有相同的类型,因为2*5和10是两个表达式,其计算值为相同的编译时值。可以用下面的语句进行检查:

\begin{cpp}
static_assert(std::is_same_v<decltype(b1), decltype(b2)>);
\end{cpp}

现在不再是这样了,b3对象的类型声明如下:

\begin{cpp}
buffer<int, 3*5> b3;
\end{cpp}

b3是一个包含15个整数的buffer,这与前面示例中包含10个整数的buffer类型不同。从概念上讲,编译器会生成以下代码:

\begin{cpp}
template <typename T, size_t S>
class buffer
{
	T data_[S];
public:
	constexpr T* data() const { return data_; }
	
	constexpr T& operator[](size_t const index)
	{
		return data_[index];
	}

	constexpr T const & operator[](size_t const index) const
	{
		return data_[index];
	}
};
\end{cpp}

这是主模板的代码,下面显示了一些特化:

\begin{cpp}
template<>
class buffer<int, 10>
{
	int data_[10];
public:
	constexpr int * data() const;
	constexpr int & operator[](const size_t index);
	constexpr const int & operator[](
		const size_t index) const;
};

template<>
class buffer<int, 15>
{
	int data_[15];
public:
	constexpr int * data() const;
	constexpr int & operator[](const size_t index);
	constexpr const int & operator[](
		const size_t index) const;
};
\end{cpp}

这个示例中,可以看到的特化的概念(本章后续的内容中详细介绍)。目前,应该注意到这两种不同的buffer类型,可以用下面的语句验证b1和b3的类型是否相同:

\begin{cpp}
static_assert(!std::is_same_v<decltype(b1), decltype(b3)>);
\end{cpp}

实践中,结构类型(如整数、浮点数或枚举类型)的使用比其他类型更常见。理解其用法,并找到相应的例子可能更容易,也有使用指针或引用的场景。下面的例子中,将研究如何使用指向函数参数的指针。先来看看代码:

\begin{cpp}
struct device
{
	virtual void output() = 0;
	virtual ~device() {}
};

template <void (*action)()>
struct smart_device : device
{
	void output() override
	{
		(*action)();
	}
};
\end{cpp}

device是一个基类,具有一个名为output的纯虚函数(以及一个虚析构函数)。这是名为smart\_device的类模板的基类,该类模板通过函数指针调用函数来实现output虚函数,此函数的指针会作为传递给类模板的非类型模板的实参。下面的示例展示了如何使用:

\begin{cpp}
void say_hello_in_english()
{
	std::cout << "Hello, world!\n";
}

void say_hello_in_spanish()
{
	std::cout << "Hola mundo!\n";
}

auto w1 =
	std::make_unique<smart_device<&say_hello_in_english>>();
w1->output();

auto w2 =
	std::make_unique<smart_device<&say_hello_in_spanish>>();
w2->output();
\end{cpp}

w1和w2是两个unique\_ptr对象,指向相同类型的对象,但事实并非如此。因为smart\_device<\&say\_hello\_in\_english>和smart\_device<\&say\_hello\_in\_spanish>是不同的类型,所以它们的函数指针实例化不同。这可以很容易确定:

\begin{cpp}
static_assert(!std::is_same_v<decltype(w1), decltype(w2)>);
\end{cpp}

另一方面,若用std::unique\_ptr<device>替换auto,那么w1和w2是指向基类device的智能指针,因此具有相同的类型:

\begin{cpp}
std::unique_ptr<device> w1 =
	std::make_unique<smart_device<&say_hello_in_english>>();
w1->output();

std::unique_ptr<device> w2 =
	std::make_unique<smart_device<&say_hello_in_spanish>>();
w2->output();

static_assert(std::is_same_v<decltype(w1), decltype(w2)>);
\end{cpp}

虽然这个例子使用的是指向函数的指针,但也可以假设一个类似的例子,用于指向成员函数的指针。前面的例子可以转换为以下(仍然使用相同的基类设备)方式:

\begin{cpp}
template <typename Command, void (Command::*action)()>
struct smart_device : device
{
	smart_device(Command& command) : cmd(command) {}
	
	void output() override
	{
		(cmd.*action)();
	}
private:
	Command& cmd;
};

struct hello_command
{
	void say_hello_in_english()
	{
		std::cout << "Hello, world!\n";
	}

	void say_hello_in_spanish()
	{
		std::cout << "Hola mundo!\n";
	}
};
\end{cpp}

这些类的使用方式如下所示:

\begin{cpp}
hello_command cmd;

auto w1 = std::make_unique<
	smart_device<hello_command,
		&hello_command::say_hello_in_english>>(cmd);
w1->output();

auto w2 = std::make_unique<
	smart_device<hello_command,
		&hello_command::say_hello_in_spanish>>(cmd);
w2->output();
\end{cpp}

C++17中,引入了一种指定非类型模板参数的新形式,使用auto(包括auto*和auto\&)或decltype(auto)来代替类型名,编译器可以从作为实参提供的表达式中推断参数的类型。若派生的类型不允许用于非类型模板参数,编译器将生成错误的类型。让我们来看一个例子:

\begin{cpp}
template <auto x>
struct foo
{ /* … */ };
\end{cpp}

这个类模板可以这样用:

\begin{cpp}
foo<42> f1;   // foo<int>
foo<42.0> f2; // foo<double> in C++20, error for older
              // versions
foo<"42"> f3; // error
\end{cpp}

第一个例子中,对于f1,编译器将实参的类型推断为int。第二个例子中,对于f2,编译器将类型推断为double,但需要支持C++20。以前的C++版本中,这一行会产生错误。C++20之前,浮点类型不允许作为非类型模板参数。所以,最后一行会出错,因为“42”是一个字符串字面量,不能用作非类型模板参数。

然而,最后一个例子可以在C++20中通过将字面值字符串,包装在结构字面值类中来解决。这个类将字符串字面值的字符存储在一个固定长度的数组中:

\begin{cpp}
template<size_t N>
struct string_literal
{
	constexpr string_literal(const char(&str)[N])
	{
		std::copy_n(str, N, value);
	}
	char value[N];
};
\end{cpp}

前面的foo类模板也需要修改,需要显式使用string\_literal,而不是auto:

\begin{cpp}
template <string_literal x>
struct foo
{
};
\end{cpp}

C++20中,foo<"42"> f;的声明将在编译时通过。

auto也可以与非类型模板参数包一起使用,类型为每个模板参数独立推导。模板参数的类型不需要相同:

\begin{cpp}
template<auto... x>
struct foo
{ /* ... */ };

foo<42, 42.0, false, 'x'> f;
\end{cpp}

本例中,编译器将模板参数的类型分别推断为int、double、bool和char。

第三类也是最后一类模板参数是双重模板参数。

\subsubsection{2.4.3\hspace{0.2cm}双重模板参数}

尽管这个名字有点奇怪,指的是一类模板参数,这些参数本身就是模板。这些参数可以像类型模板参数一样指定,带或不带名称,带或不带默认值,以及作为带或不带名称的参数包。C++17起,关键字class和typename都可以用来引入双重模板参数。此版本之前,只能使用class关键字。

为了展示双重模板参数的使用,先来看一下下面两个类模板:

\begin{cpp}
template <typename T>
class simple_wrapper
{
public:
	T value;
};

template <typename T>
class fancy_wrapper
{
public:
	fancy_wrapper(T const v) :value(v)
	{
	}

	T const& get() const { return value; }
	
	template <typename U>
	U as() const
	{
		return static_cast<U>(value);
	}
private:
	T value;
};
\end{cpp}

simple\_wrapper类是一个简单的类模板,保存类型模板参数T的值。另一方面,fancy\_wrapper是一个更复杂的包装器实现,隐藏了包装的值,并公开了用于数据访问的成员函数。接下来,我们实现一个名为wrapped\_pair的类模板,包含两个包装类型的值。这可以是simple\_wrapper,fancy\_wrapper,或任何类似的类型:

\begin{cpp}
template <typename T, typename U,
		  template<typename> typename W = fancy_wrapper>
class wrapping_pair
{
public:
	wrapping_pair(T const a, U const b) :
		item1(a), item2(b)
	{
	}

	W<T> item1;
	W<U> item2;
};
\end{cpp}

wrapping\_pair类模板有三个参数。前两个是类型模板形参,名为T和U。第三个形参是模板模板参数为W,有一个默认值,即fancy\_wrapper类型。可以使用这个类模板,如下所示:

\begin{cpp}
wrapping_pair<int, double> p1(42, 42.0);
std::cout << p1.item1.get() << ' '
		  << p1.item2.get() << '\n';

wrapping_pair<int, double, simple_wrapper> p2(42, 42.0);
std::cout << p2.item1.value << ' '
		  << p2.item2.value << '\n';
\end{cpp}

这个例子中,p1是一个wrapping\_pair对象包含两个值,一个int和一个double,每个值都包装在一个fancy\_wrapper对象中。这没有显式指定,其为双重模板参数的默认值。另一方面,p2也是一个wrapping\_pair对象,也包含一个int和一个double,但是由一个simple\_wrapper对象包装的,该对象在模板实例化中进行了显式指定。

这个例子中,我们看到了默认模板参数的使用。

\subsubsection{2.4.4\hspace{0.2cm}默认模板参数}

默认模板参数的指定与默认函数参数类似,在参数列表中的等号后面。以下规则适用于默认模板参数:

\begin{itemize}
\item 
可以与任何类型的模板参数一起使用,参数包除外。

\item 
若为类模板、变量模板或类型别名的模板参数指定了默认值,则所有后续模板参数也必须具有默认值。若最后一个参数是模板参数包,则是例外。

\item 
若在函数模板中为模板参数指定了默认值,则后续模板参数也不限于具有默认值。

\item 
函数模板中,只有当参数包有默认实参或者其值可以由编译器从函数实参推导出来时,形参包才可以后面跟更多的类型参数。

\item 
不允许在友元类模板的声明中出现。

\item 
只有当友元函数模板的声明也有定义,并且在同一个翻译单元中没有其他函数声明时,才允许在友元函数模板的声明中使用。

\item 
不允许在函数模板或成员函数模板的显式特化的声明或定义中出现。
\end{itemize}

下面展示了使用默认模板参数的示例:

\begin{cpp}
template <typename T = int>
class foo { /*...*/ };

template <typename T = int, typename U = double>
class bar { /*...*/ };
\end{cpp}

声明类模板时,带默认实参的模板参数不能后面跟着不带默认实参的形参,但此限制不适用于函数模板:

\begin{cpp}
template <typename T = int, typename U>
class bar { }; // error

template <typename T = int, typename U>
void func() {} // OK
\end{cpp}

一个模板可以有多个声明(但只有一个定义),来自所有声明和定义的默认模板参数可以进行合并(与合并默认函数参数的方式相同):

\begin{cpp}
template <typename T, typename U = double>
struct foo;

template <typename T = int, typename U>
struct foo;

template <typename T, typename U>
struct foo
{
	T a;
	U b;
};
\end{cpp}

语义上等价于:

\begin{cpp}
template <typename T = int, typename U = double>
struct foo
{
	T a;
	U b;
};
\end{cpp}

但这些具有不同默认模板参数的多个声明不能按任何顺序提供,前面提到的规则仍然适用。若类模板声明的第一个形参有默认实参,而后面的形参没有默认实参即是非法:

\begin{cpp}
template <typename T = int, typename U>
struct foo; // error, U does not have a default argument

template <typename T, typename U = double>
struct foo;
\end{cpp}

对默认模板实参的另一个限制是,同一个模板形参不能在同一个作用域中赋予多个默认值。因此,下面的例子将在编译时报错:

\begin{cpp}
template <typename T = int>
struct foo;

template <typename T = int> // error redefinition
                            // of default parameter
struct foo {};
\end{cpp}

当默认模板实参使用来自类名时,成员访问限制会在声明时进行检查,而不是在模板实例化时:

\begin{cpp}
template <typename T>
struct foo
{
	protected:
	using value_type = T;
};

template <typename T, typename U = typename T::value_type>
struct bar
{
	using value_type = U;
};

bar<foo<int>> x;
\end{cpp}

当x变量定义时,bar类模板即会实例化,但是foo::value\_type的类型的定义是protected,因此不能在foo之外使用。结果是在bar类模板的声明处出现编译错误。

了解了这些内容后,我们先结束模板参数的话题。下一节中我们将探讨的是模板实例化,是根据模板定义和一组模板参数创建函数、类或变量的新定义。








