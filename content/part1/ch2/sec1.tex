\section{定义函数模板}
函数模板的定义方式与常规函数类似,只是函数声明之前是关键字template,尖括号之间是模板参数列表。下面是一个简单的函数模板示例:

\begin{cpp}
template <typename T>
T add(T const a, T const b)
{
	return a + b;
}
\end{cpp}

这个函数有两个参数a和b,为相同的T类型。该类型列在模板参数列表中,通过关键字typename或class(前者在本例中使用,本书中也使用)引入。这个函数所做的只是将两个参数相加,并返回该操作的结果,该结果应该具有相同的T类型。

函数模板只是创建实际函数的蓝图,只存在于源码中。除非在源码中显式调用,否则函数模板不会出现在编译后的可执行文件中。当编译器遇到对函数模板的调用,并且能够将提供的实参及其类型与函数模板的形参匹配时,它将根据模板和用于调用它的实参生成一个实际的函数。为了理解这一点,来看一个例子:

\begin{cpp}
auto a = add(42, 21)
\end{cpp}

这段代码中,使用两个int形参42和21调用add函数。编译器能够从所提供参数的类型推导出模板形参T,因此无需显式地提供。下面两个调用也是可能的,而且与前面的调用相同:

\begin{cpp}
auto a = add<int>(42, 21);
auto a = add<>(42, 21);
\end{cpp}

这种使用方式,将使编译器将生成以下函数(不同编译器生成的实际代码可能不同):

\begin{cpp}
int add(const int a, const int b)
{
	return a + b;
}
\end{cpp}

若将调用方式更改为以下形式,则显式地为模板形参T提供实参short:

\begin{cpp}
auto b = add<short>(42, 21);
\end{cpp}

这种情况下,编译器将生成该函数的另一个实例化,使用short而不是int。这个新的实例化会如下所示:

\begin{cpp}
short add(const short a, const int b)
{
	return static_cast<short>(a + b);
}
\end{cpp}

若两个参数的类型不明确,编译器将无法自动推断出它们的类型。下面的调用就是这种情况:

\begin{cpp}
auto d = add(41.0, 21);
\end{cpp}

这个例子中,41.0是一个double类型,而21是一个int类型。add函数模板有两个相同类型的形参,因此编译器无法将其与提供的实参进行匹配,并将报错。为了避免这种情况,并且假设希望模板为double实例化,这时必须显式地指定类型:

\begin{cpp}
auto d = add<double>(41.0, 21);
\end{cpp}

只要两个实参具有相同的类型,并且加法运算符可用于实参的类型,就可以按照前面所示的方式调用函数模板add。若加法操作符不可用,则编译器将无法生成实例化,即使模板参数已正确解析:

\begin{cpp}
class foo
{
	int value;
public:
	explicit foo(int const i):value(i)
	{ }
	
	explicit operator int() const { return value; }
};

auto f = add(foo(42), foo(41));
\end{cpp}

这种情况下,编译器将报错,即没有为类型为foo的参数找到二进制加法操作符。当然,对于不同的编译器,实际的消息是不同的,所有错误都是如此。为了能够对foo类型的参数调用add,必须重载此类型的加法操作符。可能的实现如下所示:

\begin{cpp}
foo operator+(foo const a, foo const b)
{
	return foo((int)a + (int)b);
}
\end{cpp}

目前为止,我们看到的所有例子都是只有一个模板形参。但当模板可以有任意数量的参数,甚至可以有可变数量的参数(后一个主题将在第3章中讨论)。下面的函数是一个有两个类型模板形参的函数模板:

\begin{cpp}
template <typename Input, typename Predicate>
int count_if(Input start, Input end, Predicate p)
{
	int total = 0;
	for (Input i = start; i != end; i++)
	{
		if (p(*i))
			total++;
	}
	return total;
}
\end{cpp}

此函数接受两个指向范围和谓词的开始和结束的输入迭代器,并返回范围内与谓词匹配的元素数量。这个函数,至少在概念上,非常类似于标准库中<algorithm>头文件中的std::count\_if函数,开发者应该始终倾向于使用标准算法,而不是自己手工实现。但对于本主题而言,这个函数是一个很好的示例,可以帮助理解模板的工作方式。

可以这样使用count\_if函数:

\begin{cpp}
int main()
{
	int arr[]{ 1,1,2,3,5,8,11 };
	int odds = count_if(
				std::begin(arr), std::end(arr),
				[](int const n) { return n % 2 == 1; });
	std::cout << odds << '\n';
}
\end{cpp}

同样,不需要显式地指定类型模板形参的实参(输入迭代器的类型和一元谓词的类型),因为编译器能够从调用中进行推断。

关于函数模板还有很多东西需要学习,本节仅提供了使用方面的介绍。现在来了解定义类模板的基础知识。


















