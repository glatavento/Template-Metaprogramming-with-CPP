\section{可变参数别名模板}
所有可以模板化的东西也可以变参数化,别名模板是一组类型的别名(另一个名称)。可变别名模板是具有可变数量模板参数的类型族的名称。编写别名模板会相当简单的,先来看一个例子:

\begin{cpp}
template <typename T, typename... Args>
struct foo
{
};

template <typename... Args>
using int_foo = foo<int, Args...>;
\end{cpp}

类模板foo是可变参数的,并且至少接受一个类型模板参数。int\_foo只是一个名称,用于从foo类型实例化的类型,并且int作为第一个类型模板参数的模板:

\begin{cpp}
foo<double, char, int> f1;
foo<int, char, double> f2;
int_foo<char, double> f3;
static_assert(std::is_same_v<decltype(f2), decltype(f3)>);
\end{cpp}

这段代码中,f1和f2和f3是不同foo类型的实例,都是从foo的不同模板参数集实例化的。然而,f2和f3是同一类型的实例,所以int\_foo<char, double>是foo<int, char, double>类型的别名。

前面介绍了一个类似的例子,尽管有点复杂。标准库包含一个名为std::integer\_sequence的类模板,表示一个编译时的整数序列,以及一堆别名模板,从而帮助创建各种类型的此类整数序列。虽然这是一个简化的代码段,在概念上的实现如下所示:

\begin{cpp}
template<typename T, T... Ints>
struct integer_sequence
{};

template<std::size_t... Ints>
using index_sequence = integer_sequence<std::size_t,
								        Ints...>;

template<typename T, std::size_t N, T... Is>
struct make_integer_sequence :
	make_integer_sequence<T, N - 1, N - 1, Is...>
{};

template<typename T, T... Is>
struct make_integer_sequence<T, 0, Is...> :
	integer_sequence<T, Is...>
{};

template<std::size_t N>
using make_index_sequence = make_integer_sequence<std::size_t,
												  N>;

template<typename... T>
using index_sequence_for =
	make_index_sequence<sizeof...(T)>;
\end{cpp}

这里有三个别名模板:

\begin{itemize}
\item
index\_sequence, 为size\_t类型创建integer\_sequence;这是一个可变别名模板。

\item
index\_sequence\_for, 从参数包中创建integer\_sequence;这也是一个可变别名模板。

\item
make\_index\_sequence, 为size\_t类型创建了integer\_sequence,值为0,1,2,…,N-1。与前面的模板不同,这不是可变参数模板的别名。
\end{itemize}

本章要讨论的最后一个主题是可变变量模板。










