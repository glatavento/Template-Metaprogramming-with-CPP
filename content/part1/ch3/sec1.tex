\section{可变参数模板的需求}
最著名的C/C++函数之一是printf,将格式化输出写入标准输出流。实际上,I/O库中有一系列用于写入格式化输出的函数,其中还包括fprintf(写入文件流)、sprint和snprintf(写入字符缓冲区)。这些函数很相似,接受一个定义输出格式的字符串和可变数量的参数。C++语言提供了一种方法,用可变数量的参数来编写自己的函数。下面是一个例子,函数可以接受一个或多个参数,并返回最小值:

\begin{cppcode}
#include<stdarg.h>

int min(int count, ...)
{
	va_list args;
	va_start(args, count);
	
	int val = va_arg(args, int);
	for (int i = 1; i < count; i++)
	{
		int n = va_arg(args, int);
		if (n < val)
			val = n;
	}

	va_end(args);
	
	return val;
}

int main()
{
	std::cout << "min(42, 7)=" << min(2, 42, 7) << '\n';
	std::cout << "min(1,5,3,-4,9)=" <<
				  min(5, 1, 5, 3, -4,
	           9) << '\n';
}
\end{cppcode}

这个实现特定于int类型,也可以编写一个类似的函数,即函数模板:

\begin{cppcode}
template <typename T>
T min(int count, ...)
{
	va_list args;
	va_start(args, count);
	
	T val = va_arg(args, T);
	for (int i = 1; i < count; i++)
	{
		T n = va_arg(args, T);
		if (n < val)
		val = n;
	}

	va_end(args);
	
	return val;
}

int main()
{
	std::cout << "min(42.0, 7.5)="
		      << min<double>(2, 42.0, 7.5) << '\n';
	std::cout << "min(1,5,3,-4,9)="
  	          << min<int>(5, 1, 5, 3, -4, 9) << '\n';
}
\end{cppcode}

这样的代码,无论是否通用,都有几个缺点:

\begin{itemize}
  \item 需要使用几个宏:va_list(提供对其他宏的信息)、va_start(启动参数的迭代)、va_arg(提供下一个参数)和va_end(停止参数的迭代)。
  \item 即使传递给函数的参数的数量和类型在编译时已知,求值部分也发生在运行时。
  \item 以这种方式实现的可变函数类型不安全。va_macros执行低内存操作,类型转换在运行时的va_arg中完成,会导致运行时异常。
  \item 可变参函数需要以某种方式指定变量参数的数量。min函数的实现中,第一个参数表示参数的数量。类似printf的函数接受一个格式化字符串,从中确定预期参数的数量。例如,printf函数计算并忽略额外的参数(若提供的参数多于格式化字符串中指定的数量),但若提供的参数较少,则为未定义行为。
\end{itemize}

此外,C++11之前,只有函数是可变参的,有些类也可以从拥有可变参数数量的数据成员中受益。典型的例子是tuple类和variant类,前者表示固定大小的异构值集合,后者是类型安全的联合体。

可变参数模板有助于解决所有这些问题,在编译时求值类型安全,不需要宏,不需要显式指定参数的数量,可以编写可变参数函数模板和可变参数类模板。此外,也有可变参数变量模板和可变参数别名模板。

下一节,我们开始研究可变参数函数模板。






























