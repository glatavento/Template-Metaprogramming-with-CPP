\section{编写模板}
本节中,我们将从三个简单的示例开始编写C++模板,每个示例对应前面给出的代码片段。

前面max函数的模板如下所示:

\begin{cpp}
template <typename T>
T max(T const a, T const b)
{
	return a > b ? a : b;
}
\end{cpp}

这里的类型名称(例如int或double)被T(代表类型)取代。T称为类型模板形参,并通过语法template<typename T>或typename<class T>引入。T是一个参数,可以使用任意的名称。

此时,源码中的模板只是一个蓝图,编译器将根据其使用情况生成代码,并将为使用模板的每种类型实例化一个函数重载:

\begin{cpp}
struct foo{};
int main()
{
	foo f1, f2;
	max(1, 2); // OK, compares ints
	max(1.0, 2.0); // OK, compares doubles
	max(f1, f2); // Error, operator> not overloaded for
	             // foo
}
\end{cpp}

代码中,首先使用两个整数调用max,operator>可用于int类型。这将生成一个重载int max(int const a, int const b)。其次,用两个double值调用max,这也没关系,因为operator>适用于double值。因此,编译器将生成另一个重载,double max(double const a, double const b)。第三次调用max将生成一个编译器错误,因为foo类型没有重载operator>。

这里不讨论太多细节,使用max函数的方式如下所示:

\begin{cpp}
max<int>(1, 2);
max<double>(1.0, 2.0);
max<foo>(f1, f2);
\end{cpp}

编译器能够推断推导参数的类型,因此写上去就是是多余的。但在某些情况下,编译器无法推导参数类型,所以需要使用以下语法显式指定类型。

第二个例子涉及前一节中的函数,是处理void*参数的quicksort()实现,可以很容易地改为模板版本:

\begin{cpp}
template <typename T>
void swap(T* a, T* b)
{
	T t = *a;
	*a = *b;
	*b = t;
}

template <typename T>
int partition(T arr[], int const low, int const high)
{
	T pivot = arr[high];
	int i = (low - 1);
	
	for (int j = low; j <= high - 1; j++)
	{
		if (arr[j] < pivot)
		{
			i++;
			swap(&arr[i], &arr[j]);
		}
	}

	swap(&arr[i + 1], &arr[high]);
	
	return i + 1;
}

template <typename T>
void quicksort(T arr[], int const low, int const high)
{
	if (low < high)
	{
		int const pi = partition(arr, low, high);
		quicksort(arr, low, pi - 1);
		quicksort(arr, pi + 1, high);
	}
}
\end{cpp}

快速排序函数模板的使用与前面的非常类似,并且不需要将指针传递给回调函数了:

\begin{cpp}
int main()
{
	int arr[] = { 13, 1, 8, 3, 5, 2, 1 };
	int n = sizeof(arr) / sizeof(arr[0]);
	quicksort(arr, 0, n - 1);
}
\end{cpp}

前一节中看到的第三个例子是vector类,其模板版本如下所示:

\begin{cpp}
template <typename T>
struct vector
{
	vector();
	
	size_t size() const;
	size_t capacity() const;
	bool empty() const;
	
	void clear();
	void resize(size_t const size);
	
	void push_back(T value);
	void pop_back();
	
	T at(size_t const index) const;
	T operator[](size_t const index) const;
	
private:
	T* data_;
	size_t size_;
	size_t capacity_;
};
\end{cpp}

max函数那种情况,变化很小。类的模板声明中,元素的int类型可以使用类型模板形参T取代。实现方式如下:

\begin{cpp}
int main()
{
	vector<int> v;
	v.push_back(1);
	v.push_back(2);
}
\end{cpp}

声明变量v时必须指定元素的类型。代码中,v的元素是int,否则编译器将无法推断它们的类型。C++17中,有些情况下可以不声明类型,可将其称为\textbf{类模板参数推导}(将在第4章中讨论)。

第四个也是最后一个例子,只有类型不同的情况下声明几个变量。可以用一个模板替换所有这些变量,如下所示:

\begin{cpp}
template<typename T>
constexpr T NewLine = T('\n');
\end{cpp}

该模板的使用方式如下:

\begin{cpp}
int main()
{
	std::wstring test = L"demo";
	test += NewLine<wchar_t>;
	std::wcout << test;
}
\end{cpp}

本节中的示例表明,无论模板表示函数、类还是变量,声明和使用模板的语法都是相同的。这将引导我们进入下一节,通过模板类型而来了解模板。














