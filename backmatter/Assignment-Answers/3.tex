\subsubsection{习题 1}

什么是可变参数模板,这种模板有什么用?

\subsubsection{参考答案}

可变参数模板是具有可变数量参数的模板,不仅允许编写参数数量可变的函数,还允许我们编写类模板、变量模板和别名模板。与其他方法(如使用va_宏)不同,它们是类型安全的,不需要宏,也不需要显式地指定参数的数量。


\subsubsection{习题 2}

什么是参数包?

\subsubsection{参考答案}

有两种参数包:模板参数包和函数参数包。前者是接受零个、一个或多个模板实参的模板形参。后者是接受零个、一个或多个函数参数的函数参数。

\subsubsection{习题 3}

什么是可以展开参数包的上下文?

\subsubsection{参考答案}

参数包可以在多种上下文中展开,如下所示:模板形参列表、模板实参列表、函数形参列表、圆括号括起来的初始化式、大括号括起来的初始化式、基形说明符和成员初始化式列表、折叠表达式、using声明、Lambda捕获、sizeof...操作符、对齐说明符和属性列表。

\subsubsection{习题 4}

什么是折叠表达式?

\subsubsection{参考答案}

折叠表达式是一种包含参数包的表达式,将参数包中的元素折叠(或减少)到二进制运算符上。

\subsubsection{习题 5}

使用折叠表达式有什么好处?

\subsubsection{参考答案}

折叠表达式的好处包括编写的代码更少、更简单,模板实例化更少,具有更快的编译时间和潜在的更快的代码,因为多个函数只使用一个表达式。












