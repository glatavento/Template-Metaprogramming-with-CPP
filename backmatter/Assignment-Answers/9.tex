\subsubsection{习题 1}

什么是范围?

\subsubsection{参考答案}

范围是元素序列的抽象,用开始迭代器和结束迭代器定义。开始迭代器指向序列中的第一个元素,结束迭代器指向序列的最后一个元素。


\subsubsection{习题 2}

什么是范围库中的视图?

\subsubsection{参考答案}

C++范围库中的视图,也称为范围适配器,是实现一种算法的对象,该算法以一个或多个范围作为输入。可能还有其他参数,并返回一个经过调整的范围。视图可惰性求值,所以可以直到其元素迭代时,才进行适配。

\subsubsection{习题 3}

什么是约束算法?

\subsubsection{参考答案}

约束算法是现有标准库算法的实现,在C++20范围库中,其之所以“受约束”,是因为它们的模板参数使用C++20的概念进行约束。这些算法中,值范围接受单个范围参数,而不是需要begin-end迭代器对进行指定,但也存在接受迭代器-哨兵对的重载。

\subsubsection{习题 4}

什么是哨兵?

\subsubsection{参考答案}

哨点是对结束迭代器的抽象。这使得结束迭代器的类型可能与范围迭代器的类型不同,哨兵不能解除引用或增加。当对范围尽头的测试依赖于某些可变(动态)条件,并且直到某些事情发生(条件变为false)时,才知道处于范围末尾时,哨兵是有用的。

\subsubsection{习题 5}

如何检查哨兵类型是否与迭代器类型相对应?

\subsubsection{参考答案}

可以使用<iterator>头文件中的std::sentinel\_for概念来检查哨兵类型是否可以与迭代器类型一起使用。












