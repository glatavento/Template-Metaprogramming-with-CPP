\subsubsection{习题 1}

CRTP可以解决哪些问题?

\subsubsection{参考答案}

奇异迭代模板模式(CRTP)通常用于解决,诸如向类型添加公共功能和避免代码重复、限制类型可以实例化的次数或实现复合设计模式等问题。


\subsubsection{习题 2}

什么是Mixins?其目的是什么?

\subsubsection{参考答案}

Mixins是一些小类型,通过它们补充的类继承来为其他类添加功能。这与CRTP模式相反。

\subsubsection{习题 3}

什么是类型擦除?

\subsubsection{参考答案}

类型擦除是用于描述从类型中删除信息的模式的术语,从而使不相关的类型能够以通用的方式进行处理。虽然类型擦除的形式可以通过void指针或多态性实现,但真正的类型擦除模式是通过模板实现的(C++中)。

\subsubsection{习题 4}

什么是标记分派,它的替代方案是什么?

\subsubsection{参考答案}

标记分派是一种技术,使我们能够在编译时选择一个或另一个函数重载。尽管标记调度本身是std::enable\_if和SFINAE的替代方案,但它也有替代方案。在C++17中是constexpr if,在C++20中是概念。

\subsubsection{习题 5}

表达式模板是什么?可以在哪里使用它们?

\subsubsection{参考答案}

表达式模板是一种元编程技术,支持在编译时对计算进行惰性计算。这种技术的好处是,可避免在运行时执行低效的操作,代价是代码更加复杂,难以理解。表达式模板通常用于实现线性代数库。












