这里是本书的结尾。模板并不是C++编程中最简单的部分,人们通常觉得它很难或可怕。然而,模板在C++代码中大量使用,无论编写什么样的代码,很可能每天都要使用模板。我们从学习模板是什么,以及我们为什么需要它们开始这本书。然后,学习了如何定义函数模板、类模板、变量模板和别名模板。

学习了模板参数、特化和实例化。第3章中,学习了具有可变参数数量的模板,这些模板称为可变参数模板。第4章专门介绍更高级的模板概念,如名称绑定、递归、参数推导和转发引用。

然后,第5章学习了类型特征、SFINAE和constexpr if的使用,并探索了标准库中可用的类型特征集合。第6章介绍C++20标准中的概念和约束,学习了如何以不同的方式为模板参数指定约束,以及如何定义概念,以及与它们相关的一切。还探讨了标准库中可用概念的集合。

本书的最后一部分中,我们重点介绍了如何将模板用于实际目的。首先,第7章探讨了一系列模式和习惯用法,例如CRTP、混入、类型擦除、标记分派、表达式模板和类型列表。然后,第8章学习了容器、迭代器和算法,这些是标准模板库的核心,并编写了一些属于我们自己的容器和迭代器。最后,第9章专门介绍了C++20范围库,学习了范围、范围适配器,以及约束算法。

到此为止,已经完成了使用C++模板学习元编程的旅程,但这个学习过程并没有到此结束。一本书只能为你提供学习元编程所必需的信息,以一种易于理解和理解的方式组织起来。不过,只读一本书而不实践,会使学到的东西无用。各位读者现在的任务,就是把你从这本书中学到的知识运用到工作中、在学校里或在家里。因为只有通过练习,才能真正掌握C++语言和使用模板的元编程,以及其他任何其他技能。

为实现成为高产的C++模板的目标,这本书会自证其价值。在撰写本书的时候,我也尝试在简单和有意义之间寻求平衡,这样读者们就能更容易地学习和了解一些较为困难的知识点,但愿我做到了。

感谢你阅读这本书,并祝各位在实践中好运!


